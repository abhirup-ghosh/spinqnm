The LIGO Scientific Collaboration~\cite{lsc} and the Virgo
Collaboration~\cite{Virgo} have recently announced their catalogue of
gravitational-wave (GW) signals from the first
half of the third observing run (O3a)~\cite{Abbott:2020niy}. Combined
with the first and second observing-run catalogues~\cite{LIGOScientific:2018mvr}, the Advanced LIGO detectors at Hanford,
Washington and Livingston, Louisiana~\cite{Aasi:2014mqd},
and the Advanced Virgo detector in Cascina,
Italy~\cite{TheVirgo:2014hva} have now detected $50$ GW
events from the merger of compact objects like neutron stars and/or
black holes (BHs). Alongside independent claims of
detections~\cite{Nitz:2018imz,Nitz:2019hdf,Venumadhav:2019lyq,Zackay:2019btq}, these results have firmly established the field of GW astronomy, five years after the first detection of a GW passing through Earth, GW150914~\cite{Abbott:2016blz}.

The observation of GWs has had significant astrophysical and cosmological
implications~\cite{TheLIGOScientific:2016htt,GBM:2017lvd,Monitor:2017mdv,Abbott:2017xzu}. It
has also allowed us to probe fundamental
physics and test predictions of Einstein's theory of General Relativity
(GR) in the previously unexplored highly-dynamical, and strong field
regime~\cite{TheLIGOScientific:2016src,Abbott:2018lct,LIGOScientific:2019fpa,Abbott:2020jks}. In GR, a binary black hole (BBH) system is described by
three distinct phases: an early \textit{inspiral}~\cite{Blanchet:2013haa}, where the two
compact objects spiral in loosing energy because of the emission of GWs, a \textit{merger}~\cite{Pretorius:2005gq,Campanelli:2005dd,Baker:2005vv} marked by the
formation of a common apparent horizon, and a \textit{ringdown}~\cite{Vishveshwara:1970zz,Vishveshwara:1970cc,Press:1971wr,Chandrasekhar:1975zza,Detweiler:1980gk}, during which the newly formed remnant object settles down to a Kerr BH emitting quasi-normal-modes (QNMs) (i.e., damped oscillations with specific, discrete frequencies and decay times).

The LIGO-Virgo collaborations have released companion papers detailing their results
of tests of GR for GW150914 \cite{TheLIGOScientific:2016src}, and for several GW events of
the two transient catalogues (TC): GWTC-1 \cite{LIGOScientific:2018mvr,LIGOScientific:2019fpa}  and GWTC-2 \cite{Abbott:2020niy,Abbott:2020jks}.
The results include tests of GW generation and source dynamics, where bounds are placed on
parameterized deviations in the post-Newtonian (PN) coefficients describing
the early inspiral, and
phenomenological coefficients describing the intermediate (plunge) and
merger regimes of coalescence~\cite{Arun:2006hn,Arun:2006yw,Agathos:2013upa,Abbott:2018lct}; tests of GW
propagation, which assume a generalized dispersion relation and place
upper bounds on the Compton wavelength and, consequently, the mass of
the graviton~\cite{Abbott:2017vtc,Samajdar:2017mka}, and tests of the
polarization of gravitational radiation using a
multi--GW-detector network~\cite{Abbott:2017oio,Isi:2017fbj}. The GWTC-1/2 papers also
check for consistency between different portions of the signal using estimates for the predicted mass and spin of the remnant
object~\cite{Ghosh:2016xx,Ghosh:2017gfp,TheLIGOScientific:2016src}, and
consistency of the residuals with detector noise~\cite{Ghonge:2020suv,LIGOScientific:2019fpa}. None of these tests report any
departure from the predictions of GR.

The first paper on tests of GR by the LIGO Collaboration~\cite{TheLIGOScientific:2016src} also provided us with the first measurement of the dominant damped-oscillation signal in the ringdown stage of a BBH
coalescence, and more recently a similar measurement was made with the high-mass event GW190521~\cite{Abbott:2020tfl,Abbott:2020mjq}. The set of available measurements was greatly expanded in the latest LIGO-Virgo O3a testing GR paper~\cite{Abbott:2020jks} where a comprehensive analysis of the properties of the remnant object, including the ringdown stage, was reported for tens of GW events. The no-hair conjecture in GR~\cite{Carter:1971zc,Hansen:1974zz} states that an (electrically neutral) astrophysical BH is completely described by two observables: mass and spin. One
consequence of the no-hair conjecture is that the (complex) QNM
frequencies of gravitational radiation emitted by a perturbed isolated
BH are uniquely determined by its mass and spin. Hence
a test of the no-hair conjecture would involve checking for
consistency between estimates of mass and spin of the remnant object
across multiple QNM frequencies~\cite{Dreyer:2003bv,Berti:2005ys}. An inconsistency would either
indicate a non-BH nature of the remnant object, or an incompleteness
of GR as the underlying theory of gravity.

The consistency between the post-merger signal and the least damped QNM was first demonstrated in
Ref.~\cite{TheLIGOScientific:2016src} for GW150914, and confirmed through several independent analyses~\cite{Brito:2018rfr,Carullo:2019flw,Isi:2019aib,CalderonBustillo:2020tjf}. This was later extended to include
overtones in Refs.~\cite{Giesler:2019uxc,Isi:2019aib,Bhagwat:2019dtm,Forteza:2020hbw,Abbott:2020jks}. Consistency of the late-time waveform with a single QNM is a test of the ringdown of
a BBH coalescence, but not necessarily a test of the no-hair
conjecture, which requires the measurement of (at least) two QNMs (BH
spectroscopy), and consistency between them~\cite{Dreyer:2003bv,Carullo:2018sfu,Carullo:2018gah,Isi:2019aib,Bhagwat:2019bwv,Cabero:2019zyt,Maselli:2019mjd}. The nature of the remnant object has also been explored through tests of BH thermodynamics, including the Hawking's area
theorem~\cite{Cabero:2017avf,Isi:2020tac}, the Bekenstein-Hod universal bound~\cite{Carullo:2021yxh}, the BH area quantization~\cite{Foit:2016uxn,Laghi:2020rgl}, or through search for echos in the post-merger signal~\cite{Nielsen:2018lkf,Tsang:2019zra,Lo:2018sep,Abedi:2018npz,Abedi:2020sgg,Testa:2018bzd}.  None of these tests have found evidence for non-BH nature of the remnant
object (as described in GR) in LIGO-Virgo BBH observations.

Ground-based detectors, such as LIGO and Virgo, are most sensitive to
stellar-mass BH binaries that merge near the
minimum of their sensitivity band ($\sim 100$ Hz). As a consequence, the
remnant object \textit{rings down} at frequencies where the sensitivity is worse, leaving
very little signal-to-noise ratio (SNR) in the
post-merger signal. Most of the tests mentioned above restrict the data to this post-merger signal, which significantly reduces parameter-estimation capabilities. However, added to a lack of SNR is the ambiguity in clearly defining a start time for the ringdown phase (discussed, for example, in Refs.~\cite{Berti:2007fi,Baibhav:2017jhs,Bhagwat:2017tkm}). In some tests~\cite{Carullo:2018gah,Carullo:2019flw}, the ringdown start time has been left as a
free parameter to be estimated directly from the data. Whereas, in other cases, the ringdown start time is predicted from corresponding full-signal parameter-estimation analyses (see \texttt{PyRING} analysis in Ref.~\cite{Abbott:2020jks}). Uncertainties
in estimates of the ringdown start-time, as well as an overall lack of
SNR in the post-merger signal, given typical sensitivities of
ground-based detectors, result in significant statistical uncertainties
in the measurement of the QNM frequencies. And hence, one might want to look at alternate methods to measure QNMs using information from as much as the signal as possible.

An independent approach to BH spectroscopy, based on the full-signal
analysis, was introduced in Ref.~\cite{Brito:2018rfr}
(henceforth referred to as \paperone). Unlike methods that focus only
on the post-merger signal, it employs the
complete inspiral-merger-ringdown
(IMR) waveform to measure the complex QNM
frequencies. This gives access to the full SNR of the signal,
reducing measurement uncertainties. Moreover, the definition of the
ringdown start time is built into the merger-ringdown model and does not need to be
either left as an additional free parameter or fixed using alternate
definitions. While \paperone presented the method and tested it for non-spinning BBHs, here we extend the analysis to the more
realistic astrophysical case in which BHs carry spin. Furthermore,
the IMR waveform model used in this paper is more accurate than what employed in \paperone,
because it contains higher-order corrections in PN theory and it was calibrated to a much larger set
of numerical-relativity (NR) waveforms~\cite{Bohe:2016gbl}. All astrophysical BHs are expected to be spinning,
and ignoring effects of spin has been shown to introduce systematic biases in the measurement of the
source properties.

The rest of the paper is organized as follows. Section~\ref{sec:model} describes our parameterized IMR
waveform model. In Sec.~\ref{sec:method}, we define our framework to test GR, notably how we can measure
complex QNM frequencies with our parameterized model within a Bayesian formalism, and validate it using 
several synthetic-signal injection studies in Gaussian noise in Sec.~\ref{sec:results}. Then, in Sec.~\ref{sec:results_lvc},
we apply our method on real GW events in LIGO and Virgo data. In particular, we analyze
the GWTC 1/2  events and obtain the most stringent constraints on the dominant (complex) QNM frequecies. Finally, in Sec.~\ref{sec:discussion} we provide a summary of
our results and discuss future developments.

