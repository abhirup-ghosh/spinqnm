%\documentclass[twocolumn,prd,superscriptaddress,amsfonts,amssymb,amsmath,preprintnumbers,nofootinbib]{revtex4-1}
\documentclass[twocolumn,prd,aps,superscriptaddress,preprintnumbers,tightenlines,showpacs,nofootinbib,eqsecnum,amsfonts,amsmath]{revtex4-1}
\usepackage{epsfig}
\usepackage{graphics}
\usepackage{graphicx}
\usepackage{bm}
\usepackage[dvipsnames]{xcolor}
\usepackage{bm}
\usepackage{times}
\usepackage{xspace}
\usepackage[varg]{txfonts}
\usepackage[normalem]{ulem} % To get strikethrough (\sout)
\usepackage[colorlinks]{hyperref}
\usepackage[caption=false]{subfig}
\usepackage{booktabs}
\usepackage{url}
\usepackage{float}
\usepackage[bottom]{footmisc}
\usepackage{lineno}
%\linenumbers

\definecolor{LinkColor}{rgb}{0.75, 0, 0}
\definecolor{CiteColor}{rgb}{0, 0.5, 0.5}
\definecolor{UrlColor}{rgb}{0, 0, 0.75}
\hypersetup{linkcolor=LinkColor}
\hypersetup{citecolor=CiteColor}
\hypersetup{urlcolor=UrlColor}
\usepackage{perpage}
\MakePerPage{footnote}

\newcommand{\paperone}{Paper~I\xspace}
\newcommand{\abhi}[1]{\textcolor{red}{[\textit{AG: #1}]}}
\newcommand{\rb}[1]{\textcolor{blue}{[\textit{RB: #1}]}}
\newcommand{\ab}[1]{\textcolor{cyan}{#1}}
\newcommand{\comment}[1]{\textcolor{red}{[#1]}}

\newcommand{\h}{\mathpzc{h}}
\newcommand{\Hhat}{\hat{\mathpzc{H}}}
\newcommand{\B}{\mathpzc{B}}
\newcommand{\hlm}{\mathpzc{h}_{\ell m}}
\newcommand{\xilm}{\xi_{\ell m}}
\newcommand{\Ylm}{{Y}^{-2}_{\ell m}}
\newcommand{\Y}{{Y}^{-2}}
\newcommand{\hc}{h_\times}
\newcommand{\hp}{h_+}
\newcommand{\Fc}{F_\times}
\newcommand{\Fp}{F_+}
\newcommand{\Mf}{M_f}
\newcommand{\cA}{\mathpzc{A}}
\newcommand{\lm}{_{\ell m}}
\newcommand{\deff}{d_\mathrm{eff}}
\newcommand{\rmi}{\mathrm{i}}
\newcommand{\blambda}{\bm{\lambda}}
\newcommand{\btheta}{\bm{\theta}}
\newcommand{\bxi}{\bm{\xi}}
\newcommand{\bxigr}{\bm{\xi}_{\text{GR}}}
\newcommand{\bxingr}{\bm{\xi}_{\text{nGR}}}
\newcommand{\bzeta}{\bm{\zeta}}
\newcommand{\bs}[1]{\bm{\vec{S}_{#1}}}
\newcommand{\Mo}{M_{\odot}}
\newcommand{\FFe}{\mathrm{FF}_\mathrm{eff}}
\newcommand{\FF}{\mathrm{FF}}
\newcommand{\e}{\mathrm{e}}
\newcommand{\rhoopt}{\rho_\mathrm{opt}}
\newcommand{\rhosubopt}{\rho_\mathrm{subopt}}
\newcommand{\fqnm}{f}
\newcommand{\sigmaqnm}{\sigma}
\newcommand{\n}{\mathbf{n}}
\newcommand*{\skymapscale}{0.5}
\newcommand*{\paramestscale}{0.455}
\newcommand{\df}[1]{\delta f_{\text{#1}}}
\newcommand{\dtau}[1]{\delta \tau_{\text{#1}}}
\newcommand{\fngr}[1]{f_{\text{#1}}}
\newcommand{\taungr}[1]{\tau_{\text{#1}}}
\newcommand{\fgr}[1]{f ^{\text{GR}}_{\text{#1}}}
\newcommand{\taugr}[1]{\tau ^{\text{GR}}_{\text{#1}}}
\newcommand{\pSEOB}{\texttt{pSEOBNR}}
\newcommand{\SEOB}{\texttt{SEOBNR}}

\newcommand{\AEI}{\affiliation{Max Planck Institute for Gravitational Physics (Albert Einstein Institute), Am M\"uhlenberg 1, Potsdam 14476, Germany}}
\newcommand{\UMD}{\affiliation{Department of Physics, University of Maryland, College Park, MD 20742, USA}}

\begin{document}

%\title{Black hole spectroscopy using a complete gravitational wave signal from a binary black coalescence}


\title{\ab{Constraints on the quasi-normal mode frequencies of the LIGO-Virgo signals by making full use of gravitational-wave modeling}}

\author{Abhirup Ghosh}
\AEI
\author{Richard Brito}
\affiliation{Dipartimento di Fisica, ``Sapienza" Universit\`a di Roma $\&$ Sezione INFN Roma1, Piazzale Aldo Moro 5, 00185 Roma, Italia}
\author{Alessandra Buonanno}
\AEI
\UMD

\date{\today}


%%%%%%%%%%%%%%%%%%%%%%%
\begin{abstract}
  The no-hair conjecture in General Relativity states that the
  properties of an \ab{astrophysical} Kerr black hole (BH) are completely described by its
  mass and spin angular momentum. As a consequence, the complex
  quasi-normal-mode (QNM) frequencies of a binary black hole (BBH)
  ringdown can be uniquely determined by the mass and spin of the
  remnant object. Conversely, measurement of the QNM frequencies could
  be an independent test of the no-hair conjecture. This paper \ab{extends to spinning BHs earlier work} 
 \sout{outlines a} \ab{that proposed to} 
  test \sout{of} the no-hair conjecture by measuring the complex QNM
  frequencies of a BBH ringdown \sout{within the framework of, for the first
  time, a spinning} using inspiral-merger-ringdown waveform\ab{s} \sout{model}, thereby
  \sout{using} \ab{taking full advantage of} the entire signal power and removing dependency on the
  predicted or estimated start time of the proposed ringdown. We
  further demonstrate the robustness of the test against modified
  gravitational-wave (GW) signals with a ringdown different from what
  GR predicts for Kerr BHs. \ab{Our} method was used to analyse the
  properties of the merger remnants for the events observed by
  LIGO-Virgo in the first half of their third observing run (O3a) in
  the latest LIGO-Virgo publication. In this paper, for the first
  time, we analyse the GW events from the first \ab{(O1)} and second \ab{(O2)} LIGO-Virgo
  observing runs and provide joint constraints with published results
  from O3a. We also analyse two events from the O3a \ab{catalogue} \sout{list of events} 
  that were not considered \sout{for} \ab{in}  the initial \ab{LIGO-Virgo} analysis. Finally, the
  joint measurements of the fractional deviations in the frequency,
  $\df{220} = XX \pm$ and $\dtau{220} = YY \pm$ are the strongest
  constraints yet using this method. %Finally, we also present a investigation into possible systematic effects due to an incomplete understanding of the interferometric noise around the GW event on the results of this test.
\end{abstract}
%%%%%%%%%%%%%%%%%%%%%%%

\maketitle


\section{Introduction}\label{sec:intro}

\comment{AB: Please make the citations in the bibliography uniform (I would stick with citations 
taken from InSpires and remove the URL links). We will be submitting to Physical Review D, so please 
stick with American English.}

The LIGO Scientific Collaboration~\citep{lsc} and the Virgo
Collaboration\sout{s}~\citep{Virgo} \ab{have} recently announced their catalogue of
gravitational-wave (GW) \sout{observations} \ab{signals}~\citep{GWTC-2} from the first
half of the\sout{ir} third observing run (O3a)~\citep{O3reference}. Combined
with \sout{the confirmed observations from} the first and second observing-run \ab{catalogues}~\citep{abbott2019gwtc}, the Advanced LIGO detectors at Hanford,
Washington and Livingston, Louisiana~\citep{aasi2015characterization},
and the Advanced Virgo detector in Cascina,
Italy~\citep{acernese2014advanced} have now detected \sout{more than} $50$ GW
events from the merger of compact objects like neutron stars and/or
black holes (BHs).\sout{, collectively called compact binary coalescences
(CBCs).} Alongside independent claims of
detections~\citep{nitz20191,nitz20202,2019PhRvD.100b3007Z,2020PhRvD.101h3030V,Venumadhav_2020},
\sout{this} \ab{these results} \sout{has} \ab{have} firmly established the field of GW astronomy, five years \sout{on
from} \ab{after} the first \sout{ever direct} detection of a GW \ab{passing through the} \sout{on} Earth, \ab{notably GW150914} 
\sout{on September 14,2015}~\citep{abbott2016observation}.

\ab{The observation of GWs has} had significant astrophysical and cosmological
implications~\citep{LSC_2016astroph,gw170817_mma,gw170817_joint,gw170817_hubble}. It
has also allowed us to \sout{make statements in} \ab{probe} fundamental
physics \sout{.Specifically, LIGO-Virgo's observations have allowed us to} and 
test predictions of Einstein's theory of General Relativity 
(GR)~\citep{GR} in the previously unexplored \ab{highly-dynamical, and strong field 
regime} \sout{of highly relativistic, strong-field regimes of
gravity}~\citep{LSC_2016grtests,GW170817_TGR,gwtc1_tgr}. In GR\sout{, a CBC
involving two BHs (}a binary black hole (BBH) system\sout{)} is described \sout{in} \ab{by}
three \sout{distinct} phases: an early \textit{inspiral}, where the two
compact objects spiral in \ab{loosing energy because of the emission of GWs} \sout{and \textit{plunge} due to a back-reaction
of GW-emission}, a \textit{merger} marked by the
formation of a\sout{n} \ab{common} apparent horizon~\citep{NRpaper}, and a \sout{late-time} 
\textit{ringdown}, \sout{where} \ab{during which} the newly formed remnant object settles down
to a \sout{stable} Kerr \ab{BH} \sout{state through the emission of} \ab{emitting} \sout{an exponentially damped} 
quasi-normal-modes (QNMs) \sout{spectrum of gravitational
radiation}~\citep{vishu,earlyqnmpapers} \ab{(i.e., damped oscillations with specific, discrete 
frequencies and decay times)}.

The LIGO-Virgo collaborations \ab{have released companion papers \cite{}} detailing their \sout{latest} results 
of tests of GR  \sout{using events} \ab{for GW150914} \cite{} and \ab{for several GW events of} 
the \sout{first} two transient catalogues (TC): GWTC-1\cite{}  and GWTC-2 \cite{}. 
The results include tests of GW generation and source dynamics, where bounds are placed on
parameterized deviations in the post-Newtonian \ab{(PN)} coefficients describing
the early inspiral~\citep{earlydevelopmentpapers}, and
phenomenological coefficients describing the intermediate (plunge) and
merger regimes of coalescence~\citep{TIGERmethodspapers}; tests of GW
propagation, which assume a generalized dispersion relation and place
upper bounds on the Compton wavelength and, consequently, the mass of
the graviton~\citep{gw170104,samajdar2017projected}, and tests of the
polarization of gravitational radiation using a
multi--GW-detector network~\citep{gw170814,isi2017probing}. The \ab{GWTC-1/2} paper\ab{s} also 
check\sout{s} for consistency between different portions of the signal using estimates
\sout{of} \ab{for the} predicted mass and spin of the remnant
object~\citep{Ghosh:2016xx,Ghosh:2017gfp,LSC_2016grtests}, and
consistency of the residuals with interferometric
noise~\citep{Ghonge:2020suv,gwtc1_tgr}. None of these tests report any
departure from the predictions of GR.

\ab{The first paper on tests of GR by the LIGO Collaboration~\cite{} provided us with the 
first measurement of the dominant damped-oscillation signal in the ringdown stage of a BBH 
coalescence.} The LIGO-Virgo O3a testing GR paper~\cite{} \ab{has recently reported} 
\sout{also details results on tests} \ab{a comprehensive analysis of the properties of the 
remnant object, including the ringdown stage, for tens of GW events.}   
\sout{BH ringdown and nature of the remnant object, which is an active field
of research at the moment.} The no-hair conjecture in GR~\citep{}
states that an (electrically neutral) astrophysical BH is completely
described by two observables: mass and spin angular momentum. One
consequence of the no-hair conjecture is that the (complex) QNM
frequencies of gravitational radiation emitted by a perturbed isolated
BH \ab{are} \sout{is} uniquely determined by its mass and spin angular momentum. Hence
a test of the no-hair conjecture would involve checking for
consistency between estimates of mass and spin of the remnant object
across multiple QNM frequencies~\cite{Dreyer:2003bv}. An inconsistency would either
indicate a non-BH nature of the remnant object, or an incompleteness
of GR as the underlying theory of gravity. 

The consistency between the post-merger signal and the least damped QNM was first demonstrated in 
Ref.~\citep{LSC_2016grtests} for GW150914, and later extended to include
overtones in Refs.~\citep{Brito:2018rfr,Giesler:2019uxc,Isi:2019aib,Bhagwat:2019dtm,Forteza:2020hbw}. Consistency
of the late-time \sout{signal} \ab{waveform} with a single QNM is a test of the ringdown of
a BBH coalescence, but not necessarily a test of the no-hair
conjecture, which requires the measurement of (at least) two QNMs (BH
spectroscopy), and \sout{checking for} consistency between them~\cite{Dreyer:2003bv,Berti:2005ys}. Recent work
in that direction \comment{AB: which direction? There are papers by Nikhef group and 
also Brito et al., and many others that have investigated BH spectroscopy.} include~\citep{Carullo:2018gah,Carullo:2019flw,Bhagwat:2019bwv}. The
nature of the remnant object has also been explored through tests of
BH thermodynamics, like the Hawking's area
theorem~\citep{Cabero:2017avf} or through search for echos in the
post-merger
signal~\citep{Nielsen:2018lkf,Tsang:2019zra,Lo:2018sep,Abedi:2018npz,Abedi:2020sgg,Testa:2018bzd}. None
of these tests have found evidence for non-BH nature of the remnant
object (as described in GR) in LIGO-Virgo BBH observations.

\comment{AB: I don't understand the next sentence, it looks to me out of context.} 
Most of the tests mentioned above focus on analysing the post-merger
or late-time ringdown signal in isolation. \sout{Second generation
ground-based interferometric detectors like} \ab{Ground-based detectors, 
such as \sout{Advanced} LIGO and Virgo,} are most sensitive to 
stellar-mass BH binaries that merge near the
minim\ab{um}\sout{a} of their sensitivity band ($\sim 100$Hz). As a consequence, the
remnant object \textit{rings down} \comment{AB: we are not discussing the interferometer noises, 
so why to give such a detail about the shot noise?} \sout{in shot-noise dominated higher
frequencies, leaving} \ab{at frequencies where the sensitivity is worse, leaving} 
very little signal-to-noise ratio (SNR) in the
post-merger signal. \comment{AB: This discussion about the starting time of the ringdown starts too abruptly, 
and looks out of context and incomplete. Also, why are we citing only those papers? The matter 
has a long history. There have been calibrations to NR of damped- oscillation signals.} 
The ringdown start time is not a clearly defined
quantity and has been explored in detail
in Ref.~\citep{Bhagwat:2017tkm}. In other
works~\citep{Carullo:2018gah,Carullo:2019flw}, it has been left as a
free parameter to be estimated directly from the data. Uncertainties
in estimates of the ringdown start-time, as well as, an overall lack of
SNR in the post-merger signal, given typical sensitivities of
ground-based detectors, result in significant statistical uncertainties
in the measurement of the QNM frequencies. 

An independent approach to BH spectroscopy, based on the full-signal
analysis, was \sout{outlined} \ab{introduced} in Ref.~\citep{Brito:2018rfr} 
(henceforth referred to as \paperone). Unlike methods that focus \ab{only} 
on the post-merger signal, it \sout{describes a framework in which a} 
\ab{employs the} complete inspiral-merger-ringdown
(IMR) waveform \sout{model is used} to measure the complex QNM
frequencies. This \sout{allows} \ab{gives} access to the full SNR of the signal,
reducing measurement uncertainties. Moreover, the definition of the
ringdown start time is built into the \ab{merger-ringdown} model and does not need to be
either left as an additional free parameter or fixed using alternate
definitions. While \paperone presented the method \ab{and tested it for }\sout{in the context of a
non-spinning waveform model} \ab{non-spinning BBHs}, \ab{here} we extend the analysis to the \ab{more 
realistic astrophysical} case in which BHs \ab{carry} spin\sout{s in the current paper}. \ab{Furthermore, 
the IMR waveform model used in this paper is more accurate than what employed in \paperone, 
because it contains higher-order corrections in PN theory and it was calibrated to a much larger set 
of numerical-relativity (NR) waveforms~\cite{Bohe:2016gbl}.}  All astrophysical BHs are expected to be spinning, 
and ignoring effects of spin has been shown to introduce systematic biases in the measurement of the 
source properties \cite{paper_showing_systematics_from_ignoring_spin}.

The rest of the paper is organized as follows. Section~\ref{sec:model} describes our parameterized IMR
waveform model. In Sec.~\ref{sec:method}, we define our framework to test GR\ab{, notably how we can measure 
complex QNM frequencies with our parameterized model within a Bayesian formalism}. Then, in Sec.~\ref{sec:results}, 
we demonstrate our method on real GW events, as well as simulated signals. \ab{In particular, we analyze 
the ... and constrain the QNM frequecies ... Finally, in Sec.~\ref{sec:discussion} we provide a summary of
our results and discuss future developments.} 

\section{Method}

\comment{AB: In teh following, I will not use much the strike-out, but instead I will 
bracket in color the text that I have changed.}

A GW signal from the (quasi-circular) coalescence of two BHs is
completely described in GR by a \sout{fixed set of} $15$ parameters,
$\bxigr$. These can be grouped into the \emph{intrinsic} parameters:
the masses, $m_1, m_2$ and spins, $\bs1, \bs2$ of the \ab{component 
objects in the} \sout{initial} binary; and the 
\emph{extrinsic} parameters: \ab{a reference time $t_c$ and phase 
$\phi_c$,} the sky position of the binary ($\alpha$,
$\delta$), the luminosity distance, $d_L$, and \ab{the binary's} orientation
described through the inclination of the binary $\iota$ and its
polarization $\psi$. \ab{We also introduce the total mass $M = m_1+m_2$, 
and the (dimensionless) symmetric mass ratio $\nu = m_1m_2/M^2$}. 

\ab{Here, we focus on BHs with spins aligned or anti-aligned 
with the orbital angular momentum (henceforth, ``aligned-spin''). In this case,  
the GW signal depends on $11$ parameters. We denote the 
aligned-spin \ab{(dimensionless)} components as $\chi_{i} = |\vec{\bm{S}}_i|/m^2_i$, where $i=1,2$ for the two BHs.}

\subsection{Waveform Model}\label{sec:model}

As in \paperone, we use an IMR waveform model developed within the effective-one-body (EOB) 
formalism~\cite{}. \ab{However, whereas \paperone was limited to non-spinning multipolar waveforms, 
here we use as our baseline model the aligned-spin multipolar waveform model 
developed in Ref.~\citep{Cotesta:2018fcv}}. In addition to being
calibrated to NR simulations, this model also uses information from BH
perturbation theory for the merger and ringdown phases. \sout{Therefore this
waveform model provides a very accurate and faithful semi-analytic
description of the inspiral, merger and ringdown phases.} \ab{Henceforth we
will denote this model by $\SEOB$ for short}~\footnote{\ab{In the LIGO Algorithm Library (LAL), this 
waveform model is called {\tt SEOBNRv4HM}.}}.


In the observer's frame, the GW polarizations can be written as 
%
\begin{equation}
h_+(\iota,\varphi;t ) - i h_\times(\iota,\varphi;t) = \sum_{\ell, m} {}_{-\!2}Y_{\ell m}(\iota,\varphi)\, h_{\ell m}(t)\,,
\end{equation}
%
\comment{AB: there is a mismatch of notations for the parameters with respect to what was introduced above.}
where $\varphi$ is the azimuthal direction to the observer, ${}_{-\!2}Y_{\ell m}(\theta,\varphi)$ are the $-2$ spin-weighted spherical harmonics. The $\SEOB$ model we employ includes the $(\ell, |m|)=(2,2),(2,1)$, $(3,3)$, $(4,4)$, and $(5,5)$ modes~\cite{Cotesta:2018fcv}. For each $(\ell, m)$, the inspiral-(plunge-)merger-ringdown $\SEOB$ waveform is schematically given by
%
\begin{equation}
h_{\ell m}(t) = h_{\ell m}^\mathrm{insp-plunge}\, \theta(t_\mathrm{match}^{\ell m} - t) + h_{\ell m}^\mathrm{merger-RD}\,\theta(t-t_\mathrm{match}^{\ell m})\,,
\end{equation}
where $\theta(t)$ is the Heaviside step function, $h_{\ell m}^\mathrm{insp-plunge}$ represents the inspiral-plunge part of the waveform, whereas $h_{\ell m}^\mathrm{merger-RD}$ denotes the merger-ringdown \ab{waveform, which reads~\citep{Bohe:2016gbl,Cotesta:2018fcv}}
%
\begin{equation}
\label{RD}
h_{\ell m}^{\textrm{merger-RD}}(t) = \nu \ \tilde{A}_{\ell m}(t)\ e^{i \tilde{\phi}_{\ell m}(t)} \ e^{-i \sigma_{\ell m 0}(t-t_{\textrm{match}}^{\ell m})},
\end{equation}
%
where $\nu$ is the symmetric mass ratio of the binary and $\sigma_{\ell m0} = 2\pi f_{\ell m 0} -i/\tau_{\ell m 0}$ denotes the complex frequency of the fundamental QNMs of the remnant BH. We denote the oscillation frequencies by $f_{\ell m  0}\equiv \Re(\sigma_{\ell m0})/(2\pi)$ and the decay times by $\tau_{\ell m 0}\equiv -1/\Im(\sigma_{\ell m0}) $. 
The functions $\tilde{A}_{\ell m}(t)$ and $\tilde{\phi}_{\ell m}(t)$ are given by~\cite{Bohe:2016gbl,Cotesta:2018fcv}:
%
\begin{equation}
\label{eq:ansatz_amp}
\tilde{A}_{\ell m}(t) = c_{1,c}^{\ell m} \tanh[c_{1,f}^{\ell m}\ (t-t_{\textrm{match}}^{\ell m}) \ +\ c_{2,f}^{\ell m}] \ + \ c_{2,c}^{\ell m},
\end{equation}
%
\begin{equation}
\label{eq:ansatz_phase}
\tilde{\phi}_{\ell m}(t) = \phi_{\textrm{match}}^{\ell m} - d_{1,c}^{\ell m} \log\left[\frac{1+d_{2,f}^{\ell m} e^{-d_{1,f}^{\ell m}(t-t_{\textrm{match}}^{\ell m})}}{1+d_{2,f}^{\ell m}}\right],
\end{equation}
%
where $ \phi_{\textrm{match}}^{\ell m}$ is the phase of the inspiral-plunge mode $(\ell, m)$ computed at $t = t_{\textrm{match}}^{\ell m}$. The coefficients $d_{1,c}^{\ell m}$ and $c_{i,c}^{\ell m}$ with $i = 1,2$
are fixed by imposing that the functions $\tilde{A}_{\ell m}(t)$ and $\tilde{\phi}_{\ell m}(t)$ are of class $C^1$ at $t = t_{\textrm{match}}^{\ell m}$, when matching the merger-ringdown waveform to the inspiral-plunge $\SEOB$ waveform $h_{\ell m}^\mathrm{inspiral-plunge}(t)$. This allows us to write the coefficients $c_{i,c}^{\ell m}$ as~\cite{Cotesta:2018fcv}:
% in terms of $c_{1,f}^{\ell   m},\ c_{2,f}^{\ell m},\ \sigma^\textrm{R}_{\ell m},\ |h_{\ell    m}^{\textrm{insp-plunge}}(t_{\textrm{match}}^{\ell  m})|,\ \partial_t|h_{\ell    m}^{\textrm{insp-plunge}}(t_{\textrm{match}}^{\ell m})|$ as
\begin{align} 
\label{c1}
c_{1,c}^{\ell m} &= \frac{1}{c_{1,f}^{\ell
    m} \nu} \big[ \partial_t|h_{\ell
    m}^{\textrm{insp-plunge}}(t_{\textrm{match}}^{\ell m})| \nonumber \\
    &- \sigma^\textrm{R}_{\ell m} |h_{\ell
    m}^{\textrm{insp-plunge}}(t_{\textrm{match}}^{\ell
    m})|\big] \cosh^2{(c_{2,f}^{\ell m})}, \\
\label{c2}
c_{2,c}^{\ell m} &= -\frac{ |h_{\ell
    m}^{\textrm{insp-plunge}}(t_{\textrm{match}}^{\ell
    m})|}{\nu} + \frac{1}{c_{1,f}^{\ell
    m} \nu} \big[ \partial_t|h_{\ell
    m}^{\textrm{insp-plunge}}(t_{\textrm{match}}^{\ell m})|  \nonumber \\
    &- \sigma^\textrm{R}_{\ell m} |h_{\ell
    m}^{\textrm{insp-plunge}}(t_{\textrm{match}}^{\ell
    m})|\big] \cosh{(c_{2,f}^{\ell m})}\sinh{(c_{2,f}^{\ell m})}, \\ \nonumber   
\end{align}
and $d_{1,c}^{\ell m}$ as
\begin{align}
\label{d1}    
d_{1,c}^{\ell m} &= \left[\omega_{\ell m}^{\textrm{insp-plunge}}(t_{\textrm{match}}^{\ell m}) -  \sigma^\textrm{I}_{\ell
      m}\right]\frac{1+ d_{2,f}^{\ell m}}{d_{1,f}^{\ell m}d_{2,f}^{\ell m}}\,,
\end{align}
%
where we denoted $\sigma_{\ell m}^\textrm{R} \equiv \Im (\sigma_{\ell m0}) < 0$ and  $\sigma_{\ell m}^\textrm{I} \equiv -\Re (\sigma_{\ell m0})$, and $\omega_{\ell m}^{\textrm{insp-plunge}}(t)$ is the frequency of the inspiral-plunge EOB mode. The coefficients $c_{i,f}^{\ell m}$ and $d_{i,f}^{\ell m}$ are obtained through fits to NR and 
Teukolsky-equation--based waveforms and can be found in Appendix C of Ref.~\cite{Cotesta:2018fcv}.

In the $\SEOB$ model constructed in Ref.~\cite{Cotesta:2018fcv}, the
complex frequencies $\sigma_{\ell m 0}$ are expressed in terms of the
final BH mass and spin~\cite{Berti:2005ys,Berti:2009kk}, and the
latter are related to the BBH's component masses and spins through
NR--fitting-formulas obtained in
GR~\cite{Taracchini:2013rva,Hofmann:2016yih}. Here instead, in the
spirit of what was done in \paperone, we promote the QNM (complex)
frequencies to be free parameters of the model, while keeping the
inspiral-plunge modes $h_{\ell m}^\mathrm{inspiral-plunge}(t)$ fixed
to their GR values. \ab{More explicitly, we introduce a parameterized 
version of the $\SEOB$ model where the frequency and the
damping time of the ${\ell m 0}$ mode (i.e, $(f_{\ell m 0}, \tau
_{\ell m 0})$) is defined through the fractional deviations, $(\delta
f_{\ell m 0},\delta \tau_{\ell m 0})$, from the corresponding GR
values, which are obtained using NR fits~\cite{Taracchini:2013rva,Hofmann:2016yih}. Thus,} 
\begin{eqnarray}
f_{\ell m 0} &=& f_{\ell m 0}^{\text{GR}}\, (1 + \delta f_{\ell m 0})\,,\label{eq:nongr_freqs_a} \\ 
\tau _{\ell m 0} &=& \tau _{\ell m 0}^{\text{GR}}\, (1 + \delta \tau_{\ell m 0})\,. \label{eq:nongr_freqs_b}
\end{eqnarray}
\ab{We denote such a parameterized waveform model \pSEOB}~\footnote{\ab{This 
waveform model is called {\tt pSEOBNRv4HM} in LAL.}}.

\ab{As said}, the GR quantities $( f_{\ell m 0}^{GR},\tau_{\ell m 0}^{GR})$ are
constructed using the same NR--fitting--formula \sout{prescriptiion} 
described above. We note that when leaving $\sigma_{\ell m}$ to vary
freely, the functions $\tilde{A}_{\ell m}(t)$ and $\tilde{\phi}_{\ell
  m}(t)$ \sout{will} in general also differ from the GR prediction\ab{s}, since
those functions depend on the QNM complex frequencies, as can be seen
from the expressions for $c_{i,c}^{\ell m}$ and $d_{1,c}^{\ell m}$ \ab{in Eqs.~(\ref{c1}), 
(\ref{c2}), and (\ref{d1}).}


\subsection{Bayesian Parameter Estimation}\label{sec:method}

The parameterized model, $\pSEOB$, described above introduces an additional set of non-GR parameters, $\bxingr = (\delta f_{\ell m 0},\delta \tau_{\ell m 0})$, corresponding to each $(\ell,m)$ QNM present in the GR waveform model $\SEOB$. One then proceeds to use the Bayes theorem to obtain the \emph{posterior} probability distribution on $\blambda = \{\bxigr, \bxingr\}$, given a hypothesis $\mathcal{H}$:
%
\begin{equation}
P(\blambda | d, \mathcal{H}) = \frac{P(\blambda | \mathcal{H}) \, \mathcal{L}(d | \blambda, \mathcal{H})}{P(d|\mathcal{H})},
\label{eq:Bayes_theorem}
\end{equation}
%
where $P(\blambda | \mathcal{H})$ is the \emph{prior} probability distribution, and $\mathcal{L}(d | \blambda, \mathcal{H})$ is called the \emph{likelihood} function. The denominator is a normalization constant $P(d|\mathcal{H}) = \int P(\blambda | \mathcal{H}) \, \mathcal{L}(d | \blambda, \mathcal{H}) \, d\blambda$, called the marginal likelihood, or the \emph{evidence} of the hypothesis $\mathcal{H}$. In this case, our hypothesis $\mathcal{H}$ is that the data contains a GW signal that is described by the $\pSEOB$ waveform model $h(\blambda)$  and stationary Gaussian noise described by a power spectral density (PSD) $S_n(f)$. The likelihood function can consequently be defined as:
%
\begin{equation}
\mathcal{L}(d | \blambda, \mathcal{H}) \propto \exp\big[-\frac{1}{2} \langle d - h(\blambda) \, | \, d -h(\blambda) \rangle \big],
\label{eq:likelihood}
\end{equation}
%
where $\langle . | . \rangle$ is the following noise-weighted inner product:
%
\begin{equation}
\langle A | B \rangle = \int_{f_\mathrm{low}} ^{f_\mathrm{high}} df \frac{\tilde{A}^*(f)\tilde{B}(f) + \tilde{A}(f)\tilde{B}^*(f)}{S_n(f)}.
\label{eq:nwip}
\end{equation}
%
\ab{The quantity} $\tilde{A}(f)$ denotes the Fourier transform of $A(t)$ and the $^*$ indicates complex conjugation. The limits of integration ${f_\mathrm{low}}$ and ${f_\mathrm{high}}$ define the bandwidth of the sensitivity of the GW detector. We usually assume ${f_\mathrm{high}}$ to be the Nyquist frequency whereas${f_\mathrm{low}}$ is dictated by \ab{the performance of the 
GW detector at low-frequency and we assume XXX}. \comment{AB: It isn't just seismic noise. In fact, at low frequency, there are mysterious noises that people have had hard time 
to control.} \sout{the response of the gravitational detector to low-frequency seismic noise.} Owing to the large dimensionality of the parameter set $\blambda$, the posterior distribution $P(\blambda | d, \mathcal{H})$ in Eq.~(\ref{eq:Bayes_theorem}) is computed by stochastically sampling the parameter space using techniques such as Markov-chain Monte Carlo (MCMC)~\cite{} or Nested Sampling~\cite{}. For this paper, we use the \verb+LALInference+~\cite{} and \verb+Bilby+ codes~\cite{} that provide an implementation of the parallely tempered MCMC and Nested Sampling algorithms respectively, for computing the posterior distributions. 

Given the full-dimensional posterior probability density function $P(\blambda | d, \mathcal{H})$, we can marginalize over the \emph{nuisance} parameters, to obtain the marginalized posterior probability density function over the QNM parameters $\bxingr$:

\begin{equation}
P(\bxingr | d, \mathcal{H})= \int P(\blambda | d, \mathcal{H}) d\bxigr\,.
\end{equation}

For most of the results discussed in this paper, we restrict ourselves
to the $(\ell m) = (2,2)$ and/or $(3,3)$ modes. In those cases we
\sout{would} assume $\bxingr = \{\df{220},\dtau{220}\}$ and/or $
\{\df{330},\dtau{330}\}$, and all the other $(\ell m)$ modes to be
fixed at their GR predictions (i.e., $\delta f_{\ell m 0} = \delta
\tau_{\ell m 0} = 0$). This is because, for most of the high-mass BH 
events that we find most appropriate for this test, the LIGO-Virgo
observations are consistent with nearly--equal-mass face-on/off BBHs 
for which power in the subdominant modes is not enough to
attempt to measure more than one, or at most \comment{AB: it seems some wrods are missing 
to make the senetence meaningful} QNM frequencies, 
simultaneously.

\ab{Lastly}, throughout our analysis, we assume uniform priors on our non-GR QNM
parameters, $(\delta f_{\ell m 0},\delta \tau_{\ell m 0})$. However,
since the priors on $( f_{\ell m 0}^{GR},\tau_{\ell m 0}^{GR})$ are
derived through NR--fits, from the corresponding priors on the initial
masses and spins, this leads to a non-trivial prior on the final
reconstructed frequency and damping time, $( f_{\ell m 0},\tau_{\ell m
  0})$. Also, given the definition of the damping time in
Sec.~\ref{sec:model}, we \sout{realize} \ab{note} that $\delta \tau_{\ell m 0} = -1$ leads
to the imaginary part of the frequency going to infinity. We avoid
this by restricting the minimum of the prior on $\delta \tau_{\ell m
  0}$ to be greater than $-1$.

\iffalse
\subsection{Priors}

Throughout our analysis we assume a completely prior uniform in the component masses $m_1, m_2$. Our prior on the spins are uniform in the magnitude between (0,1) and isotropic in spin orientation, but finally restricted to the component parallel to the orbital angular momentum of the binary. The prior on the distance varies as $d_L^2$ giving more weightage to binaries farther out. For the rest of the parameters we use standard priors as defined in the documentation (CITE Vietch et al. 2015 LALInference paper). For our non-GR ringdown parameters, we assume uniform priors. This of course leads to a non-trivial prior on the reconstructed frequency and damping time, because of the prior on $( f_{\ell m 0}^{GR},\tau_{\ell m 0}^{GR})$, which itself depend on the prior on the initial masses and spins through NR fits \abhi{perhaps figure on priors on QNM quantities}. For $d\tau = -1$, we encounter a singularity (the imaginary part of the frequency goes to infinity), which we avoid by restricting the minimum of the prior on $d\tau$ to be greater than $-1$.
\fi

\section{Results}\label{sec:results}

\subsection{Simulations using GR signals in Gaussian Noise}\label{ssec:gr_signal}

First, we demonstrate our method on GW signals from BBH systems that are well described by GR. We inject simulated GW signals modellng inspiral, merger and ringdown of BBHs into coloured Gaussian noise with PSDs expected at design sensitivities of Advanced LIGO-Virgo CITE. In particular we choose parameters similar to two specific GW events, GW150914 CITE and GW190521 CITE. The details of the injections are provided in Table~\ref{tab:injection_values}. These two systems are representative data points for the kind of systems this test is most suitable for: high-mass events which are loud enough that the SNR pre- and post-merger return reliable parameter estimation results. To avoid possible systematic biases in our parameter estimation results due to error in modelling, we use the GR version of the same waveform, $\SEOB$ (without allowing for deviations in the QNM parameters) to simulate our GW signal. And to avoid systematic biases due to noise, we use an averaged (zero-noise) realisation of the noise. A detailed study on noise systematics for one of the GW events is presented in Appendix~\ref{sec:noise_systematics}. As in the case of the actual detections, we consider a two-detector advanced LIGO network at Hanford and Livingston, having identical PSDs. The distance to the two simulated events is rescaled such that the SNR in the detector network is same as the actual event, i.e, 24 (14) for GW150914 (GW190521). Since (almost) equal-mass binaries like GW150914 and GW190521 observed at moderately high SNRs are not expected to have a loud ringdown signal, we restrict ourselves to estimating the frequency and damping time of the $(\ell m) = (2,2)$, i.e., $\{\df{220},\dtau{220}\}$, while fixing the other QNM frequencies to their GR values. 

We find, as one might expect, that the posterior distribution on the parameters describing fractional deviations in the frequency and damping time are consistent with zero (left panels of Fig.~\ref{fig:simulated_signal_GR}). One can then convert these fractional quantities into absolute quantities using the relations given in Eqs.~\ref{eq:nongr_freqs_a} and ~\ref{eq:nongr_freqs_b}, and construct posterior distributions on these effective quantities, $(\fngr{220}, \taungr{220})$ (right panels of Fig.~\ref{fig:simulated_signal_GR}). In each of these cases, recovered posterior are consistent with the GR predictions (black solid lines).

%%%%%%%%%%%%%%%%%%%%%%%%%%%%%%%%%%%%%%%%%%%%%%%%%%%%%%%%%%%%%%%
% Table for Injections
%%%%%%%%%%%%%%%%%%%%%%%%%%%%%%%%%%%%%%%%%%%%%%%%%%%%%%%%%%%%%%%
\begin{table}[h!]
\begin{center}
\begin{tabular}{ |c|c|c|c|c|c| }
 \hline
 Event & $m_{1,d} (\Mo)$ &  $m_{2,d} (\Mo)$ & $a_{1z}$ & $a_{2z}$ & SNR \\ 
 \hline
 GW150914 & 39 & 31 & 0.0 & 0.0 & 25 \\
 GW190521 & 150 & 120 & 0.02 & -0.39 & 14 \\ 
 SXS:BBH:0166 & 72 & 12  & 0.0 & 0.0 & 75 \\
 \hline
\end{tabular}
\caption{Details of the injection parameters, chosen to be similar to the actual GW events. $(m_{1,d},m_{2,d})$ are the detector-frame masses of the primary and secondary BHs respectively.}
\label{tab:injection_values}
\end{center}
\end{table}
%%%%%%%%%%%%%%%%%%%%%%%%%%%%%%%%%%%%%%%%%%%%%%%%%%%%%%%%%%%%%%%
%%%%%%%%%%%%%%%%%%%%%%%%%%%%%%%%%%%%%%%%%%%%%%%%%%%%%%%%%%%%%%%

%%%%%%%%%%%%%%%%%%%%%%%%%%%%%%%%%%%%%%%%%%%%%%%%%%%%%%%%%%%%%%%
% Simulated siganl: GR
%%%%%%%%%%%%%%%%%%%%%%%%%%%%%%%%%%%%%%%%%%%%%%%%%%%%%%%%%%%%%%%
\begin{figure*}%[h!]
\begin{center}
	\includegraphics[width=0.5\textwidth]{figures/GW150914_simulated_signal_0p0_deltaf220_deltatau220.png}\includegraphics[width=0.5\textwidth]{figures/GW150914_simulated_signal_0p0_f220_tau220.png}	
	\includegraphics[width=0.5\textwidth]{figures/GW190521_simulated_signal_0p0_deltaf220_deltatau220.png}\includegraphics[width=0.5\textwidth]{figures/GW190521_simulated_signal_0p0_f220_tau220.png}		
	\caption{\textcolor{red}{FINAL RESULT} Posterior probability distribution on the fractional deviations in the frequency and damping time of the $(2,2)$ QNM, $\{\df{220},\dtau{220}\}$ (left panels) and the reconstructed quantities, $(\fngr{220}, \taungr{220})$ (right panels) for GR injections with initial parameters similar to GW150914 (top panels) and GW190521 (bottom panels) (Table~\ref{tab:injection_values}). The 2D contour marks the 90\% credible region, while the dashed lines on the 1D marginalized distributions mark the 90\% credible levels. The black vertical and horizontal lines mark the injection values.}
	\label{fig:simulated_signal_GR}
\end{center}
\end{figure*}
%%%%%%%%%%%%%%%%%%%%%%%%%%%%%%%%%%%%%%%%%%%%%%%%%%%%%%%%%%%%%%%
%%%%%%%%%%%%%%%%%%%%%%%%%%%%%%%%%%%%%%%%%%%%%%%%%%%%%%%%%%%%%%%

\subsection{Simulations using modified GR signals in Gaussian noise}

To demonstrate the robustness of the method in detecting possible deviations of the underlying GW signal from predictions of GR, we inject simulated GW signals which are identical to the corresponding GR prediction upto merger, and differ in their post-merger description. We again choose systems with initial-binary-parameters similar to GW150914 and GW190521, but we attach a phenomenological post-merger signal described by deviation parameters $\df{220} = \dtau{220} = 0.5$ \abhi{deviations=0.1 runs are ongoing and almost done}. In other words, we assume that the frequency and damping time of our non-GR signal is 1.5 times the corresponding GR prediction, although the pre-merger signal is identical to GR (Fig.\ref{fig:nongr_waveform}). We also avoid waveform and noise systematic biases by choosing a configuration identical to the simulations described in Sec.\ref{ssec:gr_signal}. The posterior probability distributions on $(\df{220}, \dtau{220})$ or equivalently $(\fngr{220}, \taungr{220})$ (Fig.~\ref{simulated_signal_nonGR}) are consistent with the corresponding the values of the injection parameters, indicated by the black solid lines. 

We additionally investigate the effects of erroneously assuming that an underlying modified GR signal can be well-described using GR. We do this by estimating the parameters of our modified GR signals using the GR waveform model $\SEOB$ instead of the parameterized $\pSEOB$. In such cases, we run the risk of biased parameter estimates due to an incomplete understanding of the underlying signal. The resulting posterior probability distributions are shown in the right panels of Fig.~\ref{simulated_signal_nonGR} by the pink (GW150914) and grey (GW190521) curves. The results are interesting and distinctly different for the two events. For the GW150914-like modified GR signal, the measurements of $(\fgr{220}, \taugr{220})$ (Fig.~\ref{simulated_signal_nonGR} top right panel) are consistent with the $(\fgr{220}, \taugr{220})$ measurements for a signal with no deviations (Fig.~\ref{simulated_signal_GR} top right panel). In other words, if the actual signal had deviations as large as the 50 \% of the GR prediction, the analysis with $\SEOB$ would likely have reported \emph{no} deviation from the GR prediction. However in the case of the GW190521-like modified GR signal, a simple GR analysis of the modified GR signal would have yielded measurements distinctly different from either of the two parameterized estimates: with and without deviations. The fact that the GW190521-like signal has a much lower SNR than GW150914 might be a possible reason for the the measurement of the final quantities to be more susceptible to noise. A more detailed comparison of the other parameters, like the masses and spins, between an $pSEOB$ and an $SEOB$ measurement of a modified GR signal is shown in Fig.~\ref{fig:gr_ngr_comparison}.

%%%%%%%%%%%%%%%%%%%%%%%%%%%%%%%%%%%%%%%%%%%%%%%%%%%%%%%%%%%%%%%
%%%%%%%%%%%%%%%%%%%%%%%%%%%%%%%%%%%%%%%%%%%%%%%%%%%%%%%%%%%%%%%
\begin{figure}
	\includegraphics[width=0.5\textwidth]{figures/modGR_waveforms_amplitudephase.png}
	\caption{\textcolor{red}{FINAL RESULT} Top panel: The `+'--polarisation of the gravitational waveform $h_+(t)$ from a GW150914-like event where the post-merger is described by GR (solid orange lines), i.e., $\df{220} = \dtau{220} = 0$, and where the merger-ringdown is modified (dashed grey lines), i.e., $\df{220} = \dtau{220} = 0.5$. Bottom panel: Comparison of the evolution of the amplitude (left) and instantaneous frequency (right) for the GR and modified GR signal.}
	\label{fig:nongr_waveform}
\end{figure}
%%%%%%%%%%%%%%%%%%%%%%%%%%%%%%%%%%%%%%%%%%%%%%%%%%%%%%%%%%%%%%%
%%%%%%%%%%%%%%%%%%%%%%%%%%%%%%%%%%%%%%%%%%%%%%%%%%%%%%%%%%%%%%%

%%%%%%%%%%%%%%%%%%%%%%%%%%%%%%%%%%%%%%%%%%%%%%%%%%%%%%%%%%%%%%%
% Simulated siganl: non-GR
%%%%%%%%%%%%%%%%%%%%%%%%%%%%%%%%%%%%%%%%%%%%%%%%%%%%%%%%%%%%%%%
\begin{figure*}%[h!]
\begin{center}
	\includegraphics[width=0.5\textwidth]{figures/GW150914_simulated_signal_0p5_deltaf220_deltatau220.png}\includegraphics[width=0.5\textwidth]{figures/GW150914_simulated_signal_0p5_gr_ngr_fngrtaungr.png}	
	\includegraphics[width=0.5\textwidth]{figures/GW190521_simulated_signal_0p5_deltaf220_deltatau220.png}\includegraphics[width=0.5\textwidth]{figures/GW190521_simulated_signal_0p5_gr_ngr_fngrtaungr.png}
	\caption{\textcolor{red}{FINAL RESULT} Posterior probability distribution on the fractional deviations in the frequency and damping time of the $(2,2)$ QNM, $\{\df{220},\dtau{220}\}$ (left panels) and the reconstructed quantities, $(\fngr{220}, \taungr{220})$ (right panels) for modGR injections with initial parameters similar to GW150914 (top panels) and GW190521 (bottom panels) (Table~\ref{tab:injection_values}). The underlying signal has a deviation, $\df{220} = \dtau{220} = XX$. The 2D contour marks the 90\% credible region, while the dashed lines on the 1D marginalized distributions mark the 90\% credible levels. The black vertical and horizontal lines mark the injection values. In the right panels, we additionally show measurements using a GR ($SEOB$) waveform, for GW150914 (grey) and GW190521 (pink), The measurements are visibly biased.}
	\label{fig:simulated_signal_nonGR}
\end{center}
\end{figure*}
%%%%%%%%%%%%%%%%%%%%%%%%%%%%%%%%%%%%%%%%%%%%%%%%%%%%%%%%%%%%%%%
%%%%%%%%%%%%%%%%%%%%%%%%%%%%%%%%%%%%%%%%%%%%%%%%%%%%%%%%%%%%%%%

%%%%%%%%%%%%%%%%%%%%%%%%%%%%%%%%%%%%%%%%%%%%%%%%%%%%%%%%%%%%%%%
% modified GR signal: GR vs nonGR recovery comparison
%%%%%%%%%%%%%%%%%%%%%%%%%%%%%%%%%%%%%%%%%%%%%%%%%%%%%%%%%%%%%%%
\begin{figure*}%[h!]
	\includegraphics[width=0.5\textwidth]{figures/GW150914_simulated_signal_0p5_gr_ngr_m1m2.png}\includegraphics[width=0.5\textwidth]{figures/GW190521_simulated_signal_0p5_gr_ngr_m1m2.png}
	\includegraphics[width=0.5\textwidth]{figures/GW150914_simulated_signal_0p5_gr_ngr_a1za2z.png}\includegraphics[width=0.5\textwidth]{figures/GW190521_simulated_signal_0p5_gr_ngr_a1za2z.png}	
	\includegraphics[width=0.5\textwidth]{figures/GW150914_simulated_signal_0p5_gr_ngr_fgrtaugr.png}\includegraphics[width=0.5\textwidth]{figures/GW190521_simulated_signal_0p5_gr_ngr_fgrtaugr.png}
	\includegraphics[width=0.5\textwidth]{figures/GW150914_simulated_signal_0p5_gr_ngr_Mfaf.png}\includegraphics[width=0.5\textwidth]{figures/GW190521_simulated_signal_0p5_gr_ngr_Mfaf.png}
	\caption{\textcolor{red}{FINAL RESULT} Comparison of the recovered parameters when the underlying signal is assumed to be a GR signal or a modified GR signal. In both cases, the actual underlying signal is a modified GR signal with parameters similar to GW150914 (left panels) and GW190521 (right panels) respectively. The initial parameters are given in Table~\ref{tab:injection_values}, with the QNM parameters defined by $\df{220} = \dtau{220} = 0.5$. For the GW150914 (GW190521) contours, the $SEOB$ and $pSEOB$ recoveries are indicated by blue (red) and pink (grey) curves respectively. The panels (from top to bottom) show the recoveries in (detector-frame) initial masses (first row), z-components of dimensionless initial spins (second row), GR predictions of frequency and damping time (third row) and the remnant mass and spin predictions ($M_f, a_f$) from the frequency and damping time (obtained by inverting the Berti fits) CITE}
	\label{fig:gr_ngr_comparison}
\end{figure*}
%%%%%%%%%%%%%%%%%%%%%%%%%%%%%%%%%%%%%%%%%%%%%%%%%%%%%%%%%%%%%%%
%%%%%%%%%%%%%%%%%%%%%%%%%%%%%%%%%%%%%%%%%%%%%%%%%%%%%%%%%%%%%%%


\subsection{Test of the No hair theorem}\label{ssec:nohairtheorem}

Finally, we provide a simple demonstration of a test of the no-hair theorem using our model. As described in the introduction, any test of the no-hair theorem of BHs would need to involve independent measurements of (at least) two different QNMs. We use a simulated GW signal from the SXS catalog CITE corresponding a non-spinning BBH with a mass-ratio $q=6$ (SXS:BBH:0166), rescaled to a total mass of $M=84 \Mo$ (Table~\ref{tab:injection_values}). We choose an asymmetric system to increase the SNR in the higher modes. We also rescale the distance and orientation of the binary such that the total SNR of the signal in a network of three detectors, LIGO Hanford, Livingston and Virgo, is 75. Based on the LIGO-Virgo observations during O3a, such asymmetric and loud signals are no longer just a theoretical prediction but quite plausible at design sensitivities. Using this signal, we attempt to measure the QNM frequencies $(2,\pm 2)$ and $(3,\pm 3)$ modes together (Fig.~\ref{fig:nohair_sxs}). The SNR in other sub-dominant modes is too less for us to estimate them reliably.

The fractional deviations in the estimates of the damping time and frequency of either mode is expected to be consistent with 0, as we indeed find in the left panel of Fig.~\ref{fig:nohair_sxs}. Consequently we find that the reconstructed quantities $(\fngr{220}, \taungr{220})$ and $(\fngr{330}, \taungr{330})$  are also consistent with the corresponding predictions for a BBH merger in GR. As a consequence, the information from these two independent measures correspond to a unique remnant object, which is completely described by its mass and spin angular momentum \abhi{have the mf-af plot as well as a third panel to show consistency}. For most of the events observed so far, the power in the $(3,\pm 3)$ has not been sufficient to measure it along with the $(2,\pm 2)$, or in fact, in its place. However, it might also be possible to combine information from multiple observation, as is likely over the coming few years of GW astronomy with the LIGO-Virgo detectors, to obtain meaningful constraints on the $(3,\pm 3)$ and other sub-dominant QNMs.

%%%%%%%%%%%%%%%%%%%%%%%%%%%%%%%%%%%%%%%%%%%%%%%%%%%%%%%%%%%%%%%

%%%%%%%%%%%%%%%%%%%%%%%%%%%%%%%%%%%%%%%%%%%%%%%%%%%%%%%%%%%%%%%
\begin{figure}
	\includegraphics[width=0.5\textwidth]{figures/nohair_sxs_0166_placeholder.png}	
	\caption{\textcolor{red}{DEMO FIGURE; RUN ONGOING} Posterior probability distribution on the fractional deviations (left panel) and the reconstructed (right panel) frequency and damping time of the $(2,\pm 2)$ (blue curves) and $(3,\pm 3)$ (red curves) respectively, for a numerical relativity signal corresponding to a BBH merger of $q=6$,  $M=84 \Mo$ and SNR $75$. The plus signs mark the GR predictions.}
	\label{fig:nohair_sxs}
\end{figure}
%%%%%%%%%%%%%%%%%%%%%%%%%%%%%%%%%%%%%%%%%%%%%%%%%%%%%%%%%%%%%%%
%%%%%%%%%%%%%%%%%%%%%%%%%%%%%%%%%%%%%%%%%%%%%%%%%%%%%%%%%%%%%%%

\subsection{Results on actual LIGO-Virgo results}

The LIGO-Virgo Collaboration recently released their testing GR catalogue containing results of this test for all events observed during O3a ~\cite{Abbott:2020jks}, which passed a threshold for the total (source-frame) mass $\geq 50 \Mo$ and SNRs in the pre- and post-merger regions $\geq 8$.The pre- and post-merger regions of the signal are identified as the power in the frequency content of the signal before and after the signal reaches peak amplitude, as determined by the maximum likelihood parameter estimation template. For the purpose of \cite{Abbott:2020jks}, we used a mass threshold and restricted ourselves to the highest-mass events which were expected to be most promising to study merger-ringdown. However, since our method relies on doing a parameter estimation on the entire inspiral-merger-ringdown signal, we require the SNR to be beyond a certain threshold throughout the signal for reliable parameter estimation of the initial and final quantities. In fact the SNR threshold alone should be sufficient for the analysis, and for this paper we have relaxed the mass threshold. This has added two events, GW190630$\_$185205 and GW190828$\_$063405, to the list of events considered in \cite{Abbott:2020jks} \abhi{Runs ongoing and looking promising}. Fig.~\ref{fig:o1o2_events} shows results from these two events along with GW190519$\_$153544, GW190521$\_$074359, GW190910$\_$112807.

For the first time, in this paper, we present results on the relevant events from LIGO-Virgo's first and second observing runs, alongside the above events. Applying the same thresholds as above, we find three additional events which could be included in the analysis: GW150914, GW170104, GW170729. The other high-mass events from O1-O2, GW170809, GW170814, GW170818 and GW170823 do not have an SNR of $8$ in the merger-ringdown signal. For the three relevant signals, GW150914, GW170104 and GW170729, we provide the posterior distributions in $(\df{220}, \dtau{220})$ in Fig.~\ref{fig:o1o2_events}. We also reconstruct the effective QNM parameters, $(\fngr{220}, \taungr{220})$ which are tabulated in Table~\ref{tab:qnm_o1o2_results}. \footnote{See Table IV of \cite{Abbott:2020jks} for a list of the SNR thresholds. The paper quotes them for the purpose of the IMR consistency test, but the same thresholds have been used for the $pSEOB$ test as well.} Among all the GW signals detected so far, GW150914 (solid curve in Fig.~\ref{fig:o1o2_events})is unique in its loudness, mass as well as the clarity of the signal, leading to the first, and to date, best attempt in measuring the QNM frequencies CITE.

%%%%%%%%%%%%%%%%%%%%%%%%%%%%%%%%%%%%%%%%%%%%%%%%%%%%%%%%%%%%%%%
% O1-O2 events
%%%%%%%%%%%%%%%%%%%%%%%%%%%%%%%%%%%%%%%%%%%%%%%%%%%%%%%%%%%%%%%
\begin{figure}[h!]
	\includegraphics[width=0.5\textwidth]{figures/rin_pseob_results.pdf}
	\caption{\textcolor{red}{PRELIMINARY RESULT} The 90\% credible levels of the posterior probability distribution of the fractional deviations in the frequency and damping time of the $(2,\pm 2)$ QNM, $\{\df{220},\dtau{220}\}$ and their corresponding one-dimensional marginalized posterior distributions, for events from O1, O2 and O3a passing a SNR threshold of $8$ in both the pre- and post-merger signal. The solid red curve marks the constraint the best single-event constraint, GW150914, whereas the contraints from the other events are indicated by the dash-dot curves. The joint constraints on $\{\df{220},\dtau{220}\}$ obtained using multiplying the likelihoods from individual events is given by the filled grey contours, while the hierarchical method of combination yields the black dot dashed curves.}
	\label{fig:o1o2_events}
\end{figure}
%%%%%%%%%%%%%%%%%%%%%%%%%%%%%%%%%%%%%%%%%%%%%%%%%%%%%%%%%%%%%%%
%%%%%%%%%%%%%%%%%%%%%%%%%%%%%%%%%%%%%%%%%%%%%%%%%%%%%%%%%%%%%%%

Finally, if we assume that the fractional deviations $(\df{220}, \dtau{220})$ take the same values in multiple events, we can assume the posterior of one event to be the prior for the next, and obtain a joint posterior probability distribution. For $N$ observations, where $P_j(\df{220}, \dtau{220} | d_j)$ is the posterior for the $j-th$ observation corresponding to the data set $d_j$, $j=1...N$, the joint posterior is given by:
\begin{equation}
P(\df{220}, \dtau{220} | \{d_j\}) = P(\df{220}, \dtau{220}) \prod _{j=1}^N \frac{P(\df{220}, \dtau{220} | d_j) }{P(\df{220}, \dtau{220})}
\end{equation}
where $P(\df{220}, \dtau{220})$ is the prior on $(\df{220}, \dtau{220})$. However, since we assume the prior on $(\df{220}, \dtau{220})$ to be flat or uniform, the joint posterior is equal to the joint likelihood.

We show these joint likelihoods on $(\df{220}, \dtau{220})$, as well as the corresponding one-dimensional marginalized distributions as filled grey curves in Fig.~\ref{fig:o1o2_events}. These are the strongest constraints on possible deviations in the measurement of $(\df{220}, \dtau{220})$ to date using our method.\abhi{add text on the hierarchical analysis if we want to include that analysis}

\begin{table}
\begin{flushleft}
\begin{tabular}{llllllll}
\toprule
Event & \multicolumn{2}{c}{Redshifted} & \hphantom{X} & \multicolumn{2}{c}{Redshifted} \\
& \multicolumn{2}{c}{frequency [Hz]} & \hphantom{X} & \multicolumn{2}{c}{damping time [ms]} \\[0.075cm]
\hline
& IMR  & pSEOB & \hphantom{X} & IMR  & pSEOB \\
\hline

GW150914 &
$249^{+9}_{-7}$ &
$-$ &
\hphantom{X} &
$4.1^{+0.3}_{-0.2}$ &
$-$
\\[0.075cm]

GW170104 &
$286^{+16}_{-27}$ &
$-$ &
\hphantom{X} &
$3.5^{+0.4}_{-0.3}$ &
$-$
\\[0.075cm]

GW170729 &
$161^{+13}_{-14}$ &
$-$ &
\hphantom{X} &
$7.8^{+1.8}_{-1.5}$ &
$-$
\\[0.075cm]

GWGW190630$\_$185205 &
$-$ &
$-$ &
\hphantom{X} &
$-$ &
$-$
\\[0.075cm]

GW190828$\_$063405,170729 &
$-$ &
$-$ &
\hphantom{X} &
$-$ &
$-$
\\[0.075cm]
\hline
\bottomrule
\end{tabular}
\caption{\textcolor{red}{NOT COMPLETE}}
\label{tab:qnm_o1o2_results}
\end{flushleft}
\end{table}

\section{Discussion}\label{sec:discussion}

\section*{Acknowledgements}
\label{sec:acknowledgements}
LIGO Clusters. \comment{AB: Our cluster!!!}


\appendix

\section{Study of syetematics on ringdown measurements}\label{sec:noise_systematics}

In Section~\ref{sec:method}, we mentioned that the expression of the Bayesian likelihood function outlined in Eqs.~\ref{eq:likelihood} and \ref{eq:nwip} is only valid if the interferometric noise can be described as a stationary Gaussian process. LIGO-Virgo noise, however, frequently has non-stationary and/or non-Gaussian features, for example glitches, which can affect parameter inferences unless appropriately accounted for. Here we demonstrate with an example of a GW190521-like simulated signal, systematic biases likely originating from an incomplete understanding of the noise. 

%%%%%%%%%%%%%%%%%%%%%%%%%%%%%%%%%%%%%%%%%%%%%%%%%%%%%%%%%%%%%%%
%%%%%%%%%%%%%%%%%%%%%%%%%%%%%%%%%%%%%%%%%%%%%%%%%%%%%%%%%%%%%%%
\begin{figure}
\begin{center}
	\includegraphics[width=0.5\textwidth]{figures/S190521g_swinjs.png}
	\caption{\textcolor{red}{FINAL RESULT} 90 \% credible level on the posterior probability distribution of the frequency and damping time of $(2,\pm 2)$ mode, $(\fgr{220}, \taugr{220}$ from simulations of an \texttt{NRSur7dq4} GW signal with parameters similar to the GW event, GW190521, in Gaussian noise (grey dot-dashed lines) and real interferometric noise (pink dot dashed lines). The GR prediction for the frequency and damping time is indicated by the black cross. While the Gaussian noise simulations are consistent with the prediction, at least 3 or the 5 real noise simulation are not. For comparison, we also plot the 90 \% credible level for the actual event, GW190521.}
	\label{fig:21g_systematics}
\end{center}	
\end{figure}
%%%%%%%%%%%%%%%%%%%%%%%%%%%%%%%%%%%%%%%%%%%%%%%%%%%%%%%%%%%%%%%
%%%%%%%%%%%%%%%%%%%%%%%%%%%%%%%%%%%%%%%%%%%%%%%%%%%%%%%%%%%%%%%

We use as our underlying GR signal, an NR waveform (using the waveform model \texttt{NRSur7dq4}) with parameters similar to the actual event GW190521 (as reported in Table I of \cite{Abbott:2020tfl}). The predictions of the frequency and damping time for the GR signal we choose are indicated by the black cross in Fig.~\ref{fig:21g_systematics}, and for comparison, the results on the actual signal are shown with a black solid curve. It would appear that while the measurement of the frequency is consistent with the prediction, we overestimate the damping time. 

The underlying NR signal is expected to be precessing, and since the $pSEOB$ model is built on an aligned-spin GR model, a lack of precession could bias measurements. We explore effects of possible waveform systematics by injecting the signal is coloured Gaussian noise. In this case, since the understanding of the noise realisations is complete, any measurement biases could only arise from incomplete understanding of the underlying signal. We find (grey curves in Fig.\ref{fig:21g_systematics}) our results to be consistent with the prediction thus ruling out a lack of precession to be a likely cause for the bias in the damping time measurement of the actual signal.

We subsequently study the effects of possible noise systematics by injecting the same \texttt{NRSur7dq4} GW signal in different realisations of actual interferometric noise around the real event GW190521. Since the PSDs of the GW detectors are expected to vary over longer durations of time, we select 5 different noise realisations over a segment of 2.5 hours of coincident data in both the LIGO detectors centred at the time of the actual event. The noise properties in this chunk of data, and consequently for all 5 simulated signals, is expected to be the closest to that for the actual event. The results are indicated by pink curves in Fig.~\ref{fig:21g_systematics}. For 3 of the 5 noise realisations, corresponding to $t_0-1$ hour, $t_0+0.5$ hours, and $t_0+1$ hour we recover a damping time similar to the actual event, where $t_0$ is the GPS time of the actual event. For the other two noise realisations, we estimate the consistent damping time but an off-set frequency, while the fifth noise realisation is consistent with both predictions. The SNR in the L1 Livingston detector goes down by more than 3 in some runs, indicative of how a variation in the noise strongly affects our ability to infer parameters. This also seems to indicate that a bias in the measurements of the damping time for the actual event can be unaccounted noise systematics.


%
\bibliographystyle{apsrev}
\bibliography{intro_paper}

\end{document}