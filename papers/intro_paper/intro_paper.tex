%\documentclass[twocolumn,prd,superscriptaddress,amsfonts,amssymb,amsmath,preprintnumbers,nofootinbib]{revtex4-1}
\documentclass[twocolumn,prd,aps,superscriptaddress,preprintnumbers,tightenlines,showpacs,nofootinbib,eqsecnum,amsfonts,amsmath]{revtex4-1}
\usepackage{epsfig}
\usepackage{graphics}
\usepackage{graphicx}
\usepackage{bm}
\usepackage[dvipsnames]{xcolor}
\usepackage{bm}
\usepackage{times}
\usepackage{xspace}
\usepackage[varg]{txfonts}
\usepackage[normalem]{ulem} % To get strikethrough (\sout)
\usepackage[colorlinks]{hyperref}
\usepackage[caption=false]{subfig}
\usepackage{booktabs}
\usepackage{url}
\usepackage{float}
\usepackage[bottom]{footmisc}
\usepackage{lineno}
%\linenumbers

\definecolor{LinkColor}{rgb}{0.75, 0, 0}
\definecolor{CiteColor}{rgb}{0, 0.5, 0.5}
\definecolor{UrlColor}{rgb}{0, 0, 0.75}
\hypersetup{linkcolor=LinkColor}
\hypersetup{citecolor=CiteColor}
\hypersetup{urlcolor=UrlColor}
\usepackage{perpage}
\MakePerPage{footnote}

\newcommand{\paperone}{Paper~I\xspace}
\newcommand{\abhi}[1]{\textcolor{red}{[\textit{AG: #1}]}}
\newcommand{\rb}[1]{\textcolor{blue}{[\textit{RB: #1}]}}
\newcommand{\ab}[1]{\textcolor{cyan}{#1}}
\newcommand{\comment}[1]{\textcolor{red}{[#1]}}

\newcommand{\h}{\mathpzc{h}}
\newcommand{\Hhat}{\hat{\mathpzc{H}}}
\newcommand{\B}{\mathpzc{B}}
\newcommand{\hlm}{\mathpzc{h}_{\ell m}}
\newcommand{\xilm}{\xi_{\ell m}}
\newcommand{\Ylm}{{Y}^{-2}_{\ell m}}
\newcommand{\Y}{{Y}^{-2}}
\newcommand{\hc}{h_\times}
\newcommand{\hp}{h_+}
\newcommand{\Fc}{F_\times}
\newcommand{\Fp}{F_+}
\newcommand{\Mf}{M_f}
\newcommand{\cA}{\mathpzc{A}}
\newcommand{\lm}{_{\ell m}}
\newcommand{\deff}{d_\mathrm{eff}}
\newcommand{\rmi}{\mathrm{i}}
\newcommand{\blambda}{\bm{\lambda}}
\newcommand{\btheta}{\bm{\theta}}
\newcommand{\bxi}{\bm{\xi}}
\newcommand{\bxigr}{\bm{\xi}_{\text{GR}}}
\newcommand{\bxingr}{\bm{\xi}_{\text{nGR}}}
\newcommand{\bzeta}{\bm{\zeta}}
\newcommand{\bs}[1]{\bm{\vec{S}_{#1}}}
\newcommand{\Mo}{M_{\odot}}
\newcommand{\FFe}{\mathrm{FF}_\mathrm{eff}}
\newcommand{\FF}{\mathrm{FF}}
\newcommand{\e}{\mathrm{e}}
\newcommand{\rhoopt}{\rho_\mathrm{opt}}
\newcommand{\rhosubopt}{\rho_\mathrm{subopt}}
\newcommand{\fqnm}{f}
\newcommand{\sigmaqnm}{\sigma}
\newcommand{\n}{\mathbf{n}}
\newcommand*{\skymapscale}{0.5}
\newcommand*{\paramestscale}{0.455}
\newcommand{\df}[1]{\delta f_{\text{#1}}}
\newcommand{\dtau}[1]{\delta \tau_{\text{#1}}}
\newcommand{\fngr}[1]{f_{\text{#1}}}
\newcommand{\taungr}[1]{\tau_{\text{#1}}}
\newcommand{\fgr}[1]{f ^{\text{GR}}_{\text{#1}}}
\newcommand{\taugr}[1]{\tau ^{\text{GR}}_{\text{#1}}}
\newcommand{\pSEOB}{\texttt{pSEOBNR}}
\newcommand{\SEOB}{\texttt{SEOBNR}}

\newcommand{\AEI}{\affiliation{Max Planck Institute for Gravitational Physics (Albert Einstein Institute), Am M\"uhlenberg 1, Potsdam 14476, Germany}}
\newcommand{\UMD}{\affiliation{Department of Physics, University of Maryland, College Park, MD 20742, USA}}

\begin{document}

%\title{Black hole spectroscopy using a complete gravitational wave signal from a binary black coalescence}


\title{Constraints on the quasi-normal mode frequencies of the LIGO-Virgo signals by making full use of gravitational-wave modeling}

\author{Abhirup Ghosh}
\AEI
\author{Richard Brito}
\affiliation{Dipartimento di Fisica, ``Sapienza" Universit\`a di Roma $\&$ Sezione INFN Roma1, Piazzale Aldo Moro 5, 00185 Roma, Italia}
\author{Alessandra Buonanno}
\AEI
\UMD

\date{\today}


%%%%%%%%%%%%%%%%%%%%%%%
\begin{abstract}
  The no-hair conjecture in General Relativity states that the
  properties of an astrophysical Kerr black hole (BH) are completely described by its
  mass and spin angular momentum. As a consequence, the complex
  quasi-normal-mode (QNM) frequencies of a binary black hole (BBH)
  ringdown can be uniquely determined by the mass and spin of the
  remnant object. Conversely, measurement of the QNM frequencies could
  be an independent test of the no-hair conjecture. This paper extends to spinning BHs earlier work that proposed to
  test the no-hair conjecture by measuring the complex QNM
  frequencies of a BBH ringdown using inspiral-merger-ringdown waveforms, thereby taking full advantage of the entire signal power and removing dependency on the
  predicted or estimated start time of the proposed ringdown. We
  further demonstrate the robustness of the test against modified
  gravitational-wave (GW) signals with a ringdown different from what
  GR predicts for Kerr BHs. Our method was used to analyse the
  properties of the merger remnants for the events observed by
  LIGO-Virgo in the first half of their third observing run (O3a) in
  the latest LIGO-Virgo publication. In this paper, for the first
  time, we analyse the GW events from the first (O1) and second (O2) LIGO-Virgo
  observing runs and provide joint constraints with published results
  from O3a. We also analyse two events from the O3a catalogue
  that were not considered in the initial LIGO-Virgo analysis. The
  joint measurements of the fractional deviations in the frequency and damping time,
  \macro{$\df{220} = 0.02^{+0.03}_{-0.03}$} and \macro{$\dtau{220} = 0.11 ^{+0.12} _{-0.12}$} are the strongest
  constraints yet using this method. %Finally, we also present a investigation into possible systematic effects due to an incomplete understanding of the interferometric noise around the GW event on the results of this test.

\end{abstract}
%%%%%%%%%%%%%%%%%%%%%%%

\maketitle

%%%%%%%%%%%%%%%%%
\section{Introduction}
\label{sec:intro}
\comment{AB: Please make the citations in the bibliography uniform (I would stick with citations
taken from InSpires and remove the URL links). We will be submitting to Physical Review D, so please
stick with American English. Also, please be uniform when writing captions, when using colors and style
of curves in plots, when building Tables. There is not a lot of discussion in the text on the main results of the paper (Sec. IV).
The abstract would need to be tighten and made more crispy once the quantitative results for the
bounds are included.}

\abhi{The main body can now be considered stable in terms of text and results. Some of the fine tuning mentioned above,eg, reference formatting, cosmetic changes, etc are currently ongoing.}

\abhi{All macros in the paper are currently highlighted in orange, and would require one final check before final draft.}

The LIGO Scientific Collaboration~\cite{lsc} and the Virgo
Collaboration~\cite{Virgo} have recently announced their catalogue of
gravitational-wave (GW) signals from the first
half of the third observing run (O3a)~\cite{Abbott:2020niy}. Combined
with the first and second observing-run catalogues~\cite{LIGOScientific:2018mvr}, the Advanced LIGO detectors at Hanford,
Washington and Livingston, Louisiana~\cite{Aasi:2014mqd},
and the Advanced Virgo detector in Cascina,
Italy~\cite{TheVirgo:2014hva} have now detected $50$ GW
events from the merger of compact objects like neutron stars and/or
black holes (BHs). Alongside independent claims of
detections~\cite{Nitz:2018imz,Nitz:2019hdf,Venumadhav:2019lyq,Zackay:2019btq}, these results have firmly established the field of GW astronomy, five years after the first detection of a GW passing through Earth, GW150914~\cite{Abbott:2016blz}.

The observation of GWs has had significant astrophysical and cosmological
implications~\cite{TheLIGOScientific:2016htt,GBM:2017lvd,Monitor:2017mdv,Abbott:2017xzu}. It
has also allowed us to probe fundamental
physics and test predictions of Einstein's theory of General Relativity
(GR) in the previously unexplored highly-dynamical, and strong field
regime~\cite{TheLIGOScientific:2016src,Abbott:2018lct,LIGOScientific:2019fpa,Abbott:2020jks}. In GR, a binary black hole (BBH) system is described by
three distinct phases: an early \textit{inspiral}, where the two
compact objects spiral in loosing energy because of the emission of GWs, a \textit{merger} marked by the
formation of a common apparent horizon, and a \textit{ringdown}, during which the newly formed remnant object settles down to a Kerr BH emitting quasi-normal-modes (QNMs)~\cite{Vishveshwara:1970zz,Vishveshwara:1970cc,Teukolsky:1973ha,Press:1973zz,Chandrasekhar:1975zza,Detweiler:1980gk,Buonanno:2000ef,Pretorius:2005gq,Campanelli:2005dd,Baker:2005vv,Kokkotas:1999bd,Berti:2009kk}\abhi{cite some of the other early QNM papers}\rb{added a couple of important early QNM papers but also the NR breakthrough papers and Alessandra's paper on the transition form inspiral to plunge. } (i.e., damped oscillations with specific, discrete frequencies and decay times).

The LIGO-Virgo collaborations have released companion papers detailing their results
of tests of GR for GW150914 \cite{TheLIGOScientific:2016src} and for several GW events of
the two transient catalogues (TC): GWTC-1 \cite{LIGOScientific:2018mvr,LIGOScientific:2019fpa}  and GWTC-2 \cite{Abbott:2020niy,Abbott:2020jks}.
The results include tests of GW generation and source dynamics, where bounds are placed on
parameterized deviations in the post-Newtonian (PN) coefficients describing
the early inspiral and
phenomenological coefficients describing the intermediate (plunge) and
merger regimes of coalescence~\cite{Arun:2006hn,Arun:2006yw,Agathos:2013upa,Abbott:2018lct}; tests of GW
propagation, which assume a generalized dispersion relation and place
upper bounds on the Compton wavelength and, consequently, the mass of
the graviton~\cite{Abbott:2017vtc,Samajdar:2017mka}, and tests of the
polarization of gravitational radiation using a
multi--GW-detector network~\cite{Abbott:2017oio,Isi:2017fbj}. The GWTC-1/2 papers also
check for consistency between different portions of the signal using estimates for the predicted mass and spin of the remnant
object~\cite{Ghosh:2016xx,Ghosh:2017gfp,TheLIGOScientific:2016src}, and
consistency of the residuals with interferometric
noise~\cite{Ghonge:2020suv,LIGOScientific:2019fpa}. None of these tests report any
departure from the predictions of GR.

The first paper on tests of GR by the LIGO Collaboration~\cite{TheLIGOScientific:2016src} provided us with the
first measurement of the dominant damped-oscillation signal in the ringdown stage of a BBH
coalescence, and more recently a similar measurement was made with the high-mass event GW190521~\cite{Abbott:2020tfl,Abbott:2020mjq}. This set of available measurements was greatly expanded in the latest LIGO-Virgo O3a testing GR paper~\cite{Abbott:2020jks} where a comprehensive analysis of the properties of the remnant object, including the ringdown stage, was reported for tens of GW events.\rb{slightly modified the previous two sentences to include also cite the GW190521 papers}
The no-hair conjecture in GR~\cite{Carter:1971zc,Hansen:1974zz} states that an (electrically neutral) astrophysical BH is completely
described by two observables: mass and spin angular momentum. One
consequence of the no-hair conjecture is that the (complex) QNM
frequencies of gravitational radiation emitted by a perturbed isolated
BH are uniquely determined by its mass and spin angular momentum. Hence
a test of the no-hair conjecture would involve checking for
consistency between estimates of mass and spin of the remnant object
across multiple QNM frequencies~\cite{Dreyer:2003bv}. An inconsistency would either
indicate a non-BH nature of the remnant object, or an incompleteness
of GR as the underlying theory of gravity.

The consistency between the post-merger signal and the least damped QNM was first demonstrated in
Ref.~\cite{TheLIGOScientific:2016src} for GW150914, and later extended to include
overtones in Refs.~\cite{Brito:2018rfr,Giesler:2019uxc,Isi:2019aib,Bhagwat:2019dtm,Forteza:2020hbw}. Consistency
of the late-time waveform with a single QNM is a test of the ringdown of
a BBH coalescence, but not necessarily a test of the no-hair
conjecture, which requires the measurement of (at least) two QNMs (BH
spectroscopy), and consistency between them~\cite{Dreyer:2003bv,Berti:2005ys}. Recent work
also include~\cite{Carullo:2018gah,Carullo:2019flw,Bhagwat:2019bwv}. The
nature of the remnant object has also been explored through tests of
BH thermodynamics, like the Hawking's area
theorem~\cite{Cabero:2017avf} or through search for echos in the
post-merger
signal~\cite{Nielsen:2018lkf,Tsang:2019zra,Lo:2018sep,Abedi:2018npz,Abedi:2020sgg,Testa:2018bzd}. None
of these tests have found evidence for non-BH nature of the remnant
object (as described in GR) in LIGO-Virgo BBH observations.

Ground-based detectors, such as LIGO and Virgo, are most sensitive to
stellar-mass BH binaries that merge near the
minimum of their sensitivity band ($\sim100$Hz). As a consequence, the
remnant object \textit{rings down} at frequencies where the sensitivity is worse, leaving
very little signal-to-noise ratio (SNR) in the
post-merger signal. Most of the tests mentioned above restrict the data to this post-merger signal, which signficantly reduces parameter estimation capabilities. However, added to a lack of SNR is the ambiguity in clearly defining a start time for the ringdown phase (discussed, for example in \cite{Bhagwat:2017tkm}). In some tests, for example \cite{Carullo:2018gah,Carullo:2019flw}, the ringdown start time have been left as a
free parameter to be estimated directly from the data. Whereas, in other cases, the ringdown start time is predicted from corresponding full-signal parameter estimation analyses (see \texttt{PyRING} analysis in \cite{Abbott:2020jks}). Uncertainties
in estimates of the ringdown start-time, as well as an overall lack of
SNR in the post-merger signal, given typical sensitivities of
ground-based detectors, result in significant statistical uncertainties
in the measurement of the QNM frequencies. And hence, one might want to look at alternate methods to measure QNMs using information from as much as the signal as possible.

An independent approach to BH spectroscopy, based on the full-signal
analysis, was introduced in Ref.~\cite{Brito:2018rfr}
(henceforth referred to as \paperone). Unlike methods that focus only
on the post-merger signal, it employs the
complete inspiral-merger-ringdown
(IMR) waveform to measure the complex QNM
frequencies. This gives access to the full SNR of the signal,
reducing measurement uncertainties. Moreover, the definition of the
ringdown start time is built into the merger-ringdown model and does not need to be
either left as an additional free parameter or fixed using alternate
definitions. While \paperone presented the method and tested it for non-spinning BBHs, here we extend the analysis to the more
realistic astrophysical case in which BHs carr spin. Furthermore,
the IMR waveform model used in this paper is more accurate than what employed in \paperone,
because it contains higher-order corrections in PN theory and it was calibrated to a much larger set
of numerical-relativity (NR) waveforms~\cite{Bohe:2016gbl}. All astrophysical BHs are expected to be spinning,
and ignoring effects of spin has been shown to introduce systematic biases in the measurement of the
source properties.

The rest of the paper is organized as follows. Section~\ref{sec:model} describes our parameterized IMR
waveform model. In Sec.~\ref{sec:method}, we define our framework to test GR, notably how we can measure
complex QNM frequencies with our parameterized model within a Bayesian formalism. Then, in Sec.~\ref{sec:results},
we demonstrate our method on real GW events, as well as simulated signals. In particular, we analyze
the GWTC 1/2  events and constrain the QNM frequecies to be \macro{$\df{220} = 0.02^{+0.03}_{-0.03}$} and \macro{$\dtau{220} = 0.11 ^{+0.12} _{-0.12}$}. Finally, in Sec.~\ref{sec:discussion} we provide a summary of
our results and discuss future developments.


%%%%%%%%%%%%%%%%%


%%%%%%%%%%%%%%%%%
\section{Waveform model and statistical strategy to measure quasi-normal modes}
A GW signal from the (quasi-circular) coalescence of two BHs is
completely described in GR by $15$ parameters,
$\bxigr$. These can be grouped into the \emph{intrinsic} parameters:
the (source) masses, $m_1, m_2$ and spins, $\bs1, \bs2$ of the component
objects in the binary; and the
\emph{extrinsic} parameters: a reference time $t_c$ and phase
$\phi_c$, the sky position of the binary ($\alpha$,
$\delta$), the luminosity distance, $d_L$, and the binary's orientation
described through the inclination of the binary $\iota$ and its
polarization $\psi$. We also introduce the total (source) mass $M = m_1+m_2$,
and the (dimensionless) symmetric mass ratio $\nu = m_1m_2/M^2$. When needed, 
we shall denote the detector masses as $m_{1\,\rm det}= (1+z)\,m_1$ and $m_{2 \,\rm det} = (1+z)\,m_2$, 
where $z$ is the redshift. \abhi{We also follow the convention $m_1 > m_2$, and hence the asymmetric mass ratio, $q = m_1/m_2 \geq1$.}

Here, we focus on BHs with spins aligned or anti-aligned
with the orbital angular momentum (henceforth, aligned-spin). In this case,
the GW signal depends on $11$ parameters. We denote the
aligned-spin (dimensionless) components as $\chi_{i} = |\vec{\bm{S}}_i|/m^2_i$, where $i=1,2$ for the two BHs.

\subsection{Parameterized waveform model}\label{sec:model}

As in \paperone, we use an IMR waveform model developed within the effective-one-body (EOB)
formalism~\cite{Buonanno:1998gg,Buonanno:2000ef}. However, whereas \paperone was limited to non-spinning multipolar waveforms,
here we use as our baseline model the aligned-spin multipolar waveform model
developed in Ref.~\cite{Cotesta:2018fcv}. In addition to being
calibrated to NR simulations, this model also uses information from BH
perturbation theory for the merger and ringdown phases. Henceforth we
will denote this model by $\SEOB$ for short~\footnote{In the LIGO Algorithm Library (LAL), this
waveform model is called {\tt SEOBNRv4HM}.}.


In the observer's frame, the GW polarizations can be written as
%
\begin{equation}
h_+(\iota,\varphi_0;t ) - i h_\times(\iota,\varphi_0;t) = \sum_{\ell, m} {}_{-\!2}Y_{\ell m}(\iota,\varphi_0)\, h_{\ell m}(t)\,,
\end{equation}
%
where $\varphi_0$ is the azimuthal direction to the observer (note that without loss of generality we can take $\phi_c\equiv\varphi_0$), while ${}_{-\!2}Y_{\ell m}(\iota,\varphi_0)$ are the $-2$ spin-weighted spherical harmonics. The $\SEOB$ model we employ includes the $(\ell, |m|)=(2,2),(2,1)$, $(3,3)$, $(4,4)$, and $(5,5)$ modes~\cite{Cotesta:2018fcv}. For each $(\ell, m)$, the inspiral-(plunge-)merger-ringdown $\SEOB$ waveform is schematically given by
%
\begin{equation}
h_{\ell m}(t) = h_{\ell m}^\mathrm{insp-plunge}\, \theta(t_\mathrm{match}^{\ell m} - t) + h_{\ell m}^\mathrm{merger-RD}\,\theta(t-t_\mathrm{match}^{\ell m})\,,
\end{equation}
where $\theta(t)$ is the Heaviside step function, $h_{\ell m}^\mathrm{insp-plunge}$ represents the inspiral-plunge part of the waveform, whereas $h_{\ell m}^\mathrm{merger-RD}$ denotes the merger-ringdown waveform, which reads~\citep{Bohe:2016gbl,Cotesta:2018fcv}
%
\begin{equation}
\label{RD}
h_{\ell m}^{\textrm{merger-RD}}(t) = \nu \ \tilde{A}_{\ell m}(t)\ e^{i \tilde{\phi}_{\ell m}(t)} \ e^{-i \sigma_{\ell m 0}(t-t_{\textrm{match}}^{\ell m})},
\end{equation}
%
where $\nu$ is the symmetric mass ratio of the binary and $\sigma_{\ell m0} = 2\pi f_{\ell m 0} -i/\tau_{\ell m 0}$ denotes the complex frequency of the fundamental QNMs of the remnant BH. We denote the oscillation frequencies by $f_{\ell m  0}\equiv \Re(\sigma_{\ell m0})/(2\pi)$ and the decay times by $\tau_{\ell m 0}\equiv -1/\Im(\sigma_{\ell m0}) $.
The functions $\tilde{A}_{\ell m}(t)$ and $\tilde{\phi}_{\ell m}(t)$ are given by~\cite{Bohe:2016gbl,Cotesta:2018fcv}:
%
\begin{subequations}
\begin{eqnarray}
\label{eq:ansatz_amp}
\tilde{A}_{\ell m}(t) &=& c_{1,c}^{\ell m} \tanh[c_{1,f}^{\ell m}\ (t-t_{\textrm{match}}^{\ell m}) \ +\ c_{2,f}^{\ell m}] \ + \ c_{2,c}^{\ell m},\\
\label{eq:ansatz_phase}
\tilde{\phi}_{\ell m}(t) &=& \phi_{\textrm{match}}^{\ell m} - d_{1,c}^{\ell m} \log\left[\frac{1+d_{2,f}^{\ell m} e^{-d_{1,f}^{\ell m}(t-t_{\textrm{match}}^{\ell m})}}{1+d_{2,f}^{\ell m}}\right],
\end{eqnarray}
\end{subequations}
%
where $ \phi_{\textrm{match}}^{\ell m}$ is the phase of the inspiral-plunge mode $(\ell, m)$ computed at $t = t_{\textrm{match}}^{\ell m}$. The coefficients $d_{1,c}^{\ell m}$ and $c_{i,c}^{\ell m}$ with $i = 1,2$
are fixed by imposing that the functions $\tilde{A}_{\ell m}(t)$ and $\tilde{\phi}_{\ell m}(t)$ are of class $C^1$ at $t = t_{\textrm{match}}^{\ell m}$, when matching the merger-ringdown waveform to the inspiral-plunge $\SEOB$ waveform $h_{\ell m}^\mathrm{inspiral-plunge}(t)$. This allows us to write the coefficients $c_{i,c}^{\ell m}$ as~\cite{Cotesta:2018fcv}:
%
\begin{subequations}
\begin{align}
\label{c1}
c_{1,c}^{\ell m} &= \frac{1}{c_{1,f}^{\ell
    m} \nu} \big[ \partial_t|h_{\ell
    m}^{\textrm{insp-plunge}}(t_{\textrm{match}}^{\ell m})| \nonumber \\
    &- \sigma^\textrm{R}_{\ell m} |h_{\ell
    m}^{\textrm{insp-plunge}}(t_{\textrm{match}}^{\ell
    m})|\big] \cosh^2{(c_{2,f}^{\ell m})}, \\
\label{c2}
c_{2,c}^{\ell m} &= -\frac{ |h_{\ell
    m}^{\textrm{insp-plunge}}(t_{\textrm{match}}^{\ell
    m})|}{\nu} + \frac{1}{c_{1,f}^{\ell
    m} \nu} \big[ \partial_t|h_{\ell
    m}^{\textrm{insp-plunge}}(t_{\textrm{match}}^{\ell m})|  \nonumber \\
    &- \sigma^\textrm{R}_{\ell m} |h_{\ell
    m}^{\textrm{insp-plunge}}(t_{\textrm{match}}^{\ell
    m})|\big] \cosh{(c_{2,f}^{\ell m})}\sinh{(c_{2,f}^{\ell m})}, \\ \nonumber
\end{align}
\end{subequations}
and $d_{1,c}^{\ell m}$ as
\begin{align}
\label{d1}
d_{1,c}^{\ell m} &= \left[\omega_{\ell m}^{\textrm{insp-plunge}}(t_{\textrm{match}}^{\ell m}) -  \sigma^\textrm{I}_{\ell
      m}\right]\frac{1+ d_{2,f}^{\ell m}}{d_{1,f}^{\ell m}d_{2,f}^{\ell m}}\,,
\end{align}
%
where we denoted $\sigma_{\ell m}^\textrm{R} \equiv \Im (\sigma_{\ell m0}) < 0$ and  $\sigma_{\ell m}^\textrm{I} \equiv -\Re (\sigma_{\ell m0})$, and $\omega_{\ell m}^{\textrm{insp-plunge}}(t)$ is the frequency of the inspiral-plunge EOB mode. The coefficients $c_{i,f}^{\ell m}$ and $d_{i,f}^{\ell m}$ are obtained through fits to NR and
Teukolsky-equation--based waveforms and can be found in Appendix C of Ref.~\cite{Cotesta:2018fcv}.

In the $\SEOB$ model constructed in Ref.~\cite{Cotesta:2018fcv}, the
complex frequencies $\sigma_{\ell m 0}$ are expressed in terms of the
final BH mass and spin~\cite{Berti:2005ys,Berti:2009kk}, and the
latter are related to the BBH's component masses and spins through
NR--fitting-formulas obtained in
GR~\cite{Taracchini:2013rva,Hofmann:2016yih}. Here instead, in the
spirit of what was done in \paperone, we promote the QNM (complex)
frequencies to be free parameters of the model, while keeping the
inspiral-plunge modes $h_{\ell m}^\mathrm{inspiral-plunge}(t)$ fixed
to their GR values. More explicitly, we introduce a parameterized
version of the $\SEOB$ model where the frequency and the
damping time of the ${\ell m 0}$ mode (i.e, $(f_{\ell m 0}, \tau
_{\ell m 0})$) are defined through the fractional deviations, $(\delta
f_{\ell m 0},\delta \tau_{\ell m 0})$, from the corresponding GR
values~\cite{Gossan:2011ha,Meidam:2014jpa}.


Thus,
\begin{subequations}
\begin{eqnarray}
f_{\ell m 0} &=& f_{\ell m 0}^{\text{GR}}\, (1 + \delta f_{\ell m 0})\,,\label{eq:nongr_freqs_a} \\ 
\tau _{\ell m 0} &=& \tau _{\ell m 0}^{\text{GR}}\, (1 + \delta \tau_{\ell m 0})\,. \label{eq:nongr_freqs_b}
\end{eqnarray}
\end{subequations}

\comment{AB: as discussed, please expalin very clearly how the GR frequency and decay times are computed, and mark in the equation above 
their dependence on the mass and spin parameters.}\abhi{The GR quantities $( f_{\ell m 0}^{GR},\tau_{\ell m 0}^{GR})$ are
constructed using the NR--fitting--formula from Refs.~\cite{Taracchini:2013rva,Hofmann:2016yih}, and are functions of the initial masses and spins, $(m_1, m_2, \chi_1, \chi_2)$. For every single instance of waveform generation, we choose a value for $(m_1, m_2, \chi_1, \chi_2, \delta f_{\ell m 0}, \delta \tau_{\ell m 0})$. While the mass and spin value go into the construction of $( f_{\ell m 0}^{\text{GR}}, \tau _{\ell m 0}^{\text{GR}})$, these GR quantities and then combined with the values of $(\delta f_{\ell m 0}, \delta \tau_{\ell m 0})$ as shown in the Eqs.~\ref{eq:nongr_freqs_a}-~\ref{eq:nongr_freqs_b}.. Hence,
\begin{subequations}
\begin{eqnarray}
f_{\ell m 0} &=& f_{\ell m 0}(m_1, m_2, \chi_1, \chi_2, \delta f_{\ell m 0}, \delta \tau_{\ell m 0})\\ 
\tau _{\ell m 0} &=& \tau _{\ell m 0}(m_1, m_2, \chi_1, \chi_2, \delta f_{\ell m 0}, \delta \tau_{\ell m 0})
\end{eqnarray}
\end{subequations}
We denote such a parameterized waveform model $\pSEOB$~\footnote{This
waveform model is called {\tt pSEOBNRv4HM} in LAL.}.} We note that when leaving $\sigma_{\ell m}$ to vary
freely, the functions $\tilde{A}_{\ell m}(t)$ and $\tilde{\phi}_{\ell
  m}(t)$ in general also differ from the GR predictions, since
those functions depend on the QNM complex frequencies, as can be seen
from the expressions for $c_{i,c}^{\ell m}$ and $d_{1,c}^{\ell m}$ in Eqs.~(\ref{c1}),
(\ref{c2}), and (\ref{d1}). As a consequence, the ringdown signal (amplitude and phase) 
soon after merger deviates from the one predicted by GR.


\subsection{Bayesian parameter-estimation technique}
\label{sec:method}

The parameterized model, $\pSEOB$, described above introduces an additional set of non-GR parameters, $\bxingr = (\delta f_{\ell m 0},\delta \tau_{\ell m 0})$, corresponding to each $(\ell,m)$ QNM present in the GR waveform model $\SEOB$. One then proceeds to use the Bayes theorem to obtain the \emph{posterior} probability distribution on $\blambda = \{\bxigr, \bxingr\}$, given a hypothesis $\mathcal{H}$:
%
\begin{equation}
P(\blambda | d, \mathcal{H}) = \frac{P(\blambda | \mathcal{H}) \, \mathcal{L}(d | \blambda, \mathcal{H})}{P(d|\mathcal{H})},
\label{eq:Bayes_theorem}
\end{equation}
%
where $P(\blambda | \mathcal{H})$ is the \emph{prior} probability distribution, and $\mathcal{L}(d | \blambda, \mathcal{H})$ is called the \emph{likelihood} function. The denominator is a normalization constant $P(d|\mathcal{H}) = \int P(\blambda | \mathcal{H}) \, \mathcal{L}(d | \blambda, \mathcal{H}) \, d\blambda$, called the marginal likelihood, or the \emph{evidence} of the hypothesis $\mathcal{H}$. In this case, our hypothesis $\mathcal{H}$ is that the data contains a GW signal that is described by the $\pSEOB$ waveform model $h(\blambda)$  and stationary Gaussian noise described by a power spectral density (PSD) $S_n(f)$. The likelihood function can consequently be defined as:
%
\begin{equation}
\mathcal{L}(d | \blambda, \mathcal{H}) \propto \exp\big[-\frac{1}{2} \langle d - h(\blambda) \, | \, d -h(\blambda) \rangle \big],
\label{eq:likelihood}
\end{equation}
%
where $\langle . | . \rangle$ is the usual noise-weighted inner product:
%
\begin{equation}
\langle A | B \rangle = \int_{f_\mathrm{low}} ^{f_\mathrm{high}} df \frac{\tilde{A}^*(f)\tilde{B}(f) + \tilde{A}(f)\tilde{B}^*(f)}{S_n(f)}.
\label{eq:nwip}
\end{equation}
%
The quantity $\tilde{A}(f)$ denotes the Fourier transform of $A(t)$ and the $^*$ indicates complex conjugation. The limits of integration ${f_\mathrm{low}}$ and ${f_\mathrm{high}}$ define the bandwidth of the sensitivity of the GW detector. We usually assume ${f_\mathrm{high}}$ to be the Nyquist frequency whereas ${f_\mathrm{low}}$ is dictated by the performance of the
GW detector at low-frequency. Here, we follow the choice made in the LIGO-Virgo analysis~\cite{LIGOScientific:2018mvr,Abbott:2020niy}. Owing to the large dimensionality of the parameter set $\blambda$, the posterior distribution $P(\blambda | d, \mathcal{H})$ in Eq.~(\ref{eq:Bayes_theorem}) is computed by stochastically sampling the parameter space using techniques such as Markov-chain Monte Carlo (MCMC)~\cite{Metropolis:1953am,Hastings:1970aa} or Nested Sampling~\cite{Skilling:2006gxv}. For this paper, we use the \verb+LALInference+~\cite{Veitch:2014wba} and \verb+Bilby+ codes~\cite{Ashton:2018jfp,Smith:2019ucc} that provide an implementation of the parallely tempered MCMC and Nested Sampling algorithms respectively, for computing the posterior distributions.

Given the full-dimensional posterior probability density function $P(\blambda | d, \mathcal{H})$, we can marginalize over the \emph{nuisance} parameters, to obtain the marginalized posterior probability density function over the QNM parameters $\bxingr$:

\begin{equation}
P(\bxingr | d, \mathcal{H})= \int P(\blambda | d, \mathcal{H}) d\bxigr\,.
\end{equation}

For most of the results discussed in this paper, we restrict ourselves
to the $(\ell m) = (2,2)$ and/or $(3,3)$ modes. In those cases we assume $\bxingr = \{\df{220},\dtau{220}\}$ and/or $
\{\df{330},\dtau{330}\}$, and \ab{fix} all the other $(\ell m)$ modes to \sout{be
fixed at} their GR predictions (i.e., $\delta f_{\ell m 0} = \delta
\tau_{\ell m 0} = 0$). This is because, for most of the high-mass BH
events that we find most appropriate for this test, the LIGO-Virgo
observations are consistent with nearly--equal-mass face-on/off BBHs
for which power in the subdominant modes is not enough to
attempt to measure more than one QNM complex frequency.

Lastly, throughout our analysis, we assume uniform priors on our non-GR QNM
parameters, $(\delta f_{\ell m 0},\delta \tau_{\ell m 0})$. We note that
since the priors on $( f_{\ell m 0}^{GR},\tau_{\ell m 0}^{GR})$ are
derived through NR--fits, from the corresponding priors on the initial
masses and spins, this leads to a non-trivial prior on the final
reconstructed frequency and damping time, $( f_{\ell m 0},\tau_{\ell m
  0})$. Also, given the definition of the damping time in
Sec.~\ref{sec:model}, we note that $\delta \tau_{\ell m 0} = -1$ leads
to the imaginary part of the QNM complex frequency going to infinity. We avoid
this by restricting the minimum of the prior on $\delta \tau_{\ell m
  0}$ to be greater than $-1$.

\iffalse
\subsection{Priors}

Throughout our analysis we assume a completely prior uniform in the component masses $m_1, m_2$. Our prior on the spins are uniform in the magnitude between (0,1) and isotropic in spin orientation, but finally restricted to the component parallel to the orbital angular momentum of the binary. The prior on the distance varies as $d_L^2$ giving more weightage to binaries farther out. For the rest of the parameters we use standard priors as defined in the documentation (CITE Vietch et al. 2015 LALInference paper). For our non-GR ringdown parameters, we assume uniform priors. This of course leads to a non-trivial prior on the reconstructed frequency and damping time, because of the prior on $( f_{\ell m 0}^{GR},\tau_{\ell m 0}^{GR})$, which itself depend on the prior on the initial masses and spins through NR fits \abhi{perhaps figure on priors on QNM quantities}. For $d\tau = -1$, we encounter a singularity (the imaginary part of the frequency goes to infinity), which we avoid by restricting the minimum of the prior on $d\tau$ to be greater than $-1$.
\fi

%%%%%%%%%%%%%%%%%


%%%%%%%%%%%%%%%%%
\ab{\section{Synthetic-signal injection study}}
\label{sec:results}
\subsection{Simulations using GR signals in Gaussian noise} \label{ssec:gr_signal}

We demonstrate our method using synthetic-signal injections describing GWs
from BBHs in GR. We employ coloured Gaussian noise with PSDs expected for LIGO and
Virgo detectors at design sensitivities~\cite{AdvLIGOPSD,TheVirgo:2014hva}.
For the mock BBH signals, we choose parameters similar to two specific GW events, GW150914~\cite{Abbott:2016blz} and
GW190521~\cite{Abbott:2020tfl}. We list them in Table~\ref{tab:injection_values}.
These two binary systems are representative of the kind of systems for which
the QNM measurement is most suitable, notably high-mass BBH events which are loud enough that the
pre- and post-merger SNRs return reliable parameter-estimation results.

%%%%%%%%%%%%%%%%%%%%%%%%%%%%%%%%%%%%%%%%%%%%%%%%%%%%%%%%%%%%%%%
% Table for Injections
%%%%%%%%%%%%%%%%%%%%%%%%%%%%%%%%%%%%%%%%%%%%%%%%%%%%%%%%%%%%%%%
\begin{table}[h!]
\begin{center}
\begin{tabular}{ c|c|c|c|c|c|c|c }

 \ab{synthetic signal} & \makecell{$m_{\rm 1,det}$ \\$(\Mo)$} &  \makecell{$m_{\rm 2,det}$ \\ $(\Mo)$} & $\chi_{1}$ & $\chi_{2}$ & $\rho_\text{IMR}$ & $\rho_\text{insp}$ & $\rho_\text{postinsp}$ \\
 \hline
 GW150914-like & 39 & 31 & 0.0 & 0.0 & 25 & 22 & 12 \\
 GW190521-like & 150 & 120 & 0.02 & -0.39 & 20 & 8 & 18 \\
 SXS:BBH:0166 & 72 & 12  & 0.0 & 0.0 & 71 & 58 & 41 \\

\end{tabular}
\caption{Parameters of the synthetic-signal injections, chosen to be similar to the actual GW events indicated in the first column (first two rows). The parameters $(m_{\rm 1,det},m_{\rm 2,det})$ are the detector-frame masses of the primary and secondary BHs, respectively. The third row indicates the parameters of the SXS BBH waveform used in Sec.~\ref{ssec:nohairtheorem}. $\rho_\text{IMR}$, $\rho_\text{insp}$ and $\rho_\text{postinsp}$ are the SNR of the full IMR signal, SNR upto a certain cutoff frequency, and SNR after the cutoff frequency respectively. The cutoff frequency is assumed to be the frequency at the innermost circular stable orbit (ISCO) corresponding to the remnant Kerr black hole in each case.}
\label{tab:injection_values}
\end{center}
\end{table}
%%%%%%%%%%%%%%%%%%%%%%%%%%%%%%%%%%%%%%%%%%%%%%%%%%%%%%%%%%%%%%%
%%%%%%%%%%%%%%%%%%%%%%%%%%%%%%%%%%%%%%%%%%%%%%%%%%%%%%%%%%%%%%%


%%%%%%%%%%%%%%%%%%%%%%%%%%%%%%%%%%%%%%%%%%%%%%%%%%%%%%%%%%%%%%%
% Simulated siganl: GR
%%%%%%%%%%%%%%%%%%%%%%%%%%%%%%%%%%%%%%%%%%%%%%%%%%%%%%%%%%%%%%%
\begin{figure*}[hbt]
\begin{center}
        \includegraphics[width=0.5\textwidth]{figures/GW150914_simulated_signal_0p0_deltaf220_deltatau220.png}\includegraphics[width=0.5\textwidth]{figures/GW150914_simulated_signal_0p0_f220_tau220.png}
        \includegraphics[width=0.5\textwidth]{figures/GW190521_simulated_signal_0p0_deltaf220_deltatau220.png}\includegraphics[width=0.5\textwidth]{figures/GW190521_simulated_signal_0p0_f220_tau220.png}
        \caption{\textcolor{red}{FINAL RESULT} Posterior probability distribution on the fractional deviations in the frequency and damping time of the $(2,2)$ QNM, $(\df{220},\dtau{220})$ (left panels) and the reconstructed quantities, $(\fngr{220}, \taungr{220})$ (right panels) for GR injections with initial parameters similar to GW150914 (top panels) and GW190521 (bottom panels) (Table~\ref{tab:injection_values}). The 2D contour marks the 90\% credible region, while the dashed lines on the 1D marginalized distributions mark the 90\% credible levels. The black vertical and horizontal lines mark the injection values.}
        \label{fig:simulated_signal_GR}
\end{center}
\end{figure*}
%%%%%%%%%%%%%%%%%%%%%%%%%%%%%%%%%%%%%%%%%%%%%%%%%%%%%%%%%%%%%%%
%%%%%%%%%%%%%%%%%%%%%%%%%%%%%%%%%%%%%%%%%%%%%%%%%%%%%%%%%%%%%%%

To avoid possible systematic biases in our parameter-estimation analysis
due to error in waveform modeling, we use the GR version of the same waveform,
$\SEOB$ (without allowing for deviations in the QNM parameters) to
simulate our GW signal. And to avoid systematic biases due to noise,
we use an averaged (zero-noise) realization of the noise. \footnote{A detailed
study on noise systematics for one of the GW events is presented in
Appendix~\ref{sec:noise_systematics}.}  As in the case of the actual
detections, we consider a two-detector LIGO network at
Hanford and Livingston, having identical PSDs. The distance to the two 
synthetic events is rescaled such that the SNR in the detector network
is the same as the actual events (i.e., 24 for GW150914 and 14
for GW190521). Since mearly--equal-mass binaries like GW150914 and
GW190521 observed at moderately high SNRs are not expected to have a
loud ringdown signal, we restrict ourselves to estimating the
frequency and damping time of only one QNM $(\ell m) = (2,2)$, i.e.,
$\{\df{220},\dtau{220}\}$, while fixing the other QNM frequencies to
their GR values.

We find, as one might expect, that the posterior distribution on the
parameters describing fractional deviations in the frequency and
damping time are consistent with zero (left panels of
Fig.~\ref{fig:simulated_signal_GR}). One can then convert these
fractional quantities into absolute quantities using the relations
given in Eqs.~\ref{eq:nongr_freqs_a} and ~\ref{eq:nongr_freqs_b}, and
construct posterior distributions on these effective quantities,
$(\fngr{220}, \taungr{220})$ (right panels of
Fig.~\ref{fig:simulated_signal_GR}). In each of these cases, the recovered
two-dimensional posteriors are consistent with the GR predictions
(black dashed lines).


\subsection{Simulations using non-GR signals in Gaussian noise} \label{ssec:ngr_signal}


To demonstrate the robustness of the method in detecting possible
deviations from GR, we inject synthetic GW signals which are identical to
the corresponding GR prediction up to merger, and differ in their post-merger
description. We again choose binary-parameters
similar to GW150914 and GW190521 (as given in Table ~\ref{tab:injection_values}), but
set $\df{220} = \dtau{220} = 0.1 $.
In other words, we assume that the frequency and damping time
of our non-GR signal is 10\% more than the corresponding GR prediction,
although the pre-merger signal is identical to GR. In Fig.~\ref{fig:nongr_waveform}
we show this non-GR waveform, \texttt{pSEOBNR} with respect to the 
original GR template, \texttt{SEOBNR}. We see that the waveforms are identical in amplitude
and instanteneous frequency upto the merger (lower panel) , beyond which the 
red (GR template) and blue (non-GR template) diverge. We summarize the results of the Bayesian analysis in Fig.~\ref{fig:simulated_signal_nonGR} where we
show the posterior probability distributions for $(\df{220}, \dtau{220})$, or equivalently
$(\fngr{220}, \taungr{220})$. We find that they are consistent with the corresponding
values of the injection parameters, indicated by the black dashed lines.

%%%%%%%%%%%%%%%%%%%%%%%%%%%%%%%%%%%%%%%%%%%%%%%%%%%%%%%%%%%%%%%
%%%%%%%%%%%%%%%%%%%%%%%%%%%%%%%%%%%%%%%%%%%%%%%%%%%%%%%%%%%%%%%
\begin{figure}
        \includegraphics[width=0.5\textwidth]{figures/modGR_waveforms_amplitudephase.png}
        \caption{\textcolor{red}{FINAL RESULT} Top panel: The `+'--polarization of the gravitational waveform $h_+(t)$ from a GW150914-like event where the post-merger is described by GR (i.e., $\df{220} = \dtau{220} = 0$), and where the merger-ringdown is modified (i.e., $\df{220} = \dtau{220} = 0.1$). Bottom panel: Comparison of the evolution of the amplitude, $\tilde{h}(t)$ (left) and instantaneous frequency, $f(t)$ (right) for the GR and non-GR signal.}
        \label{fig:nongr_waveform}
\end{figure}
%%%%%%%%%%%%%%%%%%%%%%%%%%%%%%%%%%%%%%%%%%%%%%%%%%%%%%%%%%%%%%%
%%%%%%%%%%%%%%%%%%%%%%%%%%%%%%%%%%%%%%%%%%%%%%%%%%%%%%%%%%%%%%%

%%%%%%%%%%%%%%%%%%%%%%%%%%%%%%%%%%%%%%%%%%%%%%%%%%%%%%%%%%%%%%%
% Simulated siganl: non-GR
%%%%%%%%%%%%%%%%%%%%%%%%%%%%%%%%%%%%%%%%%%%%%%%%%%%%%%%%%%%%%%%
\begin{figure*}%[h!]
\begin{center}
        \includegraphics[width=0.5\textwidth]{figures/GW150914_simulated_signal_0p1_deltaf220_deltatau220.png}\includegraphics[width=0.5\textwidth]{figures/GW150914_simulated_signal_0p1_gr_ngr_fngrtaungr.png}
        \includegraphics[width=0.5\textwidth]{figures/GW190521_simulated_signal_0p1_deltaf220_deltatau220.png}\includegraphics[width=0.5\textwidth]{figures/GW190521_simulated_signal_0p1_gr_ngr_fngrtaungr.png}
        \caption{\textcolor{red}{FINAL RESULT} Posterior probability distribution on the fractional deviations in the frequency and damping time of the $(2,2)$ QNM, $(\df{220},\dtau{220})$ (left panels) and the reconstructed quantities, $(\fngr{220}, \taungr{220})$ (right panels) for non-GR injections with parameters of GW150914-like (top panels) and GW190521-like (bottom panels) as given in Table~\ref{tab:injection_values}. The non-GR signal has a deviation, $\df{220} = \dtau{220} = XX$. The 2D contour marks the 90\% credible region, while the dashed lines on the 1D marginalized distributions mark the 90\% credible levels. The black vertical and horizontal lines mark the injection values. In the right panels, we additionally show measurements using a GR ($\SEOB$) waveform, for the
GW150914-like (upper panel) and GW190521-like (lower panel) injections. The measurements with $\SEOB$ waveforms are visibly biased.}
        \label{fig:simulated_signal_nonGR}
\end{center}
\end{figure*}
%%%%%%%%%%%%%%%%%%%%%%%%%%%%%%%%%%%%%%%%%%%%%%%%%%%%%%%%%%%%%%%
%%%%%%%%%%%%%%%%%%%%%%%%%%%%%%%%%%%%%%%%%%%%%%%%%%%%%%%%%%%%%%%

We additionally investigate the effects of erroneously assuming that
an underlying non-GR signal can be well-described
by a GR one. We do this by estimating the parameters of our non-GR
signals using the GR waveform model $\SEOB$ instead of the parameterized $\pSEOB$.  The resulting one- and two-dimensional posteriors are shown in the right panels of Fig.~\ref{fig:simulated_signal_nonGR} by red curves for the GW150914-like (top) and GW190521-like (bottom) signals respectively. For both signals, we find the $\SEOB$ estimates are markedly biased with respect to the $\pSEOB$ estimates. This can be explained by the inability of the $\SEOB$ template waveform to capture all the physics in the $\pSEOB$ signal. We also notice that the results are distinctly
different for the two events. For the GW150914-like non-GR signal, the measurements of $(\fgr{220}, \taugr{220})$
(top right panel in Fig.~\ref{fig:simulated_signal_nonGR}) are consistent
with the $(\fngr{220}, \taungr{220})$ measurements for a signal with \emph{no
deviations} from GR (top right panel in Fig.~\ref{fig:simulated_signal_GR}). In other
words, if the actual signal had deviations from GR as large as the $10\%$, the analysis with the GR signal $\SEOB$ would likely have
reported \emph{no} deviation from the GR prediction. However in the
case of the GW190521-like non-GR signal, a simple GR analysis of
the non-GR signal would have yielded measurements distinctly
different from either of the two parameterized estimates: with and
without deviations. \comment{AB: Not clear to me the next sentence. We are working
in Gaussian noise, so why are you referring to the noise as the culprit or the
source of difference?} The fact that the GW190521-like signal has a much
lower SNR than GW150914-like signal might be a possible reason for the
measurement of the final quantities to be more susceptible to noise. A
more detailed comparison of the other parameters, like the masses and
spins, between an $\pSEOB$ and an $\SEOB$ measurement of a modified GR
signal is shown in Fig.~\ref{fig:gr_ngr_comparison}. \comment{AB: Please
comment the figures. What are the main results? What do we deduce?}


%%%%%%%%%%%%%%%%%%%%%%%%%%%%%%%%%%%%%%%%%%%%%%%%%%%%%%%%%%%%%%%
% modified GR signal: GR vs nonGR recovery comparison
%%%%%%%%%%%%%%%%%%%%%%%%%%%%%%%%%%%%%%%%%%%%%%%%%%%%%%%%%%%%%%%
\begin{figure*}%[h!]
        \includegraphics[width=0.5\textwidth]{figures/GW150914_simulated_signal_0p5_gr_ngr_m1m2.png}\includegraphics[width=0.5\textwidth]{figures/GW190521_simulated_signal_0p5_gr_ngr_m1m2.png}
        \includegraphics[width=0.5\textwidth]{figures/GW150914_simulated_signal_0p5_gr_ngr_a1za2z.png}\includegraphics[width=0.5\textwidth]{figures/GW190521_simulated_signal_0p5_gr_ngr_a1za2z.png}
        \includegraphics[width=0.5\textwidth]{figures/GW150914_simulated_signal_0p5_gr_ngr_fgrtaugr.png}\includegraphics[width=0.5\textwidth]{figures/GW190521_simulated_signal_0p5_gr_ngr_fgrtaugr.png}
        \includegraphics[width=0.5\textwidth]{figures/GW150914_simulated_signal_0p5_gr_ngr_Mfaf.png}\includegraphics[width=0.5\textwidth]{figures/GW190521_simulated_signal_0p5_gr_ngr_Mfaf.png}
        \caption{\textcolor{red}{FINAL RESULT} Comparison of \sout{the recovered} \ab{binary's} parameters when \sout{the underlying signal is assumed to be a GR signal or a modified GR signal} \ab{a non-GR signal ($\pSEOB$) is injected and recovered with a GR ($\SEOB$) or a non-GR ($\pSEOB$) waveform model}. \sout{In both cases, the actual underlying signal is a modified GR signal with parameters similar to} \ab{The left (right) panels refer to
a GW150914-like (GW190521- like) injected signal (see Table~\ref{tab:injection_values}) with QNM deviation parameters of $\df{220} = \dtau{220} = 0.5$.} \sout{For the GW150914 (GW190521) contours, the $SEOB$ and $pSEOB$ recoveries are indicated by blue (red) and pink (grey) curves respectively.} The panels (from top to bottom) show the recoveries in (detector-frame) \sout{initial} masses (first row), \sout{z-components of dimensionless initial} \ab(dimensionless) spins (second row), GR predictions of frequency and damping time (third row) and the remnant mass and spin predictions ($M_f$, $a_f$) from the frequency and damping time
(obtained by inverting the Berti fits). \comment{AB: I would write those details about the ``Berti's fits, etc.'' in the text, not in the caption. Also please specify if the remnant mass is the detector-frame mass. Please
switch the red and blue colors. The blue colors should always refer to the same waveform model, i.e., $\pSEOB$. Please indicate on the figures: GW150914-like and GW190521-like.}}
        \label{fig:gr_ngr_comparison}
\end{figure*}
%%%%%%%%%%%%%%%%%%%%%%%%%%%%%%%%%%%%%%%%%%%%%%%%%%%%%%%%%%%%%%%
%%%%%%%%%%%%%%%%%%%%%%%%%%%%%%%%%%%%%%%%%%%%%%%%%%%%%%%%%%%%%%%


\subsection{Test of the no-hair conjecture}\label{ssec:nohairtheorem}

Finally, we provide a simple demonstration of a test of the no-hair
theorem using our model. As described in the introduction, any test of
the no-hair theorem of BHs would need to involve independent
measurements of (at least) two different QNMs.

Here, we use an NR GW signal from the SXS catalog~\cite{Mroue:2013xna}
corresponding to a non-spinning BBH with mass-ratio $q=6$ (SXS:BBH:0166) and 
total mass $M=84 \Mo$ (see Table~\ref{tab:injection_values}).
We choose an asymmetric system to increase the SNR in the higher modes.
We also choose the distance and orientation of the binary
such that the total SNR in the three-detector network of LIGO Hanford, Livingston and
Virgo, is \macro{$\sim$ 70}. Based on the LIGO-Virgo observations during during the first three observing runs, 
such asymmetric and loud signals are no longer just a theoretical
prediction, but quite plausible at design sensitivities. Using this
signal, we attempt to measure both the $(2,\pm 2)$ and
$(3,\pm 3)$ QNMs. For this injected signal the SNR in
other sub-dominant modes is too low to be able to measure them.

We summarize our results in Fig.~\ref{fig:nohair_sxs}.  Given the injection parameters, the predicted values of the $(2,\pm 2)$ and $(3,\pm 3)$ frequency and damping time are \macro{(169.45 Hz, 4.68 ms)}  and \macro{(271.21 Hz, 4.50 ms)} respectively. The left panel of Fig.~\ref{fig:nohair_sxs} shows that the two-dimensional posteriors on the $(2,\pm 2)$ and $(3,\pm 3)$ QNMs are consistent with the predictions for a BBH merger in GR, indicated by the black plus sign.  Using fitting formulae provided in ~\cite{Berti:2005ys}, specifically,  Eqs. 2.1, E1, E3 and tables VIII and IX for the fitting coefficients, we infer the two-dimensional posterior probability distirbution on the final mass and spin for the $(2,\pm 2)$ (blue) and $(3,\pm 3)$ (red) QNMs in the right panel of Fig.~\ref{fig:nohair_sxs}. The two independent estimates are consistent with each other and correspond to a unique mass and spin for the remnant BH \macro{(83.08 $\Mo$, 0.37)} indicated by the plus sign. As a consequence, this may be considered as a test of the no-hair conjecture. For most of the events observed so far, the power in the $(3,\pm 3)$ has not been sufficient to measure it along with the $(2,\pm 2)$, or in fact, in its place. However, it might also be possible to combine information from multiple observation over the coming few years to obtain meaningful constraints on the $(3,\pm 3)$ and
other sub-dominant QNMs.

%%%%%%%%%%%%%%%%%%%%%%%%%%%%%%%%%%%%%%%%%%%%%%%%%%%%%%%%%%%%%%%

%%%%%%%%%%%%%%%%%%%%%%%%%%%%%%%%%%%%%%%%%%%%%%%%%%%%%%%%%%%%%%%
\begin{figure}
        \includegraphics[width=0.5\textwidth]{figures/nohair_sxs_0166.png}
        \caption{\textcolor{red}{PRELIMINARY} Posterior probability distribution on the fractional deviations (left panel) and the reconstructed (right panel) frequency and damping time of the $(2,\pm 2)$ (blue curves) and $(3,\pm 3)$ (red curves) QNM, respectively, when a NR signal with parameters $q=6$,  $M=84 \Mo$ and SNR $=75$ is injected in Gaussian noise and recovered with the $\pSEOB$ waveform model. The plus signs mark the GR predictions.}
        \label{fig:nohair_sxs}
\end{figure}
%%%%%%%%%%%%%%%%%%%%%%%%%%%%%%%%%%%%%%%%%%%%%%%%%%%%%%%%%%%%%%%
%%%%%%%%%%%%%%%%%%%%%%%%%%%%%%%%%%%%%%%%%%%%%%%%%%%%%%%%%%%%%%%

%%%%%%%%%%%%%%%%%


%%%%%%%%%%%%%%%%%
\ab{\section{Constraints on QNM frequencies using LIGO-Virgo data}}
\comment{AB: This is the most important section in the paper, because it contains the
main results of the paper, i.e., the constraints on QNMs using real data. It needs to
be expanded and results would need to be highlighted much better than in the current version.
I would like to suggest that besides the Table and the figure with the 2D- and 1D-posteriors
that you made (which can be contrasted with Fig. 14 of the TGR GWTC-2 paper), we could have
a new figure that is either a combination of Figs. 5 and 6 or only Fig. 5 of the
TGR GWTC-2, but for the 22 QNM frequencies for the different GW events.
Figures can be used in talks, and can capture the information much better
than a list of numbers in a Table.}

The LIGO-Virgo Collaboration recently released their testing GR
catalogue containing results \sout{of this test} for all events observed
during O3a~\cite{Abbott:2020jks}, which passed a threshold for the
total (source-frame) mass $\geq 50 \Mo$ and SNRs in the pre- and
post-merger regions $\geq 8$.The pre- and post-merger regions of the
signal are identified \sout{as} \ab{from} the \ab{signal's} power \sout{in the frequency content of the
signal} before and after the signal reaches \ab{the} peak\ab{'s}
amplitude, \sout{as} \ab{which is} determined by the maximum \ab{of the}
likelihood \ab{function}  \ab{from the} parameter-estimation \ab{analysis} \sout{template}.
\comment{AB: What does it mean ``for the purpose of Ref.''? We are using present tense in the
entire paper, except perhaps the Conclusions section.}
For the purpose of Ref.~\cite{Abbott:2020jks}, we use\sout{d} a mass
threshold and restrict\sout{ed} ourselves to the highest-mass events which
\sout{were} \ab{are} expected to be most promising to study merger-ringdown. However,
since our method relies on doing a parameter estimation on the entire
inspiral-merger-ringdown signal, we require the SNR to be beyond a
certain threshold throughout the signal for reliable parameter
estimation of the initial and final quantities. \comment{AB: there are not ``initial'' and ``final'' parameters,
there are the component masses of the holes in the binary, and the
mass of the remnant formed after merger.} In fact the SNR
threshold alone should be sufficient for the analysis, and for this
paper we have relaxed the mass threshold. This has added two events,
GW190630$\_$185205 and GW190828$\_$063405, to the list of GW events
considered in \cite{Abbott:2020jks} \abhi{Runs ongoing and looking
  promising}. Fig.~\ref{fig:o1o2_events} shows results from these two
events along with GW190519$\_$153544, GW190521$\_$074359,
GW190910$\_$112807. \comment{AB: Please improve the above paragraph
and make it clearer which choice is made for the SNR threshold and
masses, and why.}

\ab{Furthermore}, for the first time, in this paper, we \ab{apply our method to
measure the QNMs to} \sout{present results on} the relevant
GW events from LIGO-Virgo's O1 and O2 runs, alongside
the above events. Applying the same thresholds as above, we find three
additional events that could be included in the analysis: GW150914,
GW170104, GW170729. The other high-mass events from O1-O2, GW170809,
GW170814, GW170818 and GW170823 do not have an SNR of $8$ in the
merger-ringdown signal. For the three relevant signals, GW150914,
GW170104 and GW170729, we \sout{provide} \ab{show} the posterior distributions in
$(\df{220}, \dtau{220})$ in Fig.~\ref{fig:o1o2_events}. We also
reconstruct the \sout{effective} QNM parameters, $(\fngr{220}, \taungr{220})$
which are tabulated in Table~\ref{tab:qnm_o1o2_results} \footnote{See
  Table IV of Ref.~\cite{Abbott:2020jks} for a list of the SNR
  thresholds. The paper quotes them for the purpose of the IMR
  consistency test, but the same thresholds have been used for the
  $pSEOB$ test, as well.} Among all the GW signals detected so far,
GW150914 (solid curve in Fig.~\ref{fig:o1o2_events}) is unique in its
loudness, \sout{mass as well as the clarity of the signal,} \comment{AB: what
is the ``clarity'' of a signal?} leading to the
first, and to date, best attempt in measuring the QNM frequencies~\cite{}.

%%%%%%%%%%%%%%%%%%%%%%%%%%%%%%%%%%%%%%%%%%%%%%%%%%%%%%%%%%%%%%%
% O1-O2 events
%%%%%%%%%%%%%%%%%%%%%%%%%%%%%%%%%%%%%%%%%%%%%%%%%%%%%%%%%%%%%%%
\begin{figure}[h!]
        \includegraphics[width=0.5\textwidth]{figures/rin_pseob_results_v2.pdf}
        \caption{\textcolor{red}{PRELIMINARY RESULT} The 90\% credible levels of the posterior probability distribution of the fractional deviations in the frequency and damping time of the $(2,\pm 2)$ QNM, $(\df{220},\dtau{220})$ and their corresponding one-dimensional marginalized posterior distributions, for events from O1, O2 and O3a passing a SNR threshold of $8$ in both the pre- and post-merger signal. The solid red curve marks the best single-event constraint, GW150914, whereas the contraints from the other events are indicated by the dash-dot curves. The joint constraints on $(\df{220},\dtau{220})$ obtained multiplying the likelihoods from individual events is given by the filled grey contours, while the hierarchical method of combination yields the black dot dashed curves.}
        \label{fig:o1o2_events}
\end{figure}
%%%%%%%%%%%%%%%%%%%%%%%%%%%%%%%%%%%%%%%%%%%%%%%%%%%%%%%%%%%%%%%
%%%%%%%%%%%%%%%%%%%%%%%%%%%%%%%%%%%%%%%%%%%%%%%%%%%%%%%%%%%%%%%

Finally, if we assume that the fractional deviations $(\df{220},
\dtau{220})$ take the same values in multiple events, we can assume
the posterior of one event to be the prior for the next, and obtain a
joint posterior probability distribution. For $N$ observations, where
$P_j(\df{220}, \dtau{220} | d_j)$ is the posterior for the $j$-th
observation corresponding to the data set $d_j$, $j=1,\dots,N$, the joint
posterior is given by:
%
\begin{equation}
P(\df{220}, \dtau{220} | \{d_j\}) = P(\df{220}, \dtau{220}) \prod _{j=1}^N \frac{P(\df{220}, \dtau{220} | d_j) }{P(\df{220}, \dtau{220})}
\end{equation}
%
where $P(\df{220}, \dtau{220})$ is the prior on $(\df{220},
\dtau{220})$. However, since we assume the prior on $(\df{220},
\dtau{220})$ to be flat (or uniform), the joint posterior is equal to
the joint likelihood.

We show these joint likelihoods on $(\df{220}, \dtau{220})$, as well as, the corresponding \ab{1D} marginalized distributions as filled grey curves in Fig.~\ref{fig:o1o2_events}.
These are the strongest constraints on possible deviations in the measurement of $(\df{220}, \dtau{220})$ to date using our method.\comment{AG: add text on the hierarchical analysis
if we want to include that analysis. AB: we want to include that analysis.}

\comment{AB: Please write in the text (equations) the results for $(\df{220}, \dtau{220})$ for GW150914, the joint likelihood and the
hierarchically combined methods.}

\begin{table}
\begin{flushleft}
\begin{tabular}{llllllll}
\toprule
Event & \multicolumn{2}{c}{Redshifted} & \hphantom{X} & \multicolumn{2}{c}{Redshifted} \\
& \multicolumn{2}{c}{frequency [Hz]} & \hphantom{X} & \multicolumn{2}{c}{damping time [ms]} \\[0.075cm]
\hline
& IMR  & \pSEOB & \hphantom{X} & IMR  & \pSEOB \\
\hline

GW150914 &
$249^{+9}_{-7}$ &
$-$ &
\hphantom{X} &
$4.1^{+0.3}_{-0.2}$ &
$-$
\\[0.075cm]

GW170104 &
$286^{+16}_{-27}$ &
$-$ &
\hphantom{X} &
$3.5^{+0.4}_{-0.3}$ &
$-$
\\[0.075cm]

GW170729 &
$161^{+13}_{-14}$ &
$-$ &
\hphantom{X} &
$7.8^{+1.8}_{-1.5}$ &
$-$
\\[0.075cm]

GW190630$\_$185205 &
$-$ &
$-$ &
\hphantom{X} &
$-$ &
$-$
\\[0.075cm]

GW190828$\_$063405 &
$-$ &
$-$ &
\hphantom{X} &
$-$ &
$-$
\\[0.075cm]
\hline
\bottomrule
\end{tabular}
\caption{\textcolor{red}{NOT COMPLETE} \comment{AB: We would need to give also the final mass and spins. I would suggest that we list in the Table
also the results from the TGR GWTC-2 paper, obtained with our method, so that all the results are in one Table.}}
\label{tab:qnm_o1o2_results}
\end{flushleft}
\end{table}

%%%%%%%%%%%%%%%%%


%%%%%%%%%%%%%%%%%
\section{Discussion}
\label{sec:discussion}
\comment{AB: I am sorry, but I had no time to write this section. Please write a first draft. I have read in the last week several papers
about extracting information on the nature of dark objects and gravity in the strong-field regime combining LIGO/Virgo and also
EHT. I wanted to refer to them --- for example Maggio et al., Volkel et al., Franciolini et al., Cano et al.
It is also good if we cite papers that have predicted QNMs in gravity theories alternative to GR. We did it in the paper with
Richard and Vivien.}

%%%%%%%%%%%%%%%%%

%%%%%%%%%%%%%%%%%
\section*{Acknowledgements}
\label{sec:acknowledgements}
LIGO Clusters. \comment{AB: Our cluster!!!}
R.B. acknowledges financial support from the European Union's Horizon 2020 research and innovation programme under the Marie Sk\l odowska-Curie grant agreement No. 792862. This research was supported by the Amaldi Research Center funded by the MIUR program ``Dipartimento di Eccellenza'' (CUP:~B81I18001170001).

%%%%%%%%%%%%%%%%%


%%%%%%%%%%%%%%%%%
\appendix
\section{Study of systematics in ringdown measurements in real, non-Gaussian noise}\label{sec:noise_systematics}


Inferences of all parameters in this paper have been done under the
assumption that the noise in the detectors is stationary and
Gaussian. In other words, detector noise follows a normal distribution
with zero mean and a PSD, $S_n(f)$, that is not a function of time, at
least during the duration of the GW signal. This
allows us to write the Bayesian likelihood function in the form given
in Eqs.~(\ref{eq:likelihood}) and ~(\ref{eq:nwip}), and perform all the
parameter estimation that follows in the results sections. However,
LIGO-Virgo noise can often have features that deviate from
stationarity and Gaussianity. If such features are not taken into
account appropriately, final estimates of parameters can get
biased. Here we demonstrate one such case by injecting in real noise a
GW190521-like signal and showing how parameter estimates can be biased
when our description of detector noise is not complete.

%%%%%%%%%%%%%%%%%%%%%%%%%%%%%%%%%%%%%%%%%%%%%%%%%%%%%%%%%%%%%%%
%%%%%%%%%%%%%%%%%%%%%%%%%%%%%%%%%%%%%%%%%%%%%%%%%%%%%%%%%%%%%%%
\begin{figure}
\begin{center}
        \includegraphics[width=0.4\textwidth]{figures/S190521g_swinjs.png}
        \caption{90 \% credible level on the posterior probability distribution of the frequency and damping time of $(2,\pm 2)$ mode, $(\fgr{220}, \taugr{220})$ using synthetic \texttt{NRSur} signals with parameters similar to the GW event, GW190521, in Gaussian noise (grey dot-dashed lines) and real interferometric noise (green dot dashed lines). The GR prediction for the frequency and damping time is indicated by the black cross. While the Gaussian noise simulations are consistent with the prediction, at least 3 of the 5 real noise simulation are not. The black curve corresponds to the measurements of the real event GW190521 reported in Ref.~\cite{Abbott:2020jks}. All signals are recovered using the $\pSEOB$ model.}
        \label{fig:21g_systematics}
\end{center}
\end{figure}
%%%%%%%%%%%%%%%%%%%%%%%%%%%%%%%%%%%%%%%%%%%%%%%%%%%%%%%%%%%%%%%
%%%%%%%%%%%%%%%%%%%%%%%%%%%%%%%%%%%%%%%%%%%%%%%%%%%%%%%%%%%%%%%

We choose a spin, precessing NR-surrogate model \texttt{NRSur}~\footnote{This waveform model is 
called \texttt{NRSur7dq4} in LAL.} (valid up to mass ratio 4) to simulate the actual GW190521 signal
observed by the LIGO and Virgo detectors~\cite{Abbott:2020tfl} (see
Table I of Ref.~\cite{Abbott:2020tfl})). The choice of the
\texttt{NRSur} model is motivated by the fact that it is the most
accurate model in the parameter range described by GW190521, because 
it is built by directly interpolating NR waveforms. In
Fig.~\ref{fig:21g_systematics}, we indicate with a black cross what
the injected \texttt{NRSur} signal predicts for the QNM $(\ell=2,m=2)$
frequency and damping time. For comparison, we also show with a
black solid curve the results obtained when recovering the actual
signal GW190521 with the $\pSEOB$ model. As seen in the plot,
while the measurement of the frequency is consistent with the
prediction, we overestimate the damping time.

To understand such offset in the decay time, we proceed as follows.
The actual GW190521 event was observed at a GPS time, 1242442967.61
seconds (roughly 03:02:49 UTC, May 21, 2019). We select a time period
of about 2.5 hours around this GPS time, create synthetic signals 
with the \texttt{NRSur} model and inject them in different stretches
of the real detector noise around the time of the actual GW event. The
PSDs of GW detectors are expected to vary over longer durations of
time, and hence the 2.5 hour stretch of noise we consider can be
assumed to have noise-properties similar to the time of the actual
event. Then, we perform Bayesian analysis against those injections 
using the $\pSEOB$ model. The results are indicated by green curves in
Fig.~\ref{fig:21g_systematics}. As it can be seen from the figure, for
3 of the 5 noise realizations, corresponding to $t_0-1$ hour,
$t_0+0.5$ hours, and $t_0+1$ hour, we recover a damping time similar to
the one obtained when using the $\pSEOB$ model against the actual event GW190521 
(black curve), where $t_0$ is the GPS time of the actual event. For
the other two noise realizations, the $\pSEOB$ model estimates 
consistently the damping time, but has an off-set frequency, 
while the fifth noise realization is consistent with both predictions.  
This study suggests that a bias in the measurements of the damping time 
for the actual event GW190521 can be explained as due to an incomplete 
description of the noise at the time of the event.

The reader might question the judiciousness of using an aligned-spin
waveform model, like $\pSEOB$, to measure a signal like GW190521 which
appears to be precessing, especially because an incomplete
understanding of the underlying signal can also lead to biases in
measured quantities, as we have already demonstrated in
Sec.~\ref{ssec:ngr_signal}. In order to explore possible effects of
missing information about in-plane spins in the $\pSEOB$ model, we repeat the
above study of injecting synthetic signals using \texttt{NRSur}
and recovering using the $\pSEOB$ model, but this time, instead of
using real detector noise, we use Gaussian noise (i.e., realizations
of noise sampled from a predicted detector PSD). Since the properties
of the noise are completely understood in this case, any residual
measurement biases can be completely attributed to diffferences in the
waveform model. The 2D posterior distributions of the frequency and
damping time measured using these Gaussian-noise signals are shown by
the grey curves in Fig.~\ref{fig:21g_systematics}. We find the
measurements to be completely consistent with the predictions of the
frequency and damping time, thus concluding that a lack of in-plane
spins in the $\pSEOB$ model does not affect our measurements of the
QNM properties. The fact that the measurement of ringdown quantities
are robust against an incomplete description of the inspiral signal 
is a crucial property of our method.


\section{Correlation of the binary's total mass with the non-GR parameters}\label{sec:correlation}

As mentioned in the main text, for low-SNR events with negligible
higher-modes and for which only the post-merger is detectable, there
is a strong degeneracy between the binary's total mass and the non-GR
deviations $(\df{220}, \dtau{220})$. For those cases, only the
reconstructed frequency and damping time $(f_{220},\tau _{220})$ can
be independently measured from the data.

To justify this statement, in Fig.~\ref{fig:correlations} we show
corner plots that illustrate the correlations between the non-GR
parameters $(\df{220}, \dtau{220})$, the detector-frame total mass
$M(1+z)$ and the reconstructed frequency and damping time
$(f_{220},\tau _{220})$ for GW150914 (left panel), corresponding to an
event for which both the pre- and post-merger phase are measurable,
and for GW190521 (right panel), an event in which the post-merger has
${\rm SNR}> 8$, but the pre-merger has an SNR below 8. For GW190521,
due the strong degeneracy between $\df{220}$ and $M(1+z)$, the 1D
posterior for $M(1+z)$ is pushed towards the upper boundary of its
prior, despite the very wide prior employed in the analysis. This in
turn renders the measurement of $\df{220}$ (and to a lesser degree of
$\dtau{220}$) highly dependent on the upper boundary of the total mass
prior. On the other hand, this issue does not significantly affect the
posteriors for the reconstructed quantities $(f_{220},\tau _{220})$
which are well measured and nearly independent on the upper prior
boundary for $M(1+z)$.
%
This is to be contrasted with the results for GW150914. In this case, 
the extra information coming from the pre-merger phase allows to break 
the degeneracy between the non-GR parameters and the total mass, 
and therefore both $(\df{220}, \dtau{220})$ and $M(1+z)$ can 
be measured at the same time. 

%%%%%%%%%%%%%%%%%%%%%%%%%%%%%%%%%%%%%%%%%%%%%%%%%%%%%%%%%%%%%%%
%%%%%%%%%%%%%%%%%%%%%%%%%%%%%%%%%%%%%%%%%%%%%%%%%%%%%%%%%%%%%%%
\begin{figure*}
\begin{center}
        \includegraphics[width=0.45\textwidth]{figures/mtotal_qnm_params_degeneracy_GW150914.pdf}\includegraphics[width=0.45\textwidth]{figures/mtotal_qnm_params_degeneracy_S190521g.pdf}
        \caption{Corner plots showing the correlations between the detector-frame total $(1+z)M$, the non-GR deviations $(\df{220}, \dtau{220})$ and the reconstructed frequency and damping time $(f_{220},\tau _{220})$. The left panel shows results for GW150914, for which ${\rm SNR}> 8$ in both the pre- and
post-merger phase of the signal, and the correlations are absent. The right panel shows the results for GW190521,  which has ${\rm SNR}> 8$ in the post-merger phase, but not in the pre-merger phase, and the correlations are present.}
        \label{fig:correlations}
\end{center}
\end{figure*}
%%%%%%%%%%%%%%%%%%%%%%%%%%%%%%%%%%%%%%%%%%%%%%%%%%%%%%%%%%%%%%%
%%%%%%%%%%%%%%%%%%%%%%%%%%%%%%%%%%%%%%%%%%%%%%%%%%%%%%%%%%%%%%%


%
\bibliographystyle{apsrev}
\bibliography{intro_paper}

\end{document}
