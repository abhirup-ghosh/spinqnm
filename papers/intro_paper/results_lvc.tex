\comment{AB: This is the most important section in the paper, because it contains the
main results of the paper, i.e., the constraints on QNMs using real data. It needs to
be expanded and results would need to be highlighted much better than in the current version.
I would like to suggest that besides the Table and the figure with the 2D- and 1D-posteriors
that you made (which can be contrasted with Fig. 14 of the TGR GWTC-2 paper), we could have
a new figure that is either a combination of Figs. 5 and 6 or only Fig. 5 of the
TGR GWTC-2, but for the 22 QNM frequencies for the different GW events.
Figures can be used in talks, and can capture the information much better
than a list of numbers in a Table.}

The LIGO-Virgo Collaboration recently released their testing GR
catalogue containing results for events observed
during O3a~\cite{Abbott:2020jks}. For the test that we here present, the results shown in~\cite{Abbott:2020jks} only include events which pass a threshold for the median redshifted total mass $\geq 90 \Mo$ and SNRs in the pre- and
post-merger regions $\geq 8$\footnote{The pre- and post-merger regions of the signal are identified from the signal's power before and after the signal reaches the peak's amplitude, which is determined from the maximum of the likelihood function obtained with the parameter-estimation analysis. The SNR values are listed in Table 4 of \cite{Abbott:2020jks}}. 
%
The SNR threshold ensures that the signal contains sufficient information in both the inspiral and merger stages to break the degeneracy between the binary's total mass and the non-GR deviations $(\df{220}, \dtau{220})$. Such strong degeneracy is present for low-SNR events with negligible higher-modes and for which only the post-merger is detectable rendering the measurement of $(\df{220}, \dtau{220})$ impossible for those cases.\rb{should we show an example where this degeneracy happens in the appendix? We can show a corner plot with $(\df{220}, \dtau{220}, M_{\rm total })$ for 21g in the appendix for example. And perhaps also for GW150914 for comparison.}\abhi{I would be fine with this. Perhaps we should have a discussion about how best to put it in the appendix, if we decide to.}
On the other hand the total mass threshold $\geq 90 \Mo$ was employed due to the fact that this analysis is computationally expensive, and also because we expected these events to be the most promising for ringdown studies. However, since the SNR threshold alone should be sufficient for the analysis, for this paper we run the test on all the events listed in Ref.~\cite{Abbott:2020jks} that have SNRs in the pre- and post-merger regions $\geq 8$, without imposing any mass threshold.

Imposing this threshold adds two events, GW190630$\_$185205 and GW190828$\_$063405, to the list of GW events considered in \cite{Abbott:2020jks}. Furthermore, for the first time, in this paper, we apply our method to
measure the QNMs to the relevant GW events from LIGO-Virgo's O1 and O2 runs, alongside
the above events. Applying the same threshold as above, we find two
additional events that could be included in the analysis: GW150914,
GW170104. The other high-mass events from O1-O2, GW170729, GW170809,
GW170814, GW170818 and GW170823 do not have an SNR $\geq 8$ in the
merger-ringdown signal. The list of the signals for which we run the analysis is given in Table~\ref{tab:qnm_o1o2_results}\footnote{See Table IV of Ref.~\cite{Abbott:2020jks} for a list of the SNR
  thresholds. The paper quotes them for the purpose of the IMR
  consistency test, but the same thresholds have been used for the
  $\pSEOB$ test, as well.}.
  
%%%%%%%%%%%%%%%%%%%%%%%%%%%%%%%%%%%%%%%%%%%%%%%%%%%%%%%%%%%%%%%
% O1-O2 events
%%%%%%%%%%%%%%%%%%%%%%%%%%%%%%%%%%%%%%%%%%%%%%%%%%%%%%%%%%%%%%%
\begin{figure*}
        \includegraphics[width=0.5\textwidth]{figures/rin_pseob_results_v2.pdf}\includegraphics[width=0.3\textwidth]{figures/rin_all_events_bounds.pdf}
        \caption{\emph{Left panel}:The 90\% credible levels of the posterior probability distribution of the fractional deviations in the frequency and damping time of the $(2,\pm 2)$ QNM, $(\df{220},\dtau{220})$ and their corresponding one-dimensional marginalized posterior distributions, for events from O1, O2 and O3a passing a SNR threshold of $8$ in both the pre- and post-merger signal. The solid red curve marks the best single-event constraint, GW150914, whereas the contraints from the other events are indicated by the dash-dot curves. The joint constraints on $(\df{220},\dtau{220})$ obtained multiplying the likelihoods from individual events is given by the filled grey contours, while the hierarchical method of combination yields the black dot dashed curves. \emph{Right panel}: 90\% credible interval on the one-dimensional marginalised posteriors on $\delta \sigma_i=(\df{220},\dtau{220})$, colored by the median redshifted total mass $(1 + z)M$, inferred assuming GR. Filled gray (unfilled black) triangles mark the constraints obtained when all the events are combined by multiplying likelihoods (hierarchically).}
        \label{fig:o1o2_events}
\end{figure*}
%%%%%%%%%%%%%%%%%%%%%%%%%%%%%%%%%%%%%%%%%%%%%%%%%%%%%%%%%%%%%%%
%%%%%%%%%%%%%%%%%%%%%%%%%%%%%%%%%%%%%%%%%%%%%%%%%%%%%%%%%%%%%%%


For all the relevant signals, we show the posterior distributions $(\df{220}, \dtau{220})$ in the left panel of Fig.~\ref{fig:o1o2_events} and also provide the reconstructed QNM parameters, $(\fngr{220}, \taungr{220})$ in Table~\ref{tab:qnm_o1o2_results}. In the right panel  of Fig.~\ref{fig:o1o2_events} we also provide a summary of the 90\% credible intervals on the one-dimensional marginalised posteriors highlighting the dependence of the constraints with respect to the total mass of the system. In general the tightest bounds are expected to come from the most massive systems, as they tend to have larger post-merger SNR. We find a similar trend in the right panel of Fig.~\ref{fig:o1o2_events}.
 
Among all the GW signals detected so far, GW150914 (solid curve in Fig.~\ref{fig:o1o2_events}) is unique in its loudness, leading to the first, and to date, best attempt in measuring the QNM frequencies~\cite{TheLIGOScientific:2016src,Brito:2018rfr,Carullo:2019flw,Isi:2019aib}. Within 90\% credibility we obtain from GW150914: 
%
\begin{equation}\label{GW150914_delta}
\df{220}=0.04^{+0.06}_{-0.04}\,\quad \dtau{220}=0.09^{+0.18}_{-0.18}\,.
\end{equation}

Stronger constraints can be obtained by combining information from all the events~\cite{Abbott:2020jks}. If we assume that the fractional deviations $(\df{220},\dtau{220})$ take the same values in multiple events, we can assume
the posterior of one event to be the prior for the next, and obtain a
joint posterior probability distribution. For $N$ observations, where
$P_j(\df{220}, \dtau{220} | d_j)$ is the posterior for the $j$-th
observation corresponding to the data set $d_j$, $j=1,\dots,N$, the joint
posterior is given by:
%
\begin{equation}
P(\df{220}, \dtau{220} | \{d_j\}) = P(\df{220}, \dtau{220}) \prod _{j=1}^N \frac{P(\df{220}, \dtau{220} | d_j) }{P(\df{220}, \dtau{220})}
\end{equation}
%
where $P(\df{220}, \dtau{220})$ is the prior on $(\df{220},
\dtau{220})$. However, since here we assume the prior on $(\df{220},
\dtau{220})$ to be flat (or uniform), the joint posterior is equal to
the joint likelihood. We show these joint likelihoods on $(\df{220}, \dtau{220})$, as well as, the corresponding 1D marginalized distributions as filled grey curves in Fig.~\ref{fig:o1o2_events}. From the joint likelihood we obtain within 90\% credibility: 
%
\begin{equation}
\df{220}=0.02^{+0.03}_{-0.03}\,,\quad \dtau{220}=0.11^{+0.12}_{-0.12}\,.
\end{equation}
%


As described in~\cite{Abbott:2020jks}, we can relax the assumption of a shared deviation across all events by using the hierarchical inference technique originally proposed in Refs.~\cite{Zimmerman:2019wzo,Isi:2019asy}. The general idea behind this technique is to assume that the non-GR parameters $(\df{220},\dtau{220})$ are drawn from a common underlying distribution, whose properties can be inferred from the population of events. Following~\cite{Zimmerman:2019wzo,Isi:2019asy,Abbott:2020jks} we model the population distribution with a Gaussian  $\mathcal{N}(\mu,\sigma)$ of unknown mean $\mu$ and standard deviation $\sigma$. We can then obtain a posterior distribution $P(\mu, \sigma |  \{d_j\}) $ for the hyperparameters $\mu$ and $\sigma$ from a joint analysis of all the GW events~\cite{Isi:2019asy}. We use the \texttt{stan}-based code~\cite{stan} developed and used in Refs.~\cite{Isi:2019asy,Abbott:2020jks} to obtain $P(\mu, \sigma |  \{d_j\}) $.
%
Finally, the distribution for a given non-GR parameter $\xi_{\text{nGR}}$ is then be obtained by computing~\cite{Isi:2019asy}
%
\begin{equation}
P(\xi_{\text{nGR}} | \{d_j\}) = \int P(\xi_{\text{nGR}} |\, \mu, \sigma)\,P(\mu, \sigma |  \{d_j\})\,d\mu\,d\sigma \,,
\end{equation}
%
where $P(\xi_{\text{nGR}} | \,\mu, \sigma) = \mathcal{N}(\mu,\sigma)$. The 1D posteriors for $\df{220}$ and $ \dtau{220}$ obtained through this technique are shown in Fig.~\ref{fig:o1o2_events} (dash-dotted curves) with corresponding median and 90\% credible interval given by: 
%
\begin{equation}
\df{220}=0.02^{+0.09}_{-0.09}\,,\quad \dtau{220}=0.13^{+0.42}_{-0.40}\,.
\end{equation}
Compared to Ref.~\cite{Abbott:2020jks} these constraints are almost a factor $\sim 4$ more stringent for $\df{220}$ and a factor $\sim 2$ for $\dtau{220}$. Similar improvements hold for the hyperparameters: $\df{220}\,(\mu=0.02^{+0.05}_{-0.05},\,\sigma<0.09)$ and $\dtau{220}\,(\mu=0.13^{+0.20}_{-0.18},\,\sigma<0.38)$.


%%%%%%%%%%%%%%%%%%%%%%%%%%%%%%%%%%%%%%%%%%%%%%%%%%%%%%%%%%%%%%%
% Table for LVC values
%%%%%%%%%%%%%%%%%%%%%%%%%%%%%%%%%%%%%%%%%%%%%%%%%%%%%%%%%%%%%%%
\begin{table}
\begin{table}
\begin{flushleft}
\begin{tabular}{llllllll}
\toprule
Event & \multicolumn{2}{c}{Redshifted} & \hphantom{X} & \multicolumn{2}{c}{Redshifted} \\
& \multicolumn{2}{c}{frequency [Hz]} & \hphantom{X} & \multicolumn{2}{c}{damping time [ms]} \\[0.075cm]
\hline
& IMR  & \pSEOB & \hphantom{X} & IMR  & \pSEOB \\
\hline

GW150914 &
$249^{+9}_{-7}$ &
$-$ &
\hphantom{X} &
$4.1^{+0.3}_{-0.2}$ &
$-$
\\[0.075cm]

GW170104 &
$286^{+16}_{i-27}$ &
$-$ &
\hphantom{X} &
$3.5^{+0.4}_{-0.3}$ &
$-$
\\[0.075cm]

GW170729 &
$161^{+13}_{-14}$ &
$-$ &
\hphantom{X} &
$7.8^{+1.8}_{-1.5}$ &
$-$
\\[0.075cm]

GW190630$\_$185205 &
$-$ &
$-$ &
\hphantom{X} &
$-$ &
$-$
\\[0.075cm]

GW190828$\_$063405 &
$-$ &
$-$ &
\hphantom{X} &
$-$ &
$-$
\\[0.075cm]
\hline
\bottomrule
\end{tabular}
\caption{\textcolor{red}{NOT COMPLETE} \comment{AB: We would need to give also the final mass and spins. I would suggest that we list in the Table
also the results from the TGR GWTC-2 paper, obtained with our method, so that all the results are in one Table.}}
\label{tab:qnm_o1o2_results}
\end{flushleft}
\end{table}

\caption{The median, and symmetric 90\% credible intervals of the frequency and damping time of the $(2,\pm2)$ QNM, and the mass and spin of the remnant object estimated from the measurements of the QNM frequencies.}
\label{tab:qnm_o1o2_results}
\end{table}

%%%%%%%%%%%%%%%%%%%%%%%%%%%%%%%%%%%%%%%%%%%%%%%%%%%%%%%%%%%%%%%
%%%%%%%%%%%%%%%%%%%%%%%%%%%%%%%%%%%%%%%%%%%%%%%%%%%%%%%%%%%%%%%
