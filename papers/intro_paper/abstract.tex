  The no-hair conjecture in General Relativity (GR) states that the
  properties of an astrophysical Kerr black hole (BH) are completely described by its
  mass and spin angular momentum. As a consequence, the complex
  quasi-normal-mode (QNM) frequencies of a binary--black-hole (BBH)
  ringdown can be uniquely determined by the mass and spin of the
  remnant object. Conversely, measurement of the QNM frequencies could
  be an independent test of the no-hair conjecture. This paper extends to spinning BHs earlier work that proposed to
  test the no-hair conjecture by measuring the complex QNM
  frequencies of a BBH ringdown using parameterized inspiral-merger-ringdown waveforms in the effective-one-body formalism, 
thereby taking full advantage of the entire signal power and removing dependency on the
  predicted or estimated start time of the ringdown. Our method was used to analyse the
  properties of the merger remnants for four BBHs observed by
  LIGO-Virgo in the first half of their third observing (O3a) run in
  the latest LIGO-Virgo publication. After testing our method with GR and non-GR synthetic-signal injections in Gaussian noise, 
we analyse for the first time four other GW events: two BBHs from the first (O1) and second (O2) LIGO-Virgo 
  observing runs, and two BBHs from the O3a run. We then provide joint constraints with published results
  from the O3a run. In the most agnostic and conservative scenario where we combine the information from different
events using a hierarchical approach, we obtain, at $90\%$ credibility, that 
the fractional deviations in the frequency (damping time) of the dominant QNM are 
$\df{220}=0.02^{+0.09}_{-0.09}$ ($\dtau{220}=0.13^{+0.42}_{-0.40}$), respectively, an improvement of a factor of $\sim 4$ ($\sim 2$) 
over the results obtained with our model in the LIGO-Virgo publication. The single-event most-stringent constraint to date continues to be 
GW150914 for which we obtain $\df{220}=0.04^{+0.06}_{-0.04}$ and $\dtau{220}=0.09^{+0.18}_{-0.18}$.