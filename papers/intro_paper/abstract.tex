  The no-hair conjecture in General Relativity states that the
  properties of an astrophysical Kerr black hole (BH) are completely described by its
  mass and spin angular momentum. As a consequence, the complex
  quasi-normal-mode (QNM) frequencies of a binary black hole (BBH)
  ringdown can be uniquely determined by the mass and spin of the
  remnant object. Conversely, measurement of the QNM frequencies could
  be an independent test of the no-hair conjecture. This paper extends to spinning BHs earlier work that proposed to
  test the no-hair conjecture by measuring the complex QNM
  frequencies of a BBH ringdown using inspiral-merger-ringdown waveforms, thereby taking full advantage of the entire signal power and removing dependency on the
  predicted or estimated start time of the proposed ringdown. We
  further demonstrate the robustness of the test against modified
  gravitational-wave (GW) signals with a ringdown different from what
  GR predicts for Kerr BHs. Our method was used to analyse the
  properties of the merger remnants for the events observed by
  LIGO-Virgo in the first half of their third observing run (O3a) in
  the latest LIGO-Virgo publication. In this paper, for the first
  time, we analyse the GW events from the first (O1) and second (O2) LIGO-Virgo
  observing runs and provide joint constraints with published results
  from O3a. We also analyse two events from the O3a catalogue
  that were not considered in the initial LIGO-Virgo analysis. The
  joint measurements of the fractional deviations in the frequency and damping time,
  $\df{220} = XX \pm$ and $\dtau{220} = YY \pm$ are the strongest
  constraints yet using this method. %Finally, we also present a investigation into possible systematic effects due to an incomplete understanding of the interferometric noise around the GW event on the results of this test.
