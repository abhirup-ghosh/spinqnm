\begin{table}[h!]
\begin{center}
\begin{tabular}{ c|c|c|c|c|c|c|c|c }

Injection &  Network & \makecell{$m_{\rm 1\,det}$ \\$(\Mo)$} &  \makecell{$m_{\rm 2\,det}$ \\ $(\Mo)$} & $\chi_{1}$ & $\chi_{2}$ & $\rho_\text{IMR}$ & $\rho_\text{insp}$ & $\rho_\text{postinsp}$ \\
 \hline
 GW150914-like & HL &39 & 31 & 0.0 & 0.0 & 25 & 22 & 12 \\
 GW190521-like & HL & 150 & 120 & 0.02 & -0.39 & 20 & 8 & 18 \\
 SXS:BBH:0166 & HLV &72 & 12  & 0.0 & 0.0 & 71 & 58 & 41 \\

\end{tabular}
\caption{Parameters of the synthetic-signal injections, chosen to be similar to the actual GW events indicated in the first column (first two rows). The parameters $(m_{\rm 1 \,det}, m_{\rm 2 \,det})$ are the detector-frame masses of the primary and secondary BHs, respectively. The third row indicates the parameters of the SXS BBH waveform used in Sec.~\ref{ssec:nohairtheorem}. The second column refers to the detector-network used, with H,L,V, referring to LIGO-Hanford, LIGO-Livingston and Virgo, respectively. The quantities $\rho_\text{IMR}$, $\rho_\text{insp}$ and $\rho_\text{postinsp}$ are the SNR of the full IMR signal, SNR upto a certain cutoff frequency, and SNR after the cutoff frequency respectively. In each case, the cutoff frequency is assumed to be the frequency at the innermost circular stable orbit (ISCO) corresponding to the remnant Kerr BH.}
\label{tab:injection_values}
\end{center}
\end{table}