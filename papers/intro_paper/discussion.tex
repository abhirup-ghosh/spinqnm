We have built a parameterized IMR waveform model, $\pSEOB$, that can 
measure the QNM complex frequencies of the remnant object formed through 
the merger of BHs with aligned or antialigned spins, thus extending  
previous work, which was limited to non-spinning BHs~\cite{Brito:2018rfr}. 
The $\pSEOB$ model was recently used to infer the QNMs of some of the 
BBH's remnants detected by LIGO and Virgo during O3a~\cite{Abbott:2020jks}. 

After testing our method to infer the QNM frequencies with GR and non-GR synthetic-signal 
injections in Gaussian noise, we have applied it to LIGO and Virgo real data. 
We have analyzed GW events in O1, O2 and O3a (a total of 4 new events) that were not examined in Ref.~\cite{Abbott:2020jks} 
with the $\pSEOB$ model (see Table~\ref{tab:qnm_o1o2_results}). After combining our new 
results with the other GW events in O3a investigated with our method~\cite{Abbott:2020jks}, 
we have obtained more stringent bounds on the dominant (or least-damped) QNM $(\ell=2,m=2)$.

More specifically, as expected, the single GW event providing the best constraint
on the least-damped QNM to date continues to be GW150914 $(\df{220}=0.04^{+0.06}_{-0.04}, 
\dtau{220}=0.09^{+0.18}_{-0.18})$. In addition, in the most agnostic and
conservative scenario where we combine the information from different
events using a hierarchical approach~\cite{Zimmerman:2019wzo,Isi:2019asy}, we obtain 
at $90\%$ credibility $(\df{220}=0.02^{+0.09}_{-0.09}, \dtau{220}=0.13^{+0.42}_{-0.40})$. Thus, 
our results constrain the frequency (decay time) of the least damped QNM to be within $\sim
9\%$ ($\sim 40\%$) of the GR prediction --- an improvement of a factor of $\sim 4$ ($\sim 2$) 
over the results obtained with the $\pSEOB$ model in Ref.~\cite{Abbott:2020jks}.

\ab{Furthermore, when assuming that the deviations from GR do not vary appreciably over the GW events 
that we have analyzed, and combine the likelihood functions, we have obtained the most stringent bounds  
on the dominant QNM $(\df{220}=0.02^{+0.03}_{-0.03}, \dtau{220}=0.11^{+0.12}_{-0.12})$. Those 
constraints are compatible, but sligthly better (especially for the decay time) than the ones recently reported in Ref.~\cite{Carullo:2021dui} 
(see first row in Table II therein), where QNM frequencies were inferred using only the post-merger part of 
the signal. We note that Ref.~\cite{Carullo:2021dui} used a larger set of GW events from O1, O2 and O3a than we did, although we expect 
that the extra GW events will not contribute significantly to the combined bound since they have lower SNRs.}

These constraints could in principle be used to constrain specific
non-GR theories and exotic compact objects~\cite{Glampedakis:2017cgd,Cardoso:2019rvt,Maggio:2020jml}. 
However, QNM computations in non-GR theories have only been done in an
handful of cases, mostly focusing on non-rotating or slowly-rotating
BH solutions~\cite{Ferrari:2000ep,Molina:2010fb,Pani:2009wy,Blazquez-Salcedo:2016enn,Blazquez-Salcedo:2017txk,Brito:2018hjh,Franciolini:2018uyq,Cardoso:2018ptl,Tattersall:2018nve,Tattersall:2019nmh,Blazquez-Salcedo:2019nwd,Silva:2019scu,Glampedakis:2019dqh,Blazquez-Salcedo:2020jee,Blazquez-Salcedo:2020caw,Cano:2020cao,Wagle:2021tam,Pierini:2021jxd}, 
or relying on the eikonal/geometric optics approximation to obtain
estimates of the QNMs for spinning BHs~\cite{Blazquez-Salcedo:2016enn,Glampedakis:2017dvb,Jai-akson:2017ldo}. The
only exceptions to this rule, that we are aware of, are computations
of the QNM of Kerr-Newman BHs in Einstein-Maxwell
theory~\cite{Pani:2013ija,Pani:2013wsa,Mark:2014aja,Dias:2015wqa} or estimates of the BBH in non-GR theories obtained
through a limited number of NR simulations~\cite{Okounkova:2019dfo,Okounkova:2019zjf}. Given these
limitations, our ability of going beyond a null test of GR and use our
results to impose precise constraints on non-GR theories with QNM
measurements is currently quite limited.

Despite these theoretical limitations there has been some recent effort to develop
parametrizations that could help mapping measurements of the
parameters $(\delta f_{\ell m 0}, \delta \tau_{\ell m0})$ onto
constraints to specific non-GR theories. This includes for example
the parametrization proposed in Ref.~\cite{Maselli:2019mjd}, recently
applied to LIGO-Virgo GW events in Ref.~\cite{Carullo:2021dui}, where
deviations from the GR QNMs explicitly depend on a perturbative
expansion in the BH spin and possible extra non-GR parameters, or the
proposal of Refs.~\cite{Cardoso:2019mqo,McManus:2019ulj} where
deviations from the GR QNMs are mapped onto generic small
modifications of the perturbation equations describing the QNMs. Other
examples also include proposals to map deviations from the GR QNMs to
coefficients in generic effective--field-theory
actions~\cite{Cardoso:2018ptl,Franciolini:2018uyq,Cano:2020cao} or to
directly relate measurements of the QNM complex frequencies to a
parametrized non-GR BH metric~\cite{Glampedakis:2017dvb,Suvorov:2021amy,Volkel:2020daa}, 
which could be then used jointly with measurements from the Event
Horizon Telescope to obtain stronger constraints on deviations from
GR~\cite{Volkel:2020daa,Volkel:2020xlc,Psaltis:2020lvx,Yang:2021zqy}.
We should note, however, that all these parametrizations are either
limited to non-spinning BHs or make use of a series expansion in the
BH spin which might limit their accuracy for highly-spinning BHs\ab{, unless 
the sensitivity of GW detectors will not allow us to access the higher 
coefficients in the spin series.}

In the future, it would be important to test whether the $\pSEOB$  model
could be used to detect deviations in waveforms obtained through NR 
simulations of specific non-GR theories. Such results are still at their 
infancy and have so far only been done for a handful of theories, focusing mostly on proof-of-concept
simulations~\cite{Healy:2011ef,Berti:2013gfa,Cao:2013osa,Okounkova:2017yby,Hirschmann:2017psw,Witek:2018dmd,Okounkova:2019dfo,Okounkova:2019zjf,Okounkova:2020rqw,East:2020hgw}. Nonetheless,
given the recent efforts put forward in order to simulate BBHs in
non-GR theories, we hope that accurate non-GR IMR waveforms will become available in the near future.

Finally, an obvious generalization of this work would be to extend the
parameterized $\pSEOB$ model to generic precessing BBHs, which can in
principle be easily done using the recently developed multipolar EOB
model reported in Ref.~\cite{Ossokine:2020kjp}. Lastly, it will be relevant 
to include GR deviations in the $\pSEOB$ model also during the late inspiral 
and plunge stages of the BBH coalescence. On the other hand, GR deviations (notably 
deviations from PN theory) for the long inspiral stage are currently 
available in the $\pSEOB$ model and have been used to set bounds on the 
PN parameters in the GW phasing using LIGO and Virgo observations~\cite{Abbott:2018lct,
LIGOScientific:2019fpa,Abbott:2020jks}.
