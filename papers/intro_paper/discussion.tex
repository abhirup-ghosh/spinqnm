\rb{Work in progress}
We demonstrated how a parameterized IMR waveform model can be used to measure the QNM complex frequencies of the remnant object formed in BBH mergers. Our model can be used for BBHs with (aligned-)spinning components, which extends previous work where a similar model for non-spinning components had been considered~\cite{Brito:2018rfr}. We showed that this model can accurately recover the QNM complex frequencies of simulated GW signals in the LIGO and Virgo detectors. Our results also demonstrate that this model can be used to perform tests of the no-hair conjecture and can also be used to detect possible deviations from GR in the ringdown part of the signal.

This model had been previously used to analyze BBHs from the first half of LIGO-Virgo's third observing run~\cite{Abbott:2020jks}. We extended these results by analyzing GW events from the first and second LIGO-Virgo observing runs and updated the joint measurements on the fractional deviations in the frequency and damping time of the least-damped QNM. As expected, the event providing the best constraint on the least-damped QNM to be date continues to be GW150914 [see \eqref{GW150914_delta}]. In addition, even in the most agnostic and conservative scenario where we combine the information from different events using an hierarchical approach, at $90\%$ credibility, we constrain the frequency of the least damped QNM to be within $\sim 9\%$ of the GR prediction -- an improvement of a factor of $\sim 4$ over the results reported in~\cite{Abbott:2020jks}. The order-of-magnitude of the constraints we obtain is largely compatible with the ones recently reported in Ref.~\cite{Carullo:2021dui}, where measurements on QNMs were done using only the post-merger part of the ringdown was analyzed.

These constraints could in principle be used to constrain specific non-GR theories and exotic compact objects (see e.g.~\cite{Glampedakis:2017cgd,Cardoso:2019rvt,Maggio:2020jml}). However, QNM computations in beyond-GR theories have only been done in an handful of case, mostly focusing on non-rotating or slowly-rotating BH solutions~\cite{Ferrari:2000ep,Molina:2010fb,Pani:2009wy,Blazquez-Salcedo:2016enn,Blazquez-Salcedo:2017txk,Brito:2018hjh,Franciolini:2018uyq,Cardoso:2018ptl,Tattersall:2018nve,Tattersall:2019nmh,Blazquez-Salcedo:2019nwd,Silva:2019scu,Glampedakis:2019dqh,Blazquez-Salcedo:2020jee,Blazquez-Salcedo:2020caw,Cano:2020cao} or relied on the eikonal/geometric optics approximation to obtain estimates of the QNMs for spinning BHs~\cite{Blazquez-Salcedo:2016enn,Glampedakis:2017dvb,Jai-akson:2017ldo}. The only exceptions to this rule are computations of the QNM of Kerr-Newman BHs in Einstein-Maxwell theory~\cite{Pani:2013ija,Pani:2013wsa,Mark:2014aja,Dias:2015wqa} and estimates of the BBH ringdown in beyond-GR theories obtained through numerical simulations~\cite{Okounkova:2019dfo,Okounkova:2019zjf}.

To constrain the deviations from GR we decided to use the most generic and conservative parametrization, Eqs.~\eqref{eq:nongr_freqs_a} and~\eqref{eq:nongr_freqs_b}, where the fractional deviations $(\delta f_{\ell m0},\delta \tau_{\ell m0})$ are taken to be completely free parameters. In principle, this model can also be used with other, more restricted, parametrizations that could help to map constraints on the QNM complex frequencies to constraints in specific beyond-GR theories. This includes for example the parametrization proposed in Ref.~\cite{Maselli:2019mjd}, recently applied to LIGO-Virgo GW events in Ref. ~\cite{Carullo:2021dui}, where deviations from the GR QNMs explicitly depend on a perturbative expansion in the BH spin and possible extra non-GR parameters. Other examples also include proposals to write the deviations from GR in terms of a small parametrized deviations of the perturbation equations~\cite{Cardoso:2019mqo,McManus:2019ulj} or proposals to directly relate measurements of the QNM complex frequencies to non-GR parametrized BH metrics~\cite{Glampedakis:2017dvb,Suvorov:2021amy,Volkel:2020daa}. As proposed in~\cite{Volkel:2020daa,Volkel:2020xlc} constraints on the BH metric using QNM measurements could then also be used jointly with constraints coming from the Event Horizon Telescope~\cite{Volkel:2020xlc,Psaltis:2020lvx,Yang:2021zqy}.

It would also be important to test our model against waveforms obtained with numerical relativity simulations of specific beyond-GR theories. Such simulations are however extremely challenging to do (when at all possible), and have so far only been done for a handful of theories, focusing mostly on proof-of-concept simulations~\cite{Healy:2011ef,Berti:2013gfa,Cao:2013osa,Okounkova:2017yby,Hirschmann:2017psw,Witek:2018dmd,Okounkova:2019dfo,Okounkova:2019zjf,Okounkova:2020rqw,East:2020hgw}. Nonetheless, given the recent efforts put forward in order to simulate BBHs in beyond-GR theories, we hope that accurate non-GR IMR waveforms will become available in the near future.

Finally, an obvious generalization of this work would be to extend the parameterized IMR model to generic precessing BBHs, which can in principle be easily done using the recently developed multipolar EOB model reported in Ref.~\cite{Ossokine:2020kjp}.
