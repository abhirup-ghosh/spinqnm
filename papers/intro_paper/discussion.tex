\comment{AB: I am sorry, but I had no time to write this section. Please write a first draft. I have read in the last week several papers
about extracting information on the nature of dark objects and gravity in the strong-field regime combining LIGO/Virgo and also
EHT. I wanted to refer to them --- for example Maggio et al., Volkel et al., Franciolini et al., Cano et al.
It is also good if we cite papers that have predicted QNMs in gravity theories alternative to GR. We did it in the paper with
Richard and Vivien.}

\rb{Work in progress (most of this was taken from \paperone}
We investigated the advantages of using IMR waveforms, with respect to 
damped-sinusoid models, to measure ringdown frequencies and damping times in the post-merger 
signal of a compact-object coalescence. To address this goal, we built a parameterized multipolar IMR waveform model within 
the EOB formalism (pEOBNR), and investigated its ability in measuring the QNM complex frequencies in GW150914, and 
in several synthetic GW signals injected in Gaussian noise.
  
We found the following important advantages: (i) using an IMR model, 
calibrated to NR waveforms, one does not need to define an \emph{a priori} unknown starting time at
which the signal can be described as a sum of exponentially damped
sinusoids~\cite{Gossan:2011ha,Meidam:2014jpa,Cabero:2017avf,Bhagwat:2017tkm,Baibhav:2017jhs,London:2018gaq}, therefore avoiding potential biases due to a non-optimal
choice of the ringdown starting time~\cite{Thrane:2017lqn}; (ii) the IMR model avoids technical
issues inherent to assuming a waveform with a cutoff at a particular
time, namely the need to know in advance the sky position and time at
coalescence~\cite{TheLIGOScientific:2016src,Cabero:2017avf}; (iii) the IMR model naturally includes important physics,
such as phase shifts between different modes, their relative
amplitudes and the presence of overtones~\cite{Buonanno:2006ui,Pan:2011gk}; and (iv) the IMR model
generically leads to stronger constraints on the QNM frequencies
compared to what can be achieved with a damped-sinusoid model.

The approach that we here presented should also be seen as
complementary to previous works on the subject. Besides directly
measuring the ringdown frequencies, our IMR model 
can also be used to validate the results obtained with the more agnostic
damped-sinusoid models. In particular, as we showed, the pEOBNR model already
provides very interesting constraints on the frequency and damping time of the dominant QNM of GW150914~\cite{Abbott:2016blz}.

This work can be improved in several fronts... The IMR model here presented could also be
extended to allow GR deviations in the inspiral phase.

QNM frequencies of spherically symmetric solutions were computed in theories such as
Einstein-Maxwell-dilaton~\cite{Ferrari:2000ep}, dynamical
Chern-Simons gravity~\cite{Molina:2010fb}, Einstein-dilaton-Gauss-Bonnet
gravity~\cite{Pani:2009wy,Blazquez-Salcedo:2016enn,Blazquez-Salcedo:2017txk}
and for some solutions in massive
(bi)gravity~\cite{Brito:2013wya,Brito:2013yxa,Babichev:2015zub}. On
the other hand, not much progress has been made to compute QNMs for
spinning BHs in alternative theories to GR, the only exception
being the Kerr-Newman case in Einstein-Maxwell
theory~\cite{Pani:2013ija,Pani:2013wsa,Mark:2014aja,Dias:2015wqa}. Most
of the estimates for QNMs of spinning BHs in modified gravity have
instead used the connection between the light ring and
QNMs~\cite{Blazquez-Salcedo:2016enn,Glampedakis:2017dvb,Jai-akson:2017ldo,Glampedakis:2017cgd},
which is formally only valid in the eikonal $\ell \to \infty$ limit
and known to fail to describe some families of QNMs when additional
degrees of freedom are present~\cite{Blazquez-Salcedo:2016enn}. 