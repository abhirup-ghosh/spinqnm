\comment{AB: I am sorry, but I had no time to write this section. Please write a first draft. I have read in the last week several papers
about extracting information on the nature of dark objects and gravity in the strong-field regime combining LIGO/Virgo and also
EHT. I wanted to refer to them --- for example Maggio et al., Volkel et al., Franciolini et al., Cano et al.
It is also good if we cite papers that have predicted QNMs in gravity theories alternative to GR. We did it in the paper with
Richard and Vivien.}

\rb{Work in progress}
In this work, we used a parameterized IMR waveform model to measure the QNM complex frequencies of the remnant object formed in BBH mergers with (aligned-)spinning components, extending previous work where only non-spinning components had been considered~\cite{Brito:2018rfr}. We demonstrated that the model can accurately recover the QNM complex frequencies for a selection of simulated GW signals, including a case more than one mode is measurable, and also showed that this test can be used to detect possible deviations from GR in the ringdown part of the signal.

This model had been previously used to analyze BBHs from the GWTC-2~\cite{Abbott:2020jks}, and we here extended those results by also analyzing GW events from the first and second LIGO-Virgo observing runs. We not find any evidence for deviations from GR and even in the most agnostic and conservative scenario, where we combine the information using an hierarchical approach, we constrain the frequency of the least damped QNM to be within $\sim 9\%$ of the GR prediction, whereas the damping time is constrained to be within $\sim 40\%$ of the GR prediction. These results provide the most-up-to-date and strongest constraints on the frequency and damping time of least-damped QNM obtained with this model. The order-of-magnitude of the constraints we obtain is largely compatible with the ones recently reported in Ref.~\cite{Carullo:2021dui}, where only the post-merger part of the ringdown was analyzed.

These constraints could in principle be used to constrain specific non-GR theories and exotic compact objects.
\rb{this is taken from \paperone, needs to be updated:}QNM frequencies of spherically symmetric solutions were computed in theories such as Einstein-Maxwell-dilaton~\cite{Ferrari:2000ep}, dynamical Chern-Simons gravity~\cite{Molina:2010fb}, Einstein-dilaton-Gauss-Bonnet gravity~\cite{Pani:2009wy,Blazquez-Salcedo:2016enn,Blazquez-Salcedo:2017txk} and for some solutions in massive (bi)gravity~\cite{Brito:2013wya,Brito:2013yxa,Babichev:2015zub}. On
the other hand, not much progress has been made to compute QNMs for
spinning BHs in alternative theories to GR, the only exception
being the Kerr-Newman case in Einstein-Maxwell
theory~\cite{Pani:2013ija,Pani:2013wsa,Mark:2014aja,Dias:2015wqa}. Most
of the estimates for QNMs of spinning BHs in modified gravity have
instead used the connection between the light ring and
QNMs~\cite{Blazquez-Salcedo:2016enn,Glampedakis:2017dvb,Jai-akson:2017ldo,Glampedakis:2017cgd},
which is formally only valid in the eikonal $\ell \to \infty$ limit
and known to fail to describe some families of QNMs when additional
degrees of freedom are present~\cite{Blazquez-Salcedo:2016enn}. 

To constrain the deviations from GR we decided to use the most generic and conservative parametrization, Eqs.~\eqref{eq:nongr_freqs_a} and~\eqref{eq:nongr_freqs_b}, where the fractional deviations $(\delta f_{\ell m0,\delta \tau_{\ell m0}$ are taken to be completely free parameters. In principle, this model can also be used with more restricted parametrizations, such as the one proposed in Ref.~\cite{Maselli:2019mjd} and recently applied on LIGO-Virgo events in Ref. ~\cite{Carullo:2021dui}, where deviations from the GR QNMs explicitly depend on a perturbative expansion in the BH spin and possible extra non-GR parameters. Other examples also include proposals to directly relate measurements of the QNM complex frequencies to non-GR parametrized BH metrics~\cite{Suvorov:2021amy,Volkel:2020daa} which, as proposed in ~\cite{Volkel:2020daa,Volkel:2020xlc} could be used jointly with constraints coming from the Event Horizon Telescope~\cite{Volkel:2020xlc,Psaltis:2020lvx}.

This work can be improved in several fronts... It could be easily extended to generic precessing waveforms. Interesting future studies include using this model using waveforms obtained in recent numerical relativity simulations of non-GR theories.