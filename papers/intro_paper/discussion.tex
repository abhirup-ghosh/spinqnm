We built a parameterized IMR waveform model that can be used to measure the QNM complex frequencies of the remnant object formed in BBH mergers. This model can be used to analyze any BBH with (aligned-)spinning components, which constitute an important extension of previous work where a similar model for non-spinning components had been considered~\cite{Brito:2018rfr}. We showed that this test can be used to accurately recover the QNM complex frequencies of simulated GW signals in the LIGO and Virgo detectors. We also demonstrated that this model can be used to perform tests of the no-hair conjecture and can also be used to detect possible small deviations from GR in the ringdown part of the signal.

This model had been previously used to analyze BBHs from the first half of LIGO-Virgo's third observing run~\cite{Abbott:2020jks}. We extended these results by analyzing GW events from the first and second LIGO-Virgo observing runs and updated the joint measurements on the fractional deviations in the frequency and damping time of the least-damped QNM. As expected, the event providing the best constraint on the least-damped QNM to be date continues to be GW150914 [see \eqref{GW150914_delta}]. In addition, even in the most agnostic and conservative scenario where we combine the information from different events using an hierarchical approach, at $90\%$ credibility, we constrain the frequency of the least damped QNM to be within $\sim 9\%$ of the GR prediction -- an improvement of a factor of $\sim 4$ over the results reported in~\cite{Abbott:2020jks}. The constraints we obtain on the frequencies and damping time of the least damped QNM are compatible with the ones recently reported in Ref.~\cite{Carullo:2021dui}, where measurements on QNMs were done using only the post-merger part of the signal was analyzed.

These constraints could in principle be used to constrain specific non-GR theories and exotic compact objects (see e.g.~\cite{Glampedakis:2017cgd,Cardoso:2019rvt,Maggio:2020jml}). However, QNM computations in beyond-GR theories have only been done in an handful of cases, mostly focusing on non-rotating or slowly-rotating BH solutions (see e.g.~\cite{Ferrari:2000ep,Molina:2010fb,Pani:2009wy,Blazquez-Salcedo:2016enn,Blazquez-Salcedo:2017txk,Brito:2018hjh,Franciolini:2018uyq,Cardoso:2018ptl,Tattersall:2018nve,Tattersall:2019nmh,Blazquez-Salcedo:2019nwd,Silva:2019scu,Glampedakis:2019dqh,Blazquez-Salcedo:2020jee,Blazquez-Salcedo:2020caw,Cano:2020cao}) or relying on the eikonal/geometric optics approximation to obtain estimates of the QNMs for spinning BHs~\cite{Blazquez-Salcedo:2016enn,Glampedakis:2017dvb,Jai-akson:2017ldo}. The only exceptions to this rule, that we are aware of, are computations of the QNM of Kerr-Newman BHs in Einstein-Maxwell theory~\cite{Pani:2013ija,Pani:2013wsa,Mark:2014aja,Dias:2015wqa} and estimates of the BBH ringdown in beyond-GR theories obtained through a limited number of numerical simulations~\cite{Okounkova:2019dfo,Okounkova:2019zjf}. Given these limitations, our ability of going beyond a  null test of GR and use our results to impose precise constraints on beyond-GR theories with QNM measurements is at this moment quite limited.

Despite these limitations there has been some recent effort to develop parametrizations that could help mapping measurements of the parameters $(\delta f_{\ell m 0}, \delta \tau_{\ell m0})$ onto constraints to specific beyond-GR theories. This includes for example the parametrization proposed in Ref.~\cite{Maselli:2019mjd}, recently applied to LIGO-Virgo GW events in Ref. ~\cite{Carullo:2021dui}, where deviations from the GR QNMs explicitly depend on a perturbative expansion in the BH spin and possible extra non-GR parameters, or the proposal of Refs.~\cite{Cardoso:2019mqo,McManus:2019ulj} where deviations from the GR QNMs are mapped onto generic small modifications of the perturbation equations describing the QNMs. Other examples also include proposals to map deviations from the GR QNMs to coefficients in generic effective field theory actions~\cite{Cardoso:2018ptl,Franciolini:2018uyq,Cano:2020cao} or to directly relate measurements of the QNM complex frequencies to a parametrized non-GR BH metric~\cite{Glampedakis:2017dvb,Suvorov:2021amy,Volkel:2020daa} which could be then used jointly with measurements coming from the Event Horizon Telescope to obtain stronger constraints on deviations from GR~\cite{Volkel:2020daa,Volkel:2020xlc,Psaltis:2020lvx,Yang:2021zqy}.
We should note, however, that all these parametrizations are either limited to non-spinning BHs or make use of a series expansion in the BH spin which might limit their accuracy for highly-spinning BHs. 

In the future, it would also be important to test whether this model could be used to detect deviations in waveforms obtained through numerical relativity simulations of specific beyond-GR theories. Such simulations are still at their infancy and have so far only been done for a handful of theories, focusing mostly on proof-of-concept simulations~\cite{Healy:2011ef,Berti:2013gfa,Cao:2013osa,Okounkova:2017yby,Hirschmann:2017psw,Witek:2018dmd,Okounkova:2019dfo,Okounkova:2019zjf,Okounkova:2020rqw,East:2020hgw}. Nonetheless, given the recent efforts put forward in order to simulate BBHs in beyond-GR theories, we hope that accurate non-GR IMR waveforms will become available in the near future.

Finally, an obvious generalization of this work would be to extend the parameterized IMR model to generic precessing BBHs, which can in principle be easily done using the recently developed multipolar EOB model reported in Ref.~\cite{Ossokine:2020kjp}.
