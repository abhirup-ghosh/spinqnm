\section{Study of systematics in ringdown measurements in real, non-Gaussian noise}\label{sec:noise_systematics}

\abhi{completely re-written this section to make it more readable and less technical.}

Inferences of all parameters in this paper have been done under the assumption that the noise in the detectors is stationary and Gaussian. In other words, detector noise follows a normal distrubution with zero mean and a PSD, $S_n(f)$, that is not a function of time, at least during the internal of the gravitational wave signal. This allows us to write the Bayesian likelihood function in the form given in Eqs.~\ref{eq:likelihood}-~\ref{eq:nwip}, and perform all the parameter estimation that follows in the results sections. However,, LIGO-Virgo noise can often have features that deviate from stationarity and Gaussianity. If such features are not taken into account appropriately, final estimates of parameters can get biased. We demonstrate one such case by injecting in real noise a GW190521-like signal and showing how parameter estimates can be biased by our understanding of detector noise is not complete.

%%%%%%%%%%%%%%%%%%%%%%%%%%%%%%%%%%%%%%%%%%%%%%%%%%%%%%%%%%%%%%%
%%%%%%%%%%%%%%%%%%%%%%%%%%%%%%%%%%%%%%%%%%%%%%%%%%%%%%%%%%%%%%%
\begin{figure}
\begin{center}
        \includegraphics[width=0.5\textwidth]{figures/S190521g_swinjs.png}
        \caption{\textcolor{red}{FINAL RESULT} 90 \% credible level on the posterior probability distribution of the frequency and damping time of $(2,\pm 2)$ mode, $(\fgr{220}, \taugr{220})$ using synthetic \texttt{NRSur7dq4} GW signals with parameters similar to the GW event, GW190521, in Gaussian noise (grey dot-dashed lines) and real interferometric noise (green dot dashed lines). The GR prediction for the frequency and damping time is indicated by the black cross. While the Gaussian noise simulations are consistent with the prediction, at least 3 or the 5 real noise simulation are not. The black curve corresponds to the measurements of the real event GW190521 reported in~\ref{Abbott:2020jks}. All signals are recovered using the $\pSEOB$ waveform model.}
        \label{fig:21g_systematics}
\end{center}
\end{figure}
%%%%%%%%%%%%%%%%%%%%%%%%%%%%%%%%%%%%%%%%%%%%%%%%%%%%%%%%%%%%%%%
%%%%%%%%%%%%%%%%%%%%%%%%%%%%%%%%%%%%%%%%%%%%%%%%%%%%%%%%%%%%%%%

We choose a spinning, precessing NR-surrogate model \texttt{NRSur7dq4} (valid up to mass ratio 4) to simulate the actual GW190521 signal observed by the LIGO and Virgo detectors~\cite{Abbott:2020tfl} (see Table I of Ref.~\cite{Abbott:2020tfl})). The choice of the  NR-surrogate model is motivated by the fact that they are the most accurate in the parameter range described by GW190521. In Fig.~\ref{fig:21g_systematics}, we indicate with a black cross what the injected \texttt{NRSur7dq4} signal predicts for the $(2,2)$ frequency and damping time,, and for comparison, we also show with a black solid curve the results obtained when recovering the the actual signal GW190521 with the waveform model $\pSEOB$. As seen in the plot, while the measurement of the frequency is consistent with the prediction, we overestimate the damping time.

The actual GW190521 event was observed at a GPS time, 1242442967.61 seconds (roughly 03:02:49 UTC, May 21, 2019). We select a time period of about 2.5 hours around this GPS time and create synthetic signals by injecting the \texttt{NRSur7dq4} GW signal in different stretches of real detector noise around the time of the actual GW event. The PSDs of GW detectors are expected to vary over
longer durations of time, and hence the 2.5 hour stretch of noise we consider can be assumed to have noise-properties similar to the time of the actual event. The results are indicated by green curves in Fig.~\ref{fig:21g_systematics}. As it can be seen from the figure, for 3 of the 5 noise realizations, corresponding to $t_0-1$ hour, $t_0+0.5$ hours, and $t_0+1$ hour we recover a damping time similar to the actual event, where $t_0$ is the GPS time of the actual event. For the other two noise realizations, we estimate the consistent damping time but an off-set frequency, while the fifth noise realization is consistent with both predictions. This seems to indicate that a bias in the measurements of the damping time for the actual event can be explained on account of an incomplete understanding of the noise at the time of the event.

The reader might question the judiciousness of using an aligned-spin waveform model, like $\pSEOB$ to measure a signal like GW190521 which appears to be precessing, especially because an incomplete understanding of the underlying signal can also lead to biases in measured quantities, as we have already demonstrated in Sec.~\ref{ssec:ngr_signal}. In order to explore possible effects of missing information about in-plane spins in $\pSEOB$, we repeat the above study of injecting synthetic signals using \texttt{NRSur7dq4} and recovering using the $\pSEOB$ template. But this time, instead of using real detector noise, we use Gaussian noise, i.e., realisations of noise sampled from a predicted detector PSD. Since the properties of the noise is completely understood in this case, any residual measurement biases can be completely attributed to diffferences in the waveform model. The 2D posterior distributions of the frequency and damping time measured using these Gaussian-noise signals is shown by the grey curves in Fig.~\ref{fig:21g_systematics}. We find the measurements to be completely consistent with the predictions of the frequency and damping time, thus concluding that a lack of in-plane spins in the $\pSEOB$ model does not affect our measurements of the QNM properties. The fact that the measurement of ringdown quantities are robust against an incomplete understanding of the inspral of a GW signal, is indeed one of the most strong points of our model.

