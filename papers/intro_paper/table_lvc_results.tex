\begin{table}
\begin{flushleft}
\begin{tabular}{llllllll}
\toprule
Event & \multicolumn{2}{c}{Redshifted} & \hphantom{X} & \multicolumn{2}{c}{Redshifted} \\
& \multicolumn{2}{c}{frequency [Hz]} & \hphantom{X} & \multicolumn{2}{c}{damping time [ms]} \\[0.075cm]
\hline
& IMR  & \pSEOB & \hphantom{X} & IMR  & \pSEOB \\
\hline

GW150914 &
$249^{+9}_{-7}$ &
$-$ &
\hphantom{X} &
$4.1^{+0.3}_{-0.2}$ &
$-$
\\[0.075cm]

GW170104 &
$286^{+16}_{i-27}$ &
$-$ &
\hphantom{X} &
$3.5^{+0.4}_{-0.3}$ &
$-$
\\[0.075cm]

GW170729 &
$161^{+13}_{-14}$ &
$-$ &
\hphantom{X} &
$7.8^{+1.8}_{-1.5}$ &
$-$
\\[0.075cm]

GW190630$\_$185205 &
$-$ &
$-$ &
\hphantom{X} &
$-$ &
$-$
\\[0.075cm]

GW190828$\_$063405 &
$-$ &
$-$ &
\hphantom{X} &
$-$ &
$-$
\\[0.075cm]
\hline
\bottomrule
\end{tabular}
\caption{\textcolor{red}{NOT COMPLETE} \comment{AB: We would need to give also the final mass and spins. I would suggest that we list in the Table
also the results from the TGR GWTC-2 paper, obtained with our method, so that all the results are in one Table.}}
\label{tab:qnm_o1o2_results}
\end{flushleft}
\end{table}
