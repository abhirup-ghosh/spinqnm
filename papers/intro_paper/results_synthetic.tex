\subsection{Simulations using GR signals in Gaussian noise} \label{ssec:gr_signal}

We demonstrate our method using synthetic-signal injections describing GWs
from BBHs in GR. We employ coloured Gaussian noise with PSDs expected for LIGO and
Virgo detectors at design sensitivities~\cite{AdvLIGOPSD,TheVirgo:2014hva}.
For the mock BBH signals, we choose parameters similar to two specific GW events, GW150914~\cite{Abbott:2016blz} and
GW190521~\cite{Abbott:2020tfl}. We list them in Table~\ref{tab:injection_values}.
These two binary systems are representative of the kind of systems for which
the QNM measurement is most suitable, notably high-mass BBH events which are loud enough that the
pre- and post-merger SNRs return reliable parameter-estimation results.

%%%%%%%%%%%%%%%%%%%%%%%%%%%%%%%%%%%%%%%%%%%%%%%%%%%%%%%%%%%%%%%
% Table for Injections
%%%%%%%%%%%%%%%%%%%%%%%%%%%%%%%%%%%%%%%%%%%%%%%%%%%%%%%%%%%%%%%
\begin{table}[h!]
\begin{center}
\begin{tabular}{ c|c|c|c|c|c|c|c }

 \ab{synthetic signal} & \makecell{$m_{\rm 1,det}$ \\$(\Mo)$} &  \makecell{$m_{\rm 2,det}$ \\ $(\Mo)$} & $\chi_{1}$ & $\chi_{2}$ & $\rho_\text{IMR}$ & $\rho_\text{insp}$ & $\rho_\text{postinsp}$ \\
 \hline
 GW150914-like & 39 & 31 & 0.0 & 0.0 & 25 & 22 & 12 \\
 GW190521-like & 150 & 120 & 0.02 & -0.39 & 20 & 8 & 18 \\
 SXS:BBH:0166 & 72 & 12  & 0.0 & 0.0 & 71 & 58 & 41 \\

\end{tabular}
\caption{Parameters of the synthetic-signal injections, chosen to be similar to the actual GW events indicated in the first column (first two rows). The parameters $(m_{\rm 1,det},m_{\rm 2,det})$ are the detector-frame masses of the primary and secondary BHs, respectively. The third row indicates the parameters of the SXS BBH waveform used in Sec.~\ref{ssec:nohairtheorem}. $\rho_\text{IMR}$, $\rho_\text{insp}$ and $\rho_\text{postinsp}$ are the SNR of the full IMR signal, SNR upto a certain cutoff frequency, and SNR after the cutoff frequency respectively. The cutoff frequency is assumed to be the frequency at the innermost circular stable orbit (ISCO) corresponding to the remnant Kerr black hole in each case.}
\label{tab:injection_values}
\end{center}
\end{table}
%%%%%%%%%%%%%%%%%%%%%%%%%%%%%%%%%%%%%%%%%%%%%%%%%%%%%%%%%%%%%%%
%%%%%%%%%%%%%%%%%%%%%%%%%%%%%%%%%%%%%%%%%%%%%%%%%%%%%%%%%%%%%%%


%%%%%%%%%%%%%%%%%%%%%%%%%%%%%%%%%%%%%%%%%%%%%%%%%%%%%%%%%%%%%%%
% Simulated siganl: GR
%%%%%%%%%%%%%%%%%%%%%%%%%%%%%%%%%%%%%%%%%%%%%%%%%%%%%%%%%%%%%%%
\begin{figure*}[hbt]
\begin{center}
        \includegraphics[width=0.5\textwidth]{figures/GW150914_simulated_signal_0p0_deltaf220_deltatau220.png}\includegraphics[width=0.5\textwidth]{figures/GW150914_simulated_signal_0p0_f220_tau220.png}
        \includegraphics[width=0.5\textwidth]{figures/GW190521_simulated_signal_0p0_deltaf220_deltatau220.png}\includegraphics[width=0.5\textwidth]{figures/GW190521_simulated_signal_0p0_f220_tau220.png}
        \caption{\textcolor{red}{FINAL RESULT} Posterior probability distribution on the fractional deviations in the frequency and damping time of the $(2,2)$ QNM, $(\df{220},\dtau{220})$ (left panels) and the reconstructed quantities, $(\fngr{220}, \taungr{220})$ (right panels) for GR injections with initial parameters similar to GW150914 (top panels) and GW190521 (bottom panels) (Table~\ref{tab:injection_values}). The 2D contour marks the 90\% credible region, while the dashed lines on the 1D marginalized distributions mark the 90\% credible levels. The black vertical and horizontal lines mark the injection values.}
        \label{fig:simulated_signal_GR}
\end{center}
\end{figure*}
%%%%%%%%%%%%%%%%%%%%%%%%%%%%%%%%%%%%%%%%%%%%%%%%%%%%%%%%%%%%%%%
%%%%%%%%%%%%%%%%%%%%%%%%%%%%%%%%%%%%%%%%%%%%%%%%%%%%%%%%%%%%%%%

To avoid possible systematic biases in our parameter-estimation analysis
due to error in waveform modeling, we use the GR version of the same waveform,
$\SEOB$ (without allowing for deviations in the QNM parameters) to
simulate our GW signal. And to avoid systematic biases due to noise,
we use an averaged (zero-noise) realization of the noise. \footnote{A detailed
study on noise systematics for one of the GW events is presented in
Appendix~\ref{sec:noise_systematics}.}  As in the case of the actual
detections, we consider a two-detector LIGO network at
Hanford and Livingston, having identical PSDs. The distance to the two 
synthetic events is rescaled such that the SNR in the detector network
is the same as the actual events (i.e., 24 for GW150914 and 14
for GW190521). Since mearly--equal-mass binaries like GW150914 and
GW190521 observed at moderately high SNRs are not expected to have a
loud ringdown signal, we restrict ourselves to estimating the
frequency and damping time of only one QNM $(\ell m) = (2,2)$, i.e.,
$\{\df{220},\dtau{220}\}$, while fixing the other QNM frequencies to
their GR values.

We find, as one might expect, that the posterior distribution on the
parameters describing fractional deviations in the frequency and
damping time are consistent with zero (left panels of
Fig.~\ref{fig:simulated_signal_GR}). One can then convert these
fractional quantities into absolute quantities using the relations
given in Eqs.~\ref{eq:nongr_freqs_a} and ~\ref{eq:nongr_freqs_b}, and
construct posterior distributions on these effective quantities,
$(\fngr{220}, \taungr{220})$ (right panels of
Fig.~\ref{fig:simulated_signal_GR}). In each of these cases, the recovered
two-dimensional posteriors are consistent with the GR predictions
(black dashed lines).


\subsection{Simulations using non-GR signals in Gaussian noise} \label{ssec:ngr_signal}


To demonstrate the robustness of the method in detecting possible
deviations from GR, we inject synthetic GW signals which are identical to
the corresponding GR prediction up to merger, and differ in their post-merger
description. We again choose binary-parameters
similar to GW150914 and GW190521 (as given in Table ~\ref{tab:injection_values}), but
set $\df{220} = \dtau{220} = 0.1 $.
In other words, we assume that the frequency and damping time
of our non-GR signal is 10\% more than the corresponding GR prediction,
although the pre-merger signal is identical to GR. In Fig.~\ref{fig:nongr_waveform}
we show this non-GR waveform, \texttt{pSEOBNR} with respect to the 
original GR template, \texttt{SEOBNR}. We see that the waveforms are identical in amplitude
and instanteneous frequency upto the merger (lower panel) , beyond which the 
red (GR template) and blue (non-GR template) diverge. We summarize the results of the Bayesian analysis in Fig.~\ref{fig:simulated_signal_nonGR} where we
show the posterior probability distributions for $(\df{220}, \dtau{220})$, or equivalently
$(\fngr{220}, \taungr{220})$. We find that they are consistent with the corresponding
values of the injection parameters, indicated by the black dashed lines.

%%%%%%%%%%%%%%%%%%%%%%%%%%%%%%%%%%%%%%%%%%%%%%%%%%%%%%%%%%%%%%%
%%%%%%%%%%%%%%%%%%%%%%%%%%%%%%%%%%%%%%%%%%%%%%%%%%%%%%%%%%%%%%%
\begin{figure}
        \includegraphics[width=0.5\textwidth]{figures/modGR_waveforms_amplitudephase.png}
        \caption{\textcolor{red}{FINAL RESULT} Top panel: The `+'--polarization of the gravitational waveform $h_+(t)$ from a GW150914-like event where the post-merger is described by GR (i.e., $\df{220} = \dtau{220} = 0$), and where the merger-ringdown is modified (i.e., $\df{220} = \dtau{220} = 0.1$). Bottom panel: Comparison of the evolution of the amplitude, $\tilde{h}(t)$ (left) and instantaneous frequency, $f(t)$ (right) for the GR and non-GR signal.}
        \label{fig:nongr_waveform}
\end{figure}
%%%%%%%%%%%%%%%%%%%%%%%%%%%%%%%%%%%%%%%%%%%%%%%%%%%%%%%%%%%%%%%
%%%%%%%%%%%%%%%%%%%%%%%%%%%%%%%%%%%%%%%%%%%%%%%%%%%%%%%%%%%%%%%

%%%%%%%%%%%%%%%%%%%%%%%%%%%%%%%%%%%%%%%%%%%%%%%%%%%%%%%%%%%%%%%
% Simulated siganl: non-GR
%%%%%%%%%%%%%%%%%%%%%%%%%%%%%%%%%%%%%%%%%%%%%%%%%%%%%%%%%%%%%%%
\begin{figure*}%[h!]
\begin{center}
        \includegraphics[width=0.5\textwidth]{figures/GW150914_simulated_signal_0p1_deltaf220_deltatau220.png}\includegraphics[width=0.5\textwidth]{figures/GW150914_simulated_signal_0p1_gr_ngr_fngrtaungr.png}
        \includegraphics[width=0.5\textwidth]{figures/GW190521_simulated_signal_0p1_deltaf220_deltatau220.png}\includegraphics[width=0.5\textwidth]{figures/GW190521_simulated_signal_0p1_gr_ngr_fngrtaungr.png}
        \caption{\textcolor{red}{FINAL RESULT} Posterior probability distribution on the fractional deviations in the frequency and damping time of the $(2,2)$ QNM, $(\df{220},\dtau{220})$ (left panels) and the reconstructed quantities, $(\fngr{220}, \taungr{220})$ (right panels) for non-GR injections with parameters of GW150914-like (top panels) and GW190521-like (bottom panels) as given in Table~\ref{tab:injection_values}. The non-GR signal has a deviation, $\df{220} = \dtau{220} = XX$. The 2D contour marks the 90\% credible region, while the dashed lines on the 1D marginalized distributions mark the 90\% credible levels. The black vertical and horizontal lines mark the injection values. In the right panels, we additionally show measurements using a GR ($\SEOB$) waveform, for the
GW150914-like (upper panel) and GW190521-like (lower panel) injections. The measurements with $\SEOB$ waveforms are visibly biased.}
        \label{fig:simulated_signal_nonGR}
\end{center}
\end{figure*}
%%%%%%%%%%%%%%%%%%%%%%%%%%%%%%%%%%%%%%%%%%%%%%%%%%%%%%%%%%%%%%%
%%%%%%%%%%%%%%%%%%%%%%%%%%%%%%%%%%%%%%%%%%%%%%%%%%%%%%%%%%%%%%%

We additionally investigate the effects of erroneously assuming that
an underlying non-GR signal can be well-described
by a GR one. We do this by estimating the parameters of our non-GR
signals using the GR waveform model $\SEOB$ instead of the parameterized $\pSEOB$.  The resulting one- and two-dimensional posteriors are shown in the right panels of Fig.~\ref{fig:simulated_signal_nonGR} by red curves for the GW150914-like (top) and GW190521-like (bottom) signals respectively. For both signals, we find the $\SEOB$ estimates are markedly biased with respect to the $\pSEOB$ estimates. This can be explained by the inability of the $\SEOB$ template waveform to capture all the physics in the $\pSEOB$ signal. We also notice that the results are distinctly
different for the two events. For the GW150914-like non-GR signal, the measurements of $(\fgr{220}, \taugr{220})$
(top right panel in Fig.~\ref{fig:simulated_signal_nonGR}) are consistent
with the $(\fngr{220}, \taungr{220})$ measurements for a signal with \emph{no
deviations} from GR (top right panel in Fig.~\ref{fig:simulated_signal_GR}). In other
words, if the actual signal had deviations from GR as large as the $10\%$, the analysis with the GR signal $\SEOB$ would likely have
reported \emph{no} deviation from the GR prediction. However in the
case of the GW190521-like non-GR signal, a simple GR analysis of
the non-GR signal would have yielded measurements distinctly
different from either of the two parameterized estimates: with and
without deviations. \comment{AB: Not clear to me the next sentence. We are working
in Gaussian noise, so why are you referring to the noise as the culprit or the
source of difference?} The fact that the GW190521-like signal has a much
lower SNR than GW150914-like signal might be a possible reason for the
measurement of the final quantities to be more susceptible to noise. A
more detailed comparison of the other parameters, like the masses and
spins, between an $\pSEOB$ and an $\SEOB$ measurement of a modified GR
signal is shown in Fig.~\ref{fig:gr_ngr_comparison}. \comment{AB: Please
comment the figures. What are the main results? What do we deduce?}


%%%%%%%%%%%%%%%%%%%%%%%%%%%%%%%%%%%%%%%%%%%%%%%%%%%%%%%%%%%%%%%
% modified GR signal: GR vs nonGR recovery comparison
%%%%%%%%%%%%%%%%%%%%%%%%%%%%%%%%%%%%%%%%%%%%%%%%%%%%%%%%%%%%%%%
\begin{figure*}%[h!]
        \includegraphics[width=0.5\textwidth]{figures/GW150914_simulated_signal_0p5_gr_ngr_m1m2.png}\includegraphics[width=0.5\textwidth]{figures/GW190521_simulated_signal_0p5_gr_ngr_m1m2.png}
        \includegraphics[width=0.5\textwidth]{figures/GW150914_simulated_signal_0p5_gr_ngr_a1za2z.png}\includegraphics[width=0.5\textwidth]{figures/GW190521_simulated_signal_0p5_gr_ngr_a1za2z.png}
        \includegraphics[width=0.5\textwidth]{figures/GW150914_simulated_signal_0p5_gr_ngr_fgrtaugr.png}\includegraphics[width=0.5\textwidth]{figures/GW190521_simulated_signal_0p5_gr_ngr_fgrtaugr.png}
        \includegraphics[width=0.5\textwidth]{figures/GW150914_simulated_signal_0p5_gr_ngr_Mfaf.png}\includegraphics[width=0.5\textwidth]{figures/GW190521_simulated_signal_0p5_gr_ngr_Mfaf.png}
        \caption{\textcolor{red}{FINAL RESULT} Comparison of \sout{the recovered} \ab{binary's} parameters when \sout{the underlying signal is assumed to be a GR signal or a modified GR signal} \ab{a non-GR signal ($\pSEOB$) is injected and recovered with a GR ($\SEOB$) or a non-GR ($\pSEOB$) waveform model}. \sout{In both cases, the actual underlying signal is a modified GR signal with parameters similar to} \ab{The left (right) panels refer to
a GW150914-like (GW190521- like) injected signal (see Table~\ref{tab:injection_values}) with QNM deviation parameters of $\df{220} = \dtau{220} = 0.5$.} \sout{For the GW150914 (GW190521) contours, the $SEOB$ and $pSEOB$ recoveries are indicated by blue (red) and pink (grey) curves respectively.} The panels (from top to bottom) show the recoveries in (detector-frame) \sout{initial} masses (first row), \sout{z-components of dimensionless initial} \ab(dimensionless) spins (second row), GR predictions of frequency and damping time (third row) and the remnant mass and spin predictions ($M_f$, $a_f$) from the frequency and damping time
(obtained by inverting the Berti fits). \comment{AB: I would write those details about the ``Berti's fits, etc.'' in the text, not in the caption. Also please specify if the remnant mass is the detector-frame mass. Please
switch the red and blue colors. The blue colors should always refer to the same waveform model, i.e., $\pSEOB$. Please indicate on the figures: GW150914-like and GW190521-like.}}
        \label{fig:gr_ngr_comparison}
\end{figure*}
%%%%%%%%%%%%%%%%%%%%%%%%%%%%%%%%%%%%%%%%%%%%%%%%%%%%%%%%%%%%%%%
%%%%%%%%%%%%%%%%%%%%%%%%%%%%%%%%%%%%%%%%%%%%%%%%%%%%%%%%%%%%%%%


\subsection{Test of the no-hair conjecture}\label{ssec:nohairtheorem}

Finally, we provide a simple demonstration of a test of the no-hair
theorem using our model. As described in the introduction, any test of
the no-hair theorem of BHs would need to involve independent
measurements of (at least) two different QNMs.

Here, we use an NR GW signal from the SXS catalog~\cite{Mroue:2013xna}
corresponding to a non-spinning BBH with mass-ratio $q=6$ (SXS:BBH:0166) and 
total mass $M=84 \Mo$ (see Table~\ref{tab:injection_values}).
We choose an asymmetric system to increase the SNR in the higher modes.
We also choose the distance and orientation of the binary
such that the total SNR in the three-detector network of LIGO Hanford, Livingston and
Virgo, is \macro{$\sim$ 70}. Based on the LIGO-Virgo observations during during the first three observing runs, 
such asymmetric and loud signals are no longer just a theoretical
prediction, but quite plausible at design sensitivities. Using this
signal, we attempt to measure both the $(2,\pm 2)$ and
$(3,\pm 3)$ QNMs. For this injected signal the SNR in
other sub-dominant modes is too low to be able to measure them.

We summarize our results in Fig.~\ref{fig:nohair_sxs}.  Given the injection parameters, the predicted values of the $(2,\pm 2)$ and $(3,\pm 3)$ frequency and damping time are \macro{(169.45 Hz, 4.68 ms)}  and \macro{(271.21 Hz, 4.50 ms)} respectively. The left panel of Fig.~\ref{fig:nohair_sxs} shows that the two-dimensional posteriors on the $(2,\pm 2)$ and $(3,\pm 3)$ QNMs are consistent with the predictions for a BBH merger in GR, indicated by the black plus sign.  Using fitting formulae provided in ~\cite{Berti:2005ys}, specifically,  Eqs. 2.1, E1, E3 and tables VIII and IX for the fitting coefficients, we infer the two-dimensional posterior probability distirbution on the final mass and spin for the $(2,\pm 2)$ (blue) and $(3,\pm 3)$ (red) QNMs in the right panel of Fig.~\ref{fig:nohair_sxs}. The two independent estimates are consistent with each other and correspond to a unique mass and spin for the remnant BH \macro{(83.08 $\Mo$, 0.37)} indicated by the plus sign. As a consequence, this may be considered as a test of the no-hair conjecture. For most of the events observed so far, the power in the $(3,\pm 3)$ has not been sufficient to measure it along with the $(2,\pm 2)$, or in fact, in its place. However, it might also be possible to combine information from multiple observation over the coming few years to obtain meaningful constraints on the $(3,\pm 3)$ and
other sub-dominant QNMs.

%%%%%%%%%%%%%%%%%%%%%%%%%%%%%%%%%%%%%%%%%%%%%%%%%%%%%%%%%%%%%%%

%%%%%%%%%%%%%%%%%%%%%%%%%%%%%%%%%%%%%%%%%%%%%%%%%%%%%%%%%%%%%%%
\begin{figure}
        \includegraphics[width=0.5\textwidth]{figures/nohair_sxs_0166.png}
        \caption{\textcolor{red}{PRELIMINARY} Posterior probability distribution on the fractional deviations (left panel) and the reconstructed (right panel) frequency and damping time of the $(2,\pm 2)$ (blue curves) and $(3,\pm 3)$ (red curves) QNM, respectively, when a NR signal with parameters $q=6$,  $M=84 \Mo$ and SNR $=75$ is injected in Gaussian noise and recovered with the $\pSEOB$ waveform model. The plus signs mark the GR predictions.}
        \label{fig:nohair_sxs}
\end{figure}
%%%%%%%%%%%%%%%%%%%%%%%%%%%%%%%%%%%%%%%%%%%%%%%%%%%%%%%%%%%%%%%
%%%%%%%%%%%%%%%%%%%%%%%%%%%%%%%%%%%%%%%%%%%%%%%%%%%%%%%%%%%%%%%
