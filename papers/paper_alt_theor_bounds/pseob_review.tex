

In this section we describe the waveform model used in this paper to measure properties of a BBH ringdown. As in \cite{Brito:2018rfr,Ghosh:2021mrv}, we use an inspiral-(plunge)-merger-ringdown (IMR) BBH waveform model where the parameters describing the remnant object are left additionally free and estimated directly from the data.

The GW signal from a BBH merger can be determined in GR by a unique set of parameters -- the masses and spins of the two BHs, $(m_1. m_2. \vec{s}_1, \vec{s}_2)$, the sky location determined by the luminosity distance $d_L$, right ascension $\alpha$ and declination $\delta$, and the orientation of the binary given by the inclination and polarisation angles, $(\iota, \psi)$. The set is completed by the choice of a reference time and phase, $(t_0. \phi_0)$. If we further assume that the spins of the individual BHs are restricted to be parallel to the orbital angular momentum, then we reduce the 6 components of spin to just 2, $(\chi_1, \chi_2)$ and our entire parameter set from 15 to 11. We define some additional parameters and set some conventions that will be useful in our analysis later, namely, the totat mass $M=m_1+m_2$, the chirp mass $\mathcal {M}=(m_{1}m_{2})^{3/5}/(m_{1}+m_{2})^{1/5}$, the asymmetric mass ratio $q=m_1/m_2$ with the convention $m_1 \geq m_2; q \geq 1$ and the symmetric mass ratio of the binary, $\nu = m_1m_2/(m_1+m_2)^2$.

The polarisations of the GW signal (in the observer's frame) are,
%
\begin{equation}
h_+(\iota,\varphi_0;t ) - i h_\times(\iota,\varphi_0;t) = \sum_{\ell, m} {}_{-\!2}Y_{\ell m}(\iota,\varphi_0)\, h_{\ell m}(t)\,,
\end{equation}
%
where ${}_{-\!2}Y_{\ell m}(\iota,\varphi_0)$ are the $-2$ spin-weighted spherical harmonics described by standard angular dependence indices $(\ell,m)$. As our baseline model, we use the computationally efficient (time-domain) multipolar waveform model for quasicircular spin-aligned BBH mergers described in \cite{Mihaylov:2021bpf} (henceforth referred to as \SEOB{} \footnote{This waveform model is available in \texttt{LALSuite} \cite{lalsuite} as the \texttt{SEOBNRv4HM\_PA} waveform approximant.}) which contains the modes, $(\ell, |m|)=(2,2),(2,1)$, $(3,3)$, $(4,4)$, and $(5,5)$~\cite{Cotesta:2018fcv,Mihaylov:2021bpf}. The model uses an effective-one-body approach combined with a post-adiabatic solution of the equations of motion to describe the inspiral-plunge waveform, $h_{\ell m}^\mathrm{insp-plunge}$. An accurate description of the merger is incorporated through calibration with NR simulations, as described in~\cite{Cotesta:2018fcv}, along with information for the merger and ringdown phases, from BH perturbation theory. The merger-ringdown waveform, $h_{\ell m}^\mathrm{merger-RD}$, is then stitched to inspiral-plunge waveform, $h_{\ell m}^\mathrm{insp-plunge}$ at a certain time $t = t_{\textrm{match}}^{\ell m}$, as
\begin{equation}
h_{\ell m}(t) = h_{\ell m}^\mathrm{insp-plunge}\, \theta(t_\mathrm{match}^{\ell m} - t) + h_{\ell m}^\mathrm{merger-RD}\,\theta(t-t_\mathrm{match}^{\ell m})\,,
\end{equation}
where $\theta(t)$ is the Heaviside step function. The merger-ringdown waveform is expressed as an exponentially damped sinusoid ~\citep{Bohe:2016gbl,Cotesta:2018fcv,Mihaylov:2021bpf}
\begin{equation}
\label{RD}
h_{\ell m}^{\textrm{merger-RD}}(t) = \nu \ \tilde{A}_{\ell m}(t)\ e^{i \tilde{\phi}_{\ell m}(t)} \ e^{-i \sigma_{\ell m 0}(t-t_{\textrm{match}}^{\ell m})},
\end{equation}
where $\sigma_{\ell m0} = 2\pi f_{\ell m 0} -i/\tau_{\ell m 0}$ are the complex frequencies of the fundamental (n=0 overtone) QNMs of the remnant BH. The frequencies $f_{\ell m}$ and damping times $/\tau_{\ell m}$ of the $(\ell m)$ QNM can be read off from the real and imaginary parts of  $\sigma_{\ell m0}$ respetively. The functions $\tilde{A}_{\ell m}(t)$ and $\tilde{\phi}_{\ell m}(t)$ are defined in~\cite{Bohe:2016gbl,Cotesta:2018fcv}.

\subsection{Construction of the merger-ringdown signal}
In the $\SEOB$ model~\cite{Mihaylov:2021bpf}, the complex frequencies $\sigma_{\ell m 0}$ are computed by first determining the final mass and spin from estimates oof the initial masses and spins through NR--fitting--formulas~\cite{Taracchini:2013rva,Hofmann:2016yih}. and then converting these estimates of the final mass and spin to the complex frequencies using BH perturbation theory--inspiral analytical fits outlined in~\cite{Berti:2005ys,Berti:2009kk}.


\iffalse
In the $\SEOB$ model constructed in Ref.~\cite{Cotesta:2018fcv}, the
complex frequencies $\sigma_{\ell m 0}$ are expressed in terms of the
final BH mass and spin~\cite{Berti:2005ys,Berti:2009kk}, and the
latter are related to the BBH's component masses and spins through
NR--fitting-formulas obtained in
GR~\cite{Taracchini:2013rva,Hofmann:2016yih}. Here instead, in the
spirit of what was done in \paperone, we promote the QNM (complex)
frequencies to be free parameters of the model, while keeping the
inspiral-plunge modes $h_{\ell m}^\mathrm{inspiral-plunge}(t)$ fixed
to their GR values. More explicitly, we introduce a parameterized
version of the $\SEOB$ model where the frequency and the
damping time of the ${\ell m 0}$ mode (i.e, $(f_{\ell m 0}, \tau
_{\ell m 0})$) are defined through the fractional deviations, $(\delta
f_{\ell m 0},\delta \tau_{\ell m 0})$, from the corresponding GR
values~\cite{Gossan:2011ha,Meidam:2014jpa}.


Thus,
\begin{subequations}
\begin{eqnarray}
f_{\ell m 0} &=& f_{\ell m 0}^{\text{GR}}\, (1 + \delta f_{\ell m 0})\,,\label{eq:nongr_freqs_a} \\ 
\tau _{\ell m 0} &=& \tau _{\ell m 0}^{\text{GR}}\, (1 + \delta \tau_{\ell m 0})\,. \label{eq:nongr_freqs_b}
\end{eqnarray}
\end{subequations}

The GR quantities $( f_{\ell m 0}^{GR},\tau_{\ell m 0}^{GR})$ are
constructed using the NR--fitting--formula from Refs.~\cite{Taracchini:2013rva,Hofmann:2016yih}, and are functions of the initial masses and spins, $(m_1, m_2, \chi_1, \chi_2)$. Hence,
\begin{subequations}
\begin{eqnarray}
f_{\ell m 0} &=& f_{\ell m 0}(m_1, m_2, \chi_1, \chi_2, \delta f_{\ell m 0}, \delta \tau_{\ell m 0})\,,\\ 
\tau _{\ell m 0} &=& \tau _{\ell m 0}(m_1, m_2, \chi_1, \chi_2, \delta f_{\ell m 0}, \delta \tau_{\ell m 0})\,.
\end{eqnarray}
\end{subequations}
We denote such a parameterized waveform model $\pSEOB$~\footnote{This
waveform model is called {\tt pSEOBNRv4HM} in LAL.}. We note that when leaving $\sigma_{\ell m}$ to vary
freely, the functions $\tilde{A}_{\ell m}(t)$ and $\tilde{\phi}_{\ell
  m}(t)$ in general also differ from the GR predictions, since
those functions depend on the QNM complex frequencies, as can be seen
from the expressions for $c_{i,c}^{\ell m}$ and $d_{1,c}^{\ell m}$ in Eqs.~(\ref{c1}),
(\ref{c2}), and (\ref{d1}). As a consequence, the ringdown signal (amplitude and phase) 
soon after merger deviates from the one predicted by GR.
\fi