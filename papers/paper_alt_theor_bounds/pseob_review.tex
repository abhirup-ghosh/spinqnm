

In this section we describe the waveform model used in our paper to measure properties of a BBH ringdown. As in \cite{Brito:2018rfr,Ghosh:2021mrv}, we use an inspiral-(plunge)-merger-ringdown (IMR) BBH waveform model where the parameters describing the remnant object are left additionally free and estimated directly from the data.

The GW signal from a BBH merger can be determined in GR by a unique set of parameters -- the masses and spins of the two BHs, $(m_1,\, m_2,\, \vec{s}_1,\, \vec{s}_2)$, the sky location determined by the luminosity distance $d_L$, right ascension $\alpha$ and declination $\delta$, and the orientation of the binary given by the inclination and polarisation angles, $(\iota, \psi)$. The set is completed by the choice of a reference time and phase, $(t_0,\, \phi_0)$. If we further assume that the spins of the individual BHs are restricted to be parallel to the orbital angular momentum, then we reduce the 6 components of spin to just 2, $(\chi_1, \chi_2)$ and our entire parameter set from 15 to 11. We define some additional parameters and set some conventions that will be useful in our analysis later, namely, the totat mass $M=m_1+m_2$, the chirp mass $\mathcal {M}=(m_{1}m_{2})^{3/5}/(m_{1}+m_{2})^{1/5}$, the asymmetric mass ratio $q=m_1/m_2$ with the convention $m_1 \geqslant m_2; q \geqslant 1$ and the symmetric mass ratio of the binary, $\nu = m_1m_2/(m_1+m_2)^2$.

The polarisations of the GW signal (in the observer's frame) are,
%
\begin{equation}
h_+(\iota,\varphi_0;t ) - i h_\times(\iota,\varphi_0;t) = \sum_{l, m} {}_{-\!2}Y_{l m}(\iota,\varphi_0)\, h_{l m}(t)\,,
\end{equation}
%
where ${}_{-\!2}Y_{l m}(\iota,\varphi_0)$ are the $-2$ spin-weighted spherical harmonics described by standard angular dependence indices $(l,m)$.\footnote{Note that we use $l$ to denote the angular dependence index and $\ell$ to denote the coupling constant in Sec.~\ref{sec:review_parspec}.} As our baseline model, we use the computationally efficient (time-domain) multipolar waveform model for quasicircular spin-aligned BBH mergers described in \cite{Mihaylov:2021bpf} (henceforth referred to as \SEOB{} \footnote{This waveform model is available in \texttt{LALSuite} \cite{lalsuite} as the \texttt{SEOBNRv4HM\_PA} waveform approximant.}) which contains the modes, $(l, |m|)=(2,2),(2,1)$, $(3,3)$, $(4,4)$, and $(5,5)$~\cite{Cotesta:2018fcv,Mihaylov:2021bpf}. The model uses an effective-one-body approach combined with a post-adiabatic solution of the equations of motion to describe the inspiral-plunge waveform, $h_{l m}^\mathrm{insp-plunge}$. An accurate description of the merger is incorporated through calibration with NR simulations, as described in~\cite{Cotesta:2018fcv}, along with information for the merger and ringdown phases, from BH perturbation theory. The merger-ringdown waveform, $h_{l m}^\mathrm{merger-RD}$, is then stitched to inspiral-plunge waveform, $h_{l m}^\mathrm{insp-plunge}$ at a certain time $t = t^{\textrm{match}}_{l m}$, as
%
\begin{equation}
h_{l m}(t) = h_{l m}^\mathrm{insp-plunge}\, \theta(t_\mathrm{match}^{l m} - t) + h_{l m}^\mathrm{merger-RD}\,\theta(t-t_\mathrm{match}^{l m})\,,
\end{equation}
%
where $\theta(t)$ is the Heaviside step function. The merger-ringdown waveform is expressed as an exponentially damped sinusoid ~\citep{Bohe:2016gbl,Cotesta:2018fcv,Mihaylov:2021bpf}
%
\begin{equation}
\label{RD}
h_{l m}^{\textrm{merger-RD}}(t) = \nu \ \tilde{A}_{l m}(t)\ e^{i \tilde{\phi}_{l m}(t)} \ e^{-i \sigma_{l m 0}(t-t^{\textrm{match}}_{l m})},
\end{equation}
%
where
%
\begin{equation}
\sigma_{l m0} = \omega_{l m 0} - i / \tau_{l m 0}\,,
\end{equation}
%
are the complex frequencies of the fundamental (n=0 overtone) QNMs of the remnant BH. The frequencies $\omega_{l m 0}$ and damping times $\tau_{l m 0}$ of the $(l, m)$ QNM can be read off from the real and imaginary parts of  $\sigma_{l m0}$ respetively. The functions $\tilde{A}_{l m}(t)$ and $\tilde{\phi}_{l m}(t)$ are defined in~\cite{Bohe:2016gbl,Cotesta:2018fcv}.

In the $\SEOB$ model~\cite{Mihaylov:2021bpf}, the complex frequencies $\sigma_{l m 0}$ are computed by first determining the final mass and spin from estimates of the initial masses and spins through NR--fitting--formulas~\cite{Taracchini:2013rva,Hofmann:2016yih} and then converting them to the complex frequencies using BH perturbation theory--inspiral analytical fits outlined in~\cite{Berti:2005ys,Berti:2009kk}. Hence,

\begin{subequations}
\begin{eqnarray}
\omega_{l m 0}^{\text{GR}} &=& \omega_{l m 0}^{\text{GR}}(m_1, m_2, \chi_1, \chi_2)\,,\\
\tau _{l m 0}^{\text{GR}} &=& \tau _{l m 0}^{\text{GR}}(m_1, m_2, \chi_1, \chi_2)\,.
\end{eqnarray}
\end{subequations}
where $(\omega_{l m 0}^{\text{GR}}, \tau _{l m 0}^{\text{GR}} )$ refer to the QNM predictions in baseline \SEOB{} model.  In this paper, however, we introduce a completely different expression of the QNM frequencies $(\omega_{l m 0}, \tau _{l m 0})$, as a series expansion on the spin of the remnant object $\chi_f$ [see Eqs.~\eqref{eq:kerr_expansion}].  For the rest of the paper, in keeping with the conventions introduced in Sec.~\ref{sec:review_parspec}, we are going to use the index $k$ to refer to $(l, m,n=0)$ mode. Accordingly, we are going to use $(\omega_k, \tau _k)$ to refer to $(\omega_{l m 0}, \tau _{l m 0})$ respectively.

In this formalism our QNM frequencies are expressed through functions,
%
\begin{subequations}
\begin{align}
\omega_k &= \omega_k(m_1, m_2, \chi_1, \chi_2,\ell, \{\delta \omega_k^{(j)}\})\\
\tau_k   &= \tau _k(m_1, m_2, \chi_1, \chi_2, \ell, \{\delta \tau_k^{(j)}\})
\end{align}
\end{subequations}
%
where the estimates of $(m_1,\, m_2,\, \chi_1,\, \chi_1)$ are used to predict the final mass and spin $(M_f,\, \chi_f)$ through~\cite{Taracchini:2013rva,Hofmann:2016yih}.
%
Additionally, $(\omega_k,\, \tau_k)$ depend on the coupling constant $\ell$ defined in Eq.~\eqref{eq:def_gamma} as well as the sets, $\{\delta \omega_k^{(j)}\},$ and $\{\delta \tau_k^{(j)}\}$, where $j$ ranges from 0 to the power of spin in Eqs.~\ref{eq:kerr_expansion} upto which we include corrections.
%
For example, if we restrict ourselves to corrections in the $j=0$, or $\chi_f^0$ terms, then we just have two additional degrees of freedom, $(\omega_k^{(0)}, \tau_k^{(0)})$ per mode $k$. We will call this parametrized waveform model $\pSEOB$ in this paper.
%
In this paper, we will restrict ourselves to the $k=(2,2,0)$ QNM and corrections to the $j=0$ and $j=1$ terms, i.e., leading and next-to-leading order corrections to the spin, $\chi_f$. Assuming a prior which is uniform on the beyond-GR parameters, $(\ell, \{\delta \omega_k^{(j)}\},\{\delta \tau_k^{(j)}\})$, we employ a Bayesian framework to stochastically sample over the parameter space using the \texttt{LALInference} algorithm.~\cite{lallsuite}.
%
The bounds on these beyond-GR parameters, for specific cases of modified gravity theories outlined in Sec.~\ref{sec:review_theories}, are provided in Sec.~\ref{sec:results}.




\iffalse
In the $\SEOB$ model constructed in Ref.~\cite{Cotesta:2018fcv}, the
complex frequencies $\sigma_{l m 0}$ are expressed in terms of the
final BH mass and spin~\cite{Berti:2005ys,Berti:2009kk}, and the
latter are related to the BBH's component masses and spins through
NR--fitting-formulas obtained in
GR~\cite{Taracchini:2013rva,Hofmann:2016yih}. Here instead, in the
spirit of what was done in \paperone, we promote the QNM (complex)
frequencies to be free parameters of the model, while keeping the
inspiral-plunge modes $h_{l m}^\mathrm{inspiral-plunge}(t)$ fixed
to their GR values. More explicitly, we introduce a parameterized
version of the $\SEOB$ model where the frequency and the
damping time of the ${l m 0}$ mode (i.e, $(\omega_{l m 0}, \tau
_{l m 0})$) are defined through the fractional deviations, $(\delta
\omega_{l m 0},\delta \tau_{l m 0})$, from the corresponding GR
values~\cite{Gossan:2011ha,Meidam:2014jpa}.


Thus,
\begin{subequations}
\begin{eqnarray}
\omega_{l m 0} &=& \omega_{l m 0}^{\text{GR}}\, (1 + \delta \omega_{l m 0})\,,\label{eq:nongr_freqs_a} \\
\tau _{l m 0} &=& \tau _{l m 0}^{\text{GR}}\, (1 + \delta \tau_{l m 0})\,. \label{eq:nongr_freqs_b}
\end{eqnarray}
\end{subequations}

The GR quantities $( \omega_{l m 0}^{GR},\tau_{l m 0}^{GR})$ are
constructed using the NR--fitting--formula from Refs.~\cite{Taracchini:2013rva,Hofmann:2016yih}, and are functions of the initial masses and spins, $(m_1, m_2, \chi_1, \chi_2)$. Hence,
\begin{subequations}
\begin{eqnarray}
\omega_{l m 0} &=& \omega_{l m 0}(m_1, m_2, \chi_1, \chi_2, \delta \omega_{l m 0}, \delta \tau_{l m 0})\,,\\
\tau _{l m 0} &=& \tau _{l m 0}(m_1, m_2, \chi_1, \chi_2, \delta \omega_{l m 0}, \delta \tau_{l m 0})\,.
\end{eqnarray}
\end{subequations}
We denote such a parameterized waveform model $\pSEOB$~\footnote{This
waveform model is called {\tt pSEOBNRv4HM} in LAL.}. We note that when leaving $\sigma_{l m}$ to vary
freely, the functions $\tilde{A}_{l m}(t)$ and $\tilde{\phi}_{l
  m}(t)$ in general also differ from the GR predictions, since
those functions depend on the QNM complex frequencies, as can be seen
from the expressions for $c_{i,c}^{l m}$ and $d_{1,c}^{l m}$ in Eqs.~(\ref{c1}),
(\ref{c2}), and (\ref{d1}). As a consequence, the ringdown signal (amplitude and phase)
soon after merger deviates from the one predicted by GR.
\fi
