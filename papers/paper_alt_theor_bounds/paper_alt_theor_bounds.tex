\documentclass[twocolumn,prd,aps,superscriptaddress,preprintnumbers,tightenlines,showpacs,nofootinbib,eqsecnum,amsfonts,amsmath,longbibliography]{revtex4-2}
\usepackage{epsfig}
\usepackage{graphics}
\usepackage{graphicx}
\usepackage{bm}
\usepackage[dvipsnames]{xcolor}
\usepackage{bm}
% \usepackage{times}
% \usepackage[varg]{txfonts}
\usepackage{txfonts}
\usepackage{xspace}
\usepackage[normalem]{ulem} % To get strikethrough (\sout)
\usepackage[colorlinks]{hyperref}
\usepackage[caption=false]{subfig}
\usepackage{booktabs}
\usepackage{url}
\usepackage{float}
\usepackage[bottom]{footmisc}
\usepackage{lineno}
%\linenumbers
\usepackage{makecell}
\usepackage{microtype}

\definecolor{LinkColor}{rgb}{0.75, 0, 0}
\definecolor{CiteColor}{rgb}{0, 0.5, 0.5}
\definecolor{UrlColor}{rgb}{0, 0, 0.75}
\hypersetup{linkcolor=LinkColor}
\hypersetup{citecolor=CiteColor}
\hypersetup{urlcolor=UrlColor}
\usepackage{perpage}
\MakePerPage{footnote}

\newcommand{\paperone}{Paper~I\xspace}

\newcommand{\h}{\mathpzc{h}}
\newcommand{\Hhat}{\hat{\mathpzc{H}}}
\newcommand{\B}{\mathpzc{B}}
\newcommand{\hlm}{\mathpzc{h}_{\ell m}}
\newcommand{\xilm}{\xi_{\ell m}}
\newcommand{\Ylm}{{Y}^{-2}_{\ell m}}
\newcommand{\Y}{{Y}^{-2}}
\newcommand{\hc}{h_\times}
\newcommand{\hp}{h_+}
\newcommand{\Fc}{F_\times}
\newcommand{\Fp}{F_+}
\newcommand{\Mf}{M_f}
\newcommand{\cA}{\mathpzc{A}}
\newcommand{\lm}{_{\ell m}}
\newcommand{\deff}{d_\mathrm{eff}}
\newcommand{\rmi}{\mathrm{i}}
\newcommand{\blambda}{\bm{\lambda}}
\newcommand{\btheta}{\bm{\theta}}
\newcommand{\bxi}{\bm{\xi}}
\newcommand{\bxigr}{\bm{\xi}_{\text{GR}}}
\newcommand{\bxingr}{\bm{\xi}_{\text{nGR}}}
\newcommand{\bzeta}{\bm{\zeta}}
\newcommand{\bs}[1]{\bm{\vec{S}_{#1}}}
\newcommand{\Mo}{M_{\odot}}
\newcommand{\FFe}{\mathrm{FF}_\mathrm{eff}}
\newcommand{\FF}{\mathrm{FF}}
\newcommand{\e}{\mathrm{e}}
\newcommand{\rhoopt}{\rho_\mathrm{opt}}
\newcommand{\rhosubopt}{\rho_\mathrm{subopt}}
\newcommand{\fqnm}{f}
\newcommand{\sigmaqnm}{\sigma}
\newcommand{\n}{\mathbf{n}}
\newcommand*{\skymapscale}{0.5}
\newcommand*{\paramestscale}{0.455}
\newcommand{\df}[1]{\delta f_{\text{#1}}}
\newcommand{\dtau}[1]{\delta \tau_{\text{#1}}}
\newcommand{\fngr}[1]{f_{\text{#1}}}
\newcommand{\taungr}[1]{\tau_{\text{#1}}}
\newcommand{\fgr}[1]{f ^{\text{GR}}_{\text{#1}}}
\newcommand{\taugr}[1]{\tau ^{\text{GR}}_{\text{#1}}}
\newcommand{\pSEOB}{\texttt{pSEOBNR}}
\newcommand{\SEOB}{\texttt{SEOBNR}}

\newcommand{\pd}{\partial}
\newcommand{\dd}{{\rm d}}
\newcommand{\dV}{{\rm d}^{4}x \, \sqrt{-g} \,}
\newcommand{\lame}{\lambda_{\rm e}}
\newcommand{\lamo}{\lambda_{\rm o}}

% Comment commands
\newcommand{\ag}[1]{{\textcolor{cyan}{{[AG: #1]}} }}
\newcommand{\hs}[1]{{\textcolor{blue}{{[HS: #1]}} }}
\newcommand{\ab}[1]{{\textcolor{green}{{[AB: #1]}} }}

\newcommand{\AEI}{\affiliation{Max Planck Institute for Gravitational Physics (Albert Einstein Institute), Am M\"uhlenberg 1, Potsdam 14476, Germany}}
\newcommand{\UMD}{\affiliation{Department of Physics, University of Maryland, College Park, Maryland 20742, USA}}

\begin{document}

% HS: temporary title. Feel free to add suggestions
\title{Constraints on higher-curvature gravity theories from binary black hole ringdown signals}
% \title{Black-hole ringdown as probe of higher-curvature gravity theories}

\author{Abhirup Ghosh}
\author{Hector O. Silva}
\AEI
\author{Alessandra Buonanno}
\AEI
\UMD

\date{\today}


%%%%%%%%%%%%%%%%%%%%%%%
\begin{abstract}
\end{abstract}
%%%%%%%%%%%%%%%%%%%%%%%

\maketitle

%%%%%%%%%%%%%%%%%
\section{Introduction}
\label{sec:intro}
%%%%%%%%%%%%%%%%%

%%%%%%%%%%%%%%%%%
\section{Methods}
\label{sec:method}

\subsection{The parametrized ringdown spin expansion coefficients}
\label{sec:review_parspec}

\hs{Here we briefly review~\cite{Maselli:2019mjd,Carullo:2021dui}.}

\subsection{The parametrized waveform model}
\label{sec:review_pSEOB}

\hs{Here we present a short review of~\cite{Brito:2018rfr,Ghosh:2021mrv} and
explain how we modified it to include the {\sc Parspec} parametrisation.}

\subsection{Overview of modified gravity theories}
\label{sec:review_theories}

\hs{Here we review class of gravity theories we will consider.
They all have higher-curvature terms, hence the `higher-curvature gravity theories'
in the title.}

A broad class modified gravity theories are capture by the action
%
\begin{align}
S &= \frac{1}{16 \pi G} \int \dd^4x \sqrt{-g}
\left[ R + \dots \right.
\nonumber \\
&\quad
\left.
\right]
\end{align}

%%%%%%%%%%%%%%%%%

%%%%%%%%%%%%%%%%%
\section{Methods}
\label{sec:methods}
%%%%%%%%%%%%%%%%%

\begin{table}[th]
\begin{tabular}{c c c c c}
\hline
\hline
Theory & $p$ & $\delta \omega$ & $\delta \tau$ & Refs. \\
\hline
\dots & \dots & \dots & \dots & \dots \\
\hline
\hline
\end{tabular}
\caption{Summary of theories consider in this paper.
\hs{Add a column with the constraint (if any) that we placed.}
\hs{TODO: add which runs are done and which need to be done.}
}
\label{tab:ref_theories}
\end{table}

\hs{We {\it must very explicitly} state our working hypothesis here:
%
\begin{itemize}
    \item That we include only the nonrotating, non-GR correction to the QNMs.
    \item That due to absence of isospectrality, when translating to the theory specific result,
    we {\it chose} to use the lowest damping, gravitational QNM of the theory.
\end{itemize}
%
We can come back to these in the Sec.~\ref{sec:discussion} as things that we
might want to revisit in the future.
}

\hs{In Table~\ref{tab:ref_theories} we summarize the relevant parameters for
the {\sc Parspec} implementation of each theory and give credit to the papers
which calculated the QNMs.}

\hs{Say which GW event we will consider and justify why.}

%%%%%%%%%%%%%%%%%
\section{Results}
\label{sec:results}
%%%%%%%%%%%%%%%%%

\hs{Here we summarise the results of our PE runs. Combine the posteriors on the
the non-GR parameter coming from different events. We could have a short
subsection for each theory. We should quote two thresholds for claiming
that we placed a bound, one using the secondary BH mass and another with
the final BH mass.}

%%%%%%%%%%%%%%%%%
\section{Discussion}
\label{sec:discussion}
%%%%%%%%%%%%%%%%%

\hs{I think an important message of the paper is that we give indication that
there are theories of gravity (such as dCS) which can by-pass observational
constraints from the inspiral phase alone, {\it yet} they do not, if we
analyse the ringdown. This is quite important because I've frequently heard
that the ``inspiral is what will constrain theories''. I think this might be
the most important message.}

%%%%%%%%%%%%%%%%%
\section*{Acknowledgements}
\label{sec:acknowledgements}
%
We thank Emanuele Berti and Andrea Maselli for discussions.
%
\hs{Anyone else? Serguei, Deyan?}
%
We thank the computational resources provided by the AEI, specifically the
{\sc Hypatia} cluster.
%
The authors would like to thank everyone at the frontline of the Covid-19
pandemic.
%%%%%%%%%%%%%%%%%

% \bibliographystyle{apsrev}
\bibliography{paper_alt_theor_bounds}

\end{document}
