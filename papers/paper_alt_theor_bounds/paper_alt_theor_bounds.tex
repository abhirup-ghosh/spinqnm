\documentclass[twocolumn,prd,aps,superscriptaddress,preprintnumbers,tightenlines,showpacs,nofootinbib,eqsecnum,amsfonts,amsmath,longbibliography]{revtex4-2}
\usepackage{epsfig}
\usepackage{graphics}
\usepackage{graphicx}
\usepackage{bm}
\usepackage[dvipsnames]{xcolor}
\usepackage{bm}
% \usepackage{times}
% \usepackage[varg]{txfonts}
\usepackage{txfonts}
\usepackage{xspace}
\usepackage[normalem]{ulem} % To get strikethrough (\sout)
\usepackage[colorlinks]{hyperref}
\usepackage[caption=false]{subfig}
\usepackage{booktabs}
\usepackage{url}
\usepackage{float}
\usepackage[bottom]{footmisc}
\usepackage{lineno}
\usepackage{mathrsfs}
%\linenumbers
\usepackage{makecell}
\usepackage{microtype}

\definecolor{LinkColor}{rgb}{0.75, 0, 0}
\definecolor{CiteColor}{rgb}{0, 0.5, 0.5}
\definecolor{UrlColor}{rgb}{0, 0, 0.75}
\hypersetup{linkcolor=LinkColor}
\hypersetup{citecolor=CiteColor}
\hypersetup{urlcolor=UrlColor}
\usepackage{perpage}
\MakePerPage{footnote}

\newcommand{\paperone}{Paper~I\xspace}

\newcommand{\h}{\mathpzc{h}}
\newcommand{\Hhat}{\hat{\mathpzc{H}}}
\newcommand{\B}{\mathpzc{B}}
\newcommand{\hlm}{\mathpzc{h}_{\ell m}}
\newcommand{\xilm}{\xi_{\ell m}}
\newcommand{\Ylm}{{Y}^{-2}_{\ell m}}
\newcommand{\Y}{{Y}^{-2}}
\newcommand{\hc}{h_\times}
\newcommand{\hp}{h_+}
\newcommand{\Fc}{F_\times}
\newcommand{\Fp}{F_+}
\newcommand{\Mf}{M_f}
\newcommand{\cA}{\mathpzc{A}}
\newcommand{\lm}{_{\ell m}}
\newcommand{\deff}{d_\mathrm{eff}}
\newcommand{\rmi}{\mathrm{i}}
\newcommand{\blambda}{\bm{\lambda}}
\newcommand{\btheta}{\bm{\theta}}
\newcommand{\bxi}{\bm{\xi}}
\newcommand{\bxigr}{\bm{\xi}_{\text{GR}}}
\newcommand{\bxingr}{\bm{\xi}_{\text{nGR}}}
\newcommand{\bzeta}{\bm{\zeta}}
\newcommand{\bs}[1]{\bm{\vec{S}_{#1}}}
\newcommand{\Mo}{M_{\odot}}
\newcommand{\FFe}{\mathrm{FF}_\mathrm{eff}}
\newcommand{\FF}{\mathrm{FF}}
\newcommand{\e}{\mathrm{e}}
\newcommand{\rhoopt}{\rho_\mathrm{opt}}
\newcommand{\rhosubopt}{\rho_\mathrm{subopt}}
\newcommand{\fqnm}{f}
\newcommand{\sigmaqnm}{\sigma}
\newcommand{\n}{\mathbf{n}}
\newcommand*{\skymapscale}{0.5}
\newcommand*{\paramestscale}{0.455}
\newcommand{\df}[1]{\delta f_{\text{#1}}}
\newcommand{\dtau}[1]{\delta \tau_{\text{#1}}}
\newcommand{\fngr}[1]{f_{\text{#1}}}
\newcommand{\taungr}[1]{\tau_{\text{#1}}}
\newcommand{\fgr}[1]{f ^{\text{GR}}_{\text{#1}}}
\newcommand{\taugr}[1]{\tau ^{\text{GR}}_{\text{#1}}}
\newcommand{\pSEOB}{\texttt{pSEOBNR}}
\newcommand{\SEOB}{\texttt{SEOBNR}}

\newcommand{\pd}{\partial}
\newcommand{\dd}{{\rm d}}
\newcommand{\dV}{{\rm d}^{4}x \, \sqrt{-g} \,}
\newcommand{\lame}{\lambda_{\rm e}}
\newcommand{\lamo}{\lambda_{\rm o}}

% Comment commands
\newcommand{\ag}[1]{{\textcolor{cyan}{{[AG: #1]}} }}
\newcommand{\hs}[1]{{\textcolor{blue}{{[HS: #1]}} }}
\newcommand{\ab}[1]{{\textcolor{green}{{[AB: #1]}} }}

\newcommand{\AEI}{\affiliation{Max Planck Institute for Gravitational Physics (Albert Einstein Institute), Am M\"uhlenberg 1, Potsdam 14476, Germany}}
\newcommand{\UMD}{\affiliation{Department of Physics, University of Maryland, College Park, Maryland 20742, USA}}

\begin{document}

% HS: temporary title. Feel free to add suggestions
% \title{Probing higher-curvature gravity theories from binary black hole ringdown signals}
\title{Black-hole ringdown as probe of higher-curvature gravity theories}

\author{Abhirup Ghosh}
\author{Hector O. Silva}
\AEI
\author{Alessandra Buonanno}
\AEI
\UMD

\date{\today}


%%%%%%%%%%%%%%%%%%%%%%%
\begin{abstract}
\end{abstract}
%%%%%%%%%%%%%%%%%%%%%%%

\maketitle
\tableofcontents

%%%%%%%%%%%%%%%%%
\section{Introduction}
\label{sec:intro}
%%%%%%%%%%%%%%%%%

%%%%%%%%%%%%%%%%%
\section{Methods}
\label{sec:method}

\subsection{The parametrized ringdown spin expansion coefficients formalism}
\label{sec:review_parspec}

\hs{Here we briefly review~\cite{Maselli:2019mjd,Carullo:2021dui}.}

The formalism modifies the Kerr quasinormal mode frequency as
%
\begin{subequations}
\begin{align}
\omega_{\iota} &= \omega^{\rm K}_{\iota} \, (1 + \delta \omega_{\iota}), \\
\tau_{\iota}   &= \tau^{\rm K}_{\iota}   \, (1 + \delta \tau_{\iota}),
\end{align}
\label{eq:general_deviation}
\end{subequations}
%
where the $\iota$ collectively represents the QNM labels $\{l,\, m,\, n\}$.


Further expand the Kerr expression as:
%
\begin{subequations}
\begin{align}
\omega_{\iota} &= (1/M) \sum_{j = 0}^{N} \, \chi^{j} \omega^{(j)}_{k} \, \left( 1 + \gamma \delta \omega^{(j)}_{k} \right), \\
\tau_{\iota}   &= M     \sum_{j = 0}^{N} \, \chi^{j} \tau^{(j)}_{k}   \, \left( 1 + \gamma \delta \tau^{(j)}_{k} \right),
\end{align}
\label{eq:kerr_expansion}
\end{subequations}

Also,
%
\begin{equation}
\gamma = \frac{\kappa}{M_{\rm s}^{p}} = \frac{\kappa (1 + z)^{p}}{M^{p}}
= \left[
\frac{\ell c^2 (1 + z)}{G M}
\right]^{p}
\end{equation}
%

\subsection{The parametrized waveform model}
\label{sec:review_pSEOB}

\hs{Here we present a short review of~\cite{Brito:2018rfr,Ghosh:2021mrv} and
explain how we modified it to include the {\sc Parspec} parametrisation.}

\subsection{Overview of modified gravity theories}
\label{sec:review_theories}

\hs{Here we review class of gravity theories we will consider.
They all have higher-curvature terms, hence the `higher-curvature gravity theories'
in the title.}

A broad class modified gravity theories are described by the action
%
\begin{equation}
S = \frac{1}{16 \pi G} \int \dd^4x \sqrt{-g}
\left[ R
- \tfrac{1}{2}(\nabla \varphi)^2
\right]
+
\sum_{n \geqslant 2} \frac{\ell^{2n - 2}}{16 \pi G} S^{(2n)},
\end{equation}
%
where $\varphi$ is a dynamical scalar field, $\ell$ some length scale, assumed
to be small compares to the length scale associated with a black hole, i.e.,
$GM \gg \ell$, and $S^{(2n)}$ are corrections to the Einstein-Klein-Gordon
action that introduce involving higher-order curvature tensors (with $2n$
metric derivatives) and scalar field.

More specifically, we work up to dimension-eight operators (i.e., $n=4$),
%
\begin{subequations}
\begin{align}
S^{(4)} &= \alpha_{\rm GB} R^{2}_{\rm GB} + \alpha_{\rm CS} {}^{*}RR,
\label{eq:s_4}
\\
S^{(6)} &= \lame R_{\mu\nu}{}^{\rho\sigma} R_{\rho\sigma}{}^{\gamma\delta} R_{\gamma \delta}{}^{\mu\nu}
+ \lamo \ell^4 R_{\mu\nu}{}^{\rho\sigma} R_{\rho\sigma}{}^{\gamma\delta} \tilde{R}_{\gamma \delta}{}^{\mu\nu},
\label{eq:s_6}
\\
S^{(8)} &= \epsilon_{1} \mathcal{C}^2  + \epsilon_{2} \tilde{\mathcal{C}}^{2}
+ \epsilon_{3} \mathcal{C} \tilde{\mathcal{C}},
\label{eq:s_8}
\end{align}
\label{eq:action_mods}
\end{subequations}
%
where we defined the Gauss-Bonnet and Pontryagin densities, respectively,
%
\begin{subequations}
\begin{align}
    R^{2}_{\rm GB} &= R_{\mu\nu\rho\sigma}R^{\mu\nu\rho\sigma} - 4 R_{\mu\nu} R^{\mu\nu} + R^2,
    \\
    {}^{*}RR &= \tfrac{1}{2} R_{\nu\mu\rho\sigma} {}^{*}R^{\mu\nu\rho\sigma},
\end{align}
\end{subequations}
%
and where $R_{\mu\nu\rho\sigma}$ is the Riemann tensor
and $\tilde{R}_{\mu\nu\rho\sigma}$ its dual, defined as
%
\begin{equation}
\tilde{R}_{\mu\nu\alpha\beta} = \tfrac{1}{2} \epsilon_{\mu\nu\rho\sigma} R^{\rho\sigma}{}_{\alpha\beta},
\end{equation}
%
with $\epsilon_{\mu\nu\rho\sigma}$ being the Levi-Civita tensor.

This action encapsulates various modified gravity theories, which often are studied on their own,
namely,
%
\begin{itemize}
    \item scalar Gauss-Bonnet gravity (sGB): corresponds to $\alpha_{\rm GB}$
          nonzero and all other dimensionless coupling constants equal to zero.
    \item dynamical Chern-Simons (dCS): this corresponds to $\alpha_{\rm CS}$
          nonzero and all other dimensionless coupling constants equal to zero.
    \item cubic effective field theory of GR (cEFTofGR): this corresponds to $\lame$, $\lamo$ nonzero
          and all other dimensionless coupling constants equal to zero.
    \item quartic effective field theory of GR (qEFTofGR): this corresponds to $\varepsilon_{i}$
          ($i = 1,2,3$) nonzero and all other dimensionless coupling constants equal to zero.
\end{itemize}

In each of these theories, slowly-rotating black hole and their quasinormal mode spectra
to leading order in spin have been calculated.
%
We can use this results to establish a mapping between theory-specific calculations and
the theory-agnostic {\sc Parspec} parametrization.
%
We discuss how we do this in the section.

In Eqs.~\eqref{eq:action_mods}, $\alpha_{\rm GB}$, $\alpha_{\rm CS}$, $\lame$,
$\lamo$ and $\varepsilon_{i}$ are dimensionless constants.

%%%%%%%%%%%%%%%%%

%%%%%%%%%%%%%%%%%
\section{Methods}
\label{sec:methods}
%%%%%%%%%%%%%%%%%

\subsection{From theory-agnostic to theory-specific QNM results}

\begin{table*}[th]
\begin{tabular}{c c c c c c}
\hline
\hline
Theory & $p$ & $\delta \omega$ & $\delta \tau$ & QNM & Constraint \\
\hline
sGB      & 4 & \dots & \dots & \cite{Pierini:2021jxd} & $\alpha^{1/2}_{\rm GB} \ell \leqslant 1.7$~km~\cite{Perkins:2021mhb} \\
dCS      & 4 & \dots & \dots & \cite{Wagle:2021tam} & $\alpha^{1/2}_{\rm GB} \ell \leqslant 8.5$~km~\cite{Silva:2020acr} \\
cEFTofGR & 4 & \dots & \dots & \cite{Cano:2021myl} & -- \\
qEFTofGR & 6 & \dots & \dots & \cite{Cano:2021myl} & -- \\
\hline
\hline
\end{tabular}
\caption{Summary of the quasinormal modes calculations.
%
We summarize each theory we have considered together with: the exponent $p$ at
which their QNM-modification enters, the corresponding modifications to the
oscillation frequency $\delta \omega$ and decay time $\delta \tau$, the
references from which we used the results from and the current best constraint
(if applicable).
%
\hs{Add a column with the constraint (if any) that we placed.}
\hs{TODO: add which runs are done and which need to be done.}
}
\label{tab:ref_theories_qnms}
\end{table*}

\hs{We {\it must very explicitly} state our working hypothesis here:
%
\begin{itemize}
    \item That we include only the nonrotating, non-GR correction to the QNMs.
    \item That due to absence of isospectrality, when translating to the theory specific result,
    we {\it chose} to use the lowest damping, gravitational QNM of the theory.
\end{itemize}
%
We can come back to these in the Sec.~\ref{sec:discussion} as things that we
might want to revisit in the future.
}

\hs{In Table~\ref{tab:ref_theories_qnms} we summarize the relevant parameters for
the {\sc Parspec} implementation of each theory and give credit to the papers
which calculated the QNMs.}

\subsection{Gravitational events used}

\hs{Say that we use GW150914~\cite{LIGOScientific:2016aoc} and GW200129~\cite{LIGOScientific:2021djp} and justify why.}

\subsection{Parameter estimation}

%%%%%%%%%%%%%%%%%
\section{Results}
\label{sec:results}
%%%%%%%%%%%%%%%%%

\hs{Here we summarize the results of our PE runs. Combine the posteriors on the
the non-GR parameter coming from different events. We could have a short
subsection for each theory. We should quote two thresholds for claiming
that we placed a bound, one using the secondary BH mass and another with
the final BH mass.}

%%%%%%%%%%%%%%%%%
\section{Discussion}
\label{sec:discussion}
%%%%%%%%%%%%%%%%%

\hs{I think an important message of the paper is that we give indication that
there are theories of gravity (such as dCS) which can by-pass observational
constraints from the inspiral phase alone, {\it yet} they do not, if we
analyze the ringdown. This is quite important because I've frequently heard
that the ``inspiral is what will constrain theories''. I think this might be
the most important message.}

%%%%%%%%%%%%%%%%%
\section*{Acknowledgements}
\label{sec:acknowledgements}
%
We thank Emanuele~Berti, Andrea~Maselli, Deyan~Mihailov and Serguei~Ossokine
for discussions.
%
We thank the computational resources provided by the AEI, specifically the
{\sc Hypatia} cluster.
%
The authors would like to thank everyone at the frontline of the Covid-19
pandemic.
%%%%%%%%%%%%%%%%%

% \bibliographystyle{apsrev}
\bibliography{paper_alt_theor_bounds}

\end{document}
