\documentclass[twocolumn,prd,aps,superscriptaddress,preprintnumbers,tightenlines,showpacs,nofootinbib,eqsecnum,amsfonts,amsmath]{revtex4-1}
\usepackage{epsfig}
\usepackage{graphics}
\usepackage{graphicx}
\usepackage{bm}
\usepackage[dvipsnames]{xcolor}
\usepackage{bm}
\usepackage{times}
% \usepackage[varg]{txfonts}
\usepackage{txfonts}
\usepackage{xspace}
\usepackage[normalem]{ulem} % To get strikethrough (\sout)
\usepackage[colorlinks]{hyperref}
\usepackage[caption=false]{subfig}
\usepackage{booktabs}
\usepackage{url}
\usepackage{float}
\usepackage[bottom]{footmisc}
\usepackage{lineno}
%\linenumbers
\usepackage{makecell}
\usepackage{microtype}

\definecolor{LinkColor}{rgb}{0.75, 0, 0}
\definecolor{CiteColor}{rgb}{0, 0.5, 0.5}
\definecolor{UrlColor}{rgb}{0, 0, 0.75}
\hypersetup{linkcolor=LinkColor}
\hypersetup{citecolor=CiteColor}
\hypersetup{urlcolor=UrlColor}
\usepackage{perpage}
\MakePerPage{footnote}

\newcommand{\paperone}{Paper~I\xspace}

\newcommand{\h}{\mathpzc{h}}
\newcommand{\Hhat}{\hat{\mathpzc{H}}}
\newcommand{\B}{\mathpzc{B}}
\newcommand{\hlm}{\mathpzc{h}_{\ell m}}
\newcommand{\xilm}{\xi_{\ell m}}
\newcommand{\Ylm}{{Y}^{-2}_{\ell m}}
\newcommand{\Y}{{Y}^{-2}}
\newcommand{\hc}{h_\times}
\newcommand{\hp}{h_+}
\newcommand{\Fc}{F_\times}
\newcommand{\Fp}{F_+}
\newcommand{\Mf}{M_f}
\newcommand{\cA}{\mathpzc{A}}
\newcommand{\lm}{_{\ell m}}
\newcommand{\deff}{d_\mathrm{eff}}
\newcommand{\rmi}{\mathrm{i}}
\newcommand{\blambda}{\bm{\lambda}}
\newcommand{\btheta}{\bm{\theta}}
\newcommand{\bxi}{\bm{\xi}}
\newcommand{\bxigr}{\bm{\xi}_{\text{GR}}}
\newcommand{\bxingr}{\bm{\xi}_{\text{nGR}}}
\newcommand{\bzeta}{\bm{\zeta}}
\newcommand{\bs}[1]{\bm{\vec{S}_{#1}}}
\newcommand{\Mo}{M_{\odot}}
\newcommand{\FFe}{\mathrm{FF}_\mathrm{eff}}
\newcommand{\FF}{\mathrm{FF}}
\newcommand{\e}{\mathrm{e}}
\newcommand{\rhoopt}{\rho_\mathrm{opt}}
\newcommand{\rhosubopt}{\rho_\mathrm{subopt}}
\newcommand{\fqnm}{f}
\newcommand{\sigmaqnm}{\sigma}
\newcommand{\n}{\mathbf{n}}
\newcommand*{\skymapscale}{0.5}
\newcommand*{\paramestscale}{0.455}
\newcommand{\df}[1]{\delta f_{\text{#1}}}
\newcommand{\dtau}[1]{\delta \tau_{\text{#1}}}
\newcommand{\fngr}[1]{f_{\text{#1}}}
\newcommand{\taungr}[1]{\tau_{\text{#1}}}
\newcommand{\fgr}[1]{f ^{\text{GR}}_{\text{#1}}}
\newcommand{\taugr}[1]{\tau ^{\text{GR}}_{\text{#1}}}
\newcommand{\pSEOB}{\texttt{pSEOBNR}}
\newcommand{\SEOB}{\texttt{SEOBNR}}

\newcommand{\pd}{\partial}
\newcommand{\dd}{{\rm d}}
\newcommand{\dV}{{\rm d}^{4}x \, \sqrt{-g} \,}
\newcommand{\lame}{\lambda_{\rm e}}
\newcommand{\lamo}{\lambda_{\rm o}}

% Comment commands
\newcommand{\ag}[1]{{\textcolor{cyan}{{[AG: #1]}} }}
\newcommand{\hs}[1]{{\textcolor{blue}{{[HS: #1]}} }}
\newcommand{\ab}[1]{{\textcolor{green}{{[AB: #1]}} }}

\newcommand{\AEI}{\affiliation{Max Planck Institute for Gravitational Physics (Albert Einstein Institute), Am M\"uhlenberg 1, Potsdam 14476, Germany}}
\newcommand{\UMD}{\affiliation{Department of Physics, University of Maryland, College Park, Maryland 20742, USA}}

\begin{document}

\title{Super-awesome paper with super-cool results}

\author{Abhirup Ghosh}
\author{Hector O. Silva}
\AEI
\author{Alessandra Buonanno}
\AEI
\UMD

\date{\today}


%%%%%%%%%%%%%%%%%%%%%%%
\begin{abstract}
\end{abstract}
%%%%%%%%%%%%%%%%%%%%%%%

\maketitle

%%%%%%%%%%%%%%%%%
\section{Introduction}
\label{sec:intro}
%%%%%%%%%%%%%%%%%

%%%%%%%%%%%%%%%%%
\section{Methods}
\label{sec:method}

\subsection{The parametrized waveform model}
\label{sec:review_pSEOB}

\subsection{The parametrized ringdown spin expansion coefficients}
\label{sec:review_parspec}

\subsection{Overview of modified gravity theories}
\label{sec:review_theories}

A broad class modified gravity theories are capture by the action
%
\begin{align}
S &= \frac{1}{16 \pi G} \int \dd^4x \sqrt{-g}
\left[ R + \dots \right.
\nonumber \\
&\quad
\left.
\right]
\end{align}
%

\begin{table}[t]
\begin{tabular}{c c c c c}
\hline
\hline
Theory & $p$ & $\delta \omega$ & $\delta \tau$ & Refs. \\
\hline
\dots & \dots & \dots & \dots & \dots \\
\hline
\hline
\end{tabular}
\caption{Summary of theories consider in this paper.
\hs{Add a column with the constraint (if any) that we placed.}}
\label{tab:ref_stars}
\end{table}

%%%%%%%%%%%%%%%%%

%%%%%%%%%%%%%%%%%
\section{Results}
\label{sec:results}
%%%%%%%%%%%%%%%%%

%%%%%%%%%%%%%%%%%
\section{Discussion}
\label{sec:discussion}
%%%%%%%%%%%%%%%%%

%%%%%%%%%%%%%%%%%
\section*{Acknowledgements}
\label{sec:acknowledgements}
%
We thank Emanuele Berti and Andrea Maselli for discussions.
%
The authors are grateful for the computational resources provided by the AEI,
specifically the {\sc Hypatia} cluster.
%
The authors would like to thank everyone at the frontline of the Covid-19
pandemic.
%%%%%%%%%%%%%%%%%

% \bibliographystyle{apsrev}
\bibliography{paper_alt_theor_bounds}

\end{document}
