\documentclass[aps,prd,10pt,preprint,
               notitlepage,onecolumn,superscriptaddress,
               eqsecnum,
               nofootinbib,tightenlines,floatfix]{revtex4-2}


\usepackage{amsmath}
\usepackage{amssymb}
\usepackage{amsfonts}
\usepackage{mathrsfs}
\usepackage{anyfontsize}
\usepackage{mathtools}

\usepackage[utf8]{inputenc}
\usepackage[T1]{fontenc}

\usepackage[linktocpage,breaklinks]{hyperref}
\usepackage[usenames,dvipsnames]{xcolor}

% Comment this out if want to use times font
% \usepackage{txfonts}
\usepackage{tensor}
\usepackage{bm}

\usepackage{graphicx}
\usepackage{epsfig}
\usepackage{epstopdf}

\usepackage{natbib}

\hypersetup{colorlinks=true,
            citecolor=NavyBlue,
            linkcolor=NavyBlue,
            urlcolor=NavyBlue}

%%%%%%%%%%%%%%%%%%%%%%%%%%%%%%%%%%%%%%%%%%%%%%%%%%%%%%%%%%%%%%%%%%%%%%
% Add collaborators here
\newcommand{\hs}[1]{{\color{magenta}[HS: #1]}}

\newcommand{\dd}{{\rm d}}
\newcommand{\ii}{{\rm i}}
\newcommand{\xp}{x^{\prime}}
\newcommand{\rp}{r^{\prime}}
\newcommand{\be}{\begin{equation}}
\newcommand{\ee}{\end{equation}}
\newcommand{\lt}{\left}
\newcommand{\rt}{\right}
\newcommand{\pd}{\partial}
\newcommand{\df}{\coloneqq}
%%%%%%%%%%%%%%%%%%%%%%%%%%%%%%%%%%%%%%%%%%%%%%%%%%%%%%%%%%%%%%%%%%%%%%

\begin{document}
\title{Some notes on quasinormal mode tests}

\begin{abstract}
Terse.
\end{abstract}

\author{Hector O. Silva}
\email{hector.silva@aei.mpg.de}
\affiliation{Max Planck Institute for Gravitational Physics,
(Albert Einstein Institute), Am M\"uhlenberg 1, 14476 Potsdam, Germany}

\date{{\today}}

\maketitle

\tableofcontents

\section{Introduction}

Here is a collection of thoughts on QNMs tests.
%
We follow the notations used in~\cite{Ghosh:2021mrv}.

\section{Test of the Kerr metric}

\subsection{Basic idea}

The idea here is to use the post-Kerr formalism~\cite{Glampedakis:2017dvb} as a null test of
the Kerr nature of the remnant BH.
%
Let us start by reviewing what was proposed there.

According to the geometrical optics (eikonal limit)-null geodesic
correspondence, the fundamental quasinormal mode frequencies are
(see~e.g.~\cite{Ferrari:1984zz,Cardoso:2008bp}) related to the orbital frequency and Lyapunov
exponent of a null particle (or a bundle of them for the later) at the black
hole's light ring.

The idea of the post-Kerr test is to write the true Kerr quasinormal mode as
%
\begin{equation}
    \sigma_{\ell m 0}^{\rm Kerr} = \sigma_{\ell m 0}^{\rm eik.} + \beta_{\ell m 0}^{\rm off-set}\,,
    \label{eq:qnm_kerr_exact}
\end{equation}
%
where $\sigma_{\ell m 0}^{\rm eik.}$ is the result predicted by the eikonal
calculation and $\beta_{\ell m 0}^{\rm off-set}$ is a known
(\cite{Glampedakis:2017dvb}, Appendix A) off-set function that encapsulates the
mismatch between eikonal and exact quasinormal frequencies.

Next, suppose that from data we inferred a frequency $\sigma_{\ell m 0}^{\rm obs.}$
which we decompose as
%
\begin{equation}
    \sigma_{\ell m 0}^{\rm obs.} = \kappa_{\ell m 0} + \beta_{\ell m 0}^{\rm off-set}\,,
\end{equation}
%
where $\beta$ is the same off-set function appearing
in~\eqref{eq:qnm_kerr_exact} and $\kappa$ is an eikonal quasinormal frequency
whose value is determined in practice subtracting off $\beta$ from the inferred
$\sigma^{\rm obs.}$.

Then, if both the spacetime and the light ring are non-Kerr
we would expect that
%
\begin{equation}
    \sigma_{\ell m 0}^{\rm obs} - \sigma_{\ell m 0}^{\rm Kerr} = \kappa_{\ell m 0} - \sigma^{\rm eik.}_{\ell m 0} \neq 0 \,.
    \label{eq:qnm_obs}
\end{equation}
%
If we assume that the `non-Kerrness' is small, we can write
%
\begin{equation}
    \delta \sigma_{\ell m 0} \equiv  \sigma_{\ell m 0}^{\rm obs} - \sigma_{\ell m 0}^{\rm Kerr}
\end{equation}
%
where the terms in the right-hand side differ, by construction, only through their eikonal contributions.
%
The good point about this `post-Kerr' approach is that $\kappa$ can be
calculated analytically given a perturbed Kerr black hole spacetime and we can
thus quantify the amount of non-Kerr of the remnant black hole.
%
Being more specific, we can quote how different is the light ring relative to Kerr.

Let us elaborate on this last point some more. In the eikonal limit we propose:
%
\begin{equation}
    {\rm Re}[\sigma] = m (\Omega_{\rm ph} + \epsilon \delta \Omega_0)
\end{equation}
%
where $\Omega_{\rm ph}$ is the known Kerr light ring orbital frequency
%
\begin{align}
    \Omega_{\rm ph} &= \pm \frac{M^{1/2}}{r_{\rm ph}^{3/2} \pm a M^{1/2}} \,,
    \\
    r_{\rm ph} &= 2M \left\{
    1 + \cos\left[ \frac{2}{3} \cos^{-1} \left( \mp \frac{a}{M} \right)\right]
    \right\}\,,
\end{align}
%
with $r_{\rm ph}$ being the equatorial light ring for prograde (upper sign) and
retrograde (lower sign) orbits. Also, $\delta \Omega_0$ is a small deviation
which can be expressed in terms of the perturbations to the $\varphi\varphi$,
$t\varphi$ and $tt$ components of the Kerr metric.
%
\hs{I thought that we could learn something about the black hole metric
following this route, but maybe this is not worth pursuing because:
%
(i) if we were able to obtain information on the $\delta g^{\rm Kerr}$
it would still be at a single point and it seems unclear how to generically reconstructed
the deformed metric (see however~\cite{Volkel:2020daa});
%
(ii) the expressions for the post-Kerr modifications depend on quite a few
metric components (three for the real part), meaning that to break the degeneracy of
them we would need quite a number of $\sigma_{\ell m 0}$.
%
Perhaps then, it is more instructive to try to constraint some coordinate independent
parameters.}
%

\subsection{An eikonal consistency check?}

There is something simpler that could be applied right away.
\hs{Not very sure about this. Discuss.}

Suppose that from out parameter estimation we are given an
%
$\sigma_{\ell m 0}^{\rm obs.}$.
%
Further assume that we have two of such modes detected (as one would require
for a no-hair test). Since we work in the eikonal limit we do cannot predict the
overtone (see~\cite{Dolan:2009nk} for an alternative approach however), so say we know the fundamental $n=0$
quasinormal modes of multipole $\ell = m = 2$.

Then, from the decomposition~\eqref{eq:qnm_obs} we can write:
%
\begin{subequations}
\begin{align}
    2 \pi f_{220} &= 2 \left( \Omega_{\rm ph} + \epsilon \delta \Omega_{0} \right) + {\rm  Re}[\beta^{\rm off-set}_{\ell m 0}]\,,
    \\
    \tau^{-1}_{220} &= (1/2) |\gamma_{\rm ph} + \epsilon \delta \gamma_{0}| + {\rm  Im}[\beta^{\rm off-set}_{\ell m 0}]\,.
\end{align}
\end{subequations}
%
These equations can be cast in the form used in~\cite{Ghosh:2021mrv}, where $\delta f_{220}$
and $\delta \tau_{220}$ now depend on $(\delta \Omega_0,\, \delta \gamma_0)$.
%
This allows us to translate the posterior distributions of~\cite{Ghosh:2021mrv} into posteriors of more
physical observables.
%
\hs{TODO: can this be translated onto shadow properties~\cite{Yang:2021zqy}?}

Another possibility is to simply test within GR the validity of the geometrical optics / null
geodesic correspondence. The idea is as follows. First write:
%
\begin{equation}
    \sigma_{\ell m 0} = 2 \pi f_{\ell m 0} + \ii / \tau_{\ell m 0}\,.
\end{equation}
%
where in GR,
%
\begin{subequations}
\begin{align}
    2 \pi f_{\ell m 0} &= m \Omega_{\rm ph} + {\rm  Re}[\beta^{\rm off-set}_{\ell m 0}]\,,
    \\
    \tau^{-1}_{\ell m 0} &= (1/2) |\gamma_{\rm ph}|+ {\rm  Im}[\beta^{\rm off-set}_{\ell m 0}]\,,
\label{eq:re_im_eikonal}
\end{align}
\end{subequations}
%
where $\beta^{\rm off-set}_{\ell m 0} = \beta^{\rm off-set}_{\ell m 0}(M_f, \chi_f)$ are known for $\ell = m $.
%
We can then promote ($\Omega_{\rm ph}$, $\gamma_{\rm ph}$) to free parameters in the
doing parameter estimation together with $M_f(m_1, m_2, \chi_1, \chi_2)$ and $\chi_f(m_1, m_2, \chi_1, \chi_2)$.
%
\emph{If the geometrical optics / null geodesic picture is correct}:
%
\begin{itemize}
    \item from a single QNM detection we would expect that the $(\Omega_{\rm ph}, \gamma_{\rm ph})$ joint posterior contour will be consistent
    with the joint posterior obtained by transforming $(M_f, \chi_f) \mapsto (\Omega_{\rm ph}, \gamma_{\rm ph})$.
    \item from two QNM detections we would expect that the posteriors of $\gamma_{\rm ph}$ (since
    the eikonal formula is independent on $\ell$ and $m$) and of $\Omega_{\rm ph} / m$ would be consistent. (There would be three posteriors: two from each mode
    and a third from $(M_f, \chi_f) \mapsto (\Omega_{\rm ph}, \gamma_{\rm ph})$).
\end{itemize}

In practice, this is not different from the usual no-hair test, but it allows us to test a \emph{physical image
on how we interpret the QNMs}.


\subsection{Assumptions and caveats}
%
There are two caveats that we need to keep in mind:
%
\begin{itemize}
\item We assume that the off-set function $\beta$ is the same in the Kerr and
non-Kerr spacetimes. This may not be the case in general and therefore any
bound on the deviation parameters would at best \emph{be conservative} (one can
imagine that knowing exactly $\beta$ for a given theory would only strengthen
the constraints;
%
\item The quasinormal mode formulas of~\cite{Glampedakis:2017dvb} does not
capture coupling between gravitational and extra degrees of freedom which
are ubiquitous in beyond-GR theories.
%
However, one can imagine that if a beyond-GR theory is perturbatively away from
GR there is a regime where the equation almost decouple the results are a
good first approximation to be used in absence of exact quasinormal mode
frequencies calculations.
\end{itemize}

\section{Theory-specific calculations}

Here we summarize some classes of theories for which quasinormal modes
of black holes have been computed in the past. We use the acronyms NR (nonrotating),
SR (slowly-rotating) and RR (rapidly-rotating).

\begin{itemize}
    \item Einstein-dilaton-Gauss-Bonnet -- (NR)~\cite{Blazquez-Salcedo:2016enn},
        (SR)~\cite{Pierini:2021jxd}.~\hs{Having results at both NR and SR approximations
        could be useful to assess {\sc parSpec} truncated at different spin-orders}.
    \item dynamical Chern-Simons -- (NR)~\cite{Molina:2010fb}, (SR)~\cite{Wagle:2021tam,Srivastava:2021imr}.
    \item Einstein-Maxwell-dilaton -- (NR)~\cite{Ferrari:1984zz}, (SR)~\cite{Brito:2018hjh}.
    \item EFTofGR -- (NR, cubic)~\cite{deRham:2020ejn} and (NR, quadratic)~\cite{Cardoso:2018ptl}.
    \item Einstein-Aether -- (NR, odd/axial parity)~\cite{Tsujikawa:2021typ}. Analysis of quadratic Lagrangian only; no mode calculation.
\end{itemize}

\section{Effective hydrodynamical properties of exotic matter}

In~\cite{Yunes:2016jcc}, a few order-of-magnitude estimates were places on the effective
viscosity of an `exotic matter alternative to BH' was placed.
%
The starting point is a Newtonian, quasi-incompressible star with a density $\rho$,
radius $R$ and which perturbed by a spherical harmonic mode $Y_{\ell m}$.
%
From~\cite{Cutler:1987ApJ314}, the shear $\bar{\eta}$ and bulk $\bar{\zeta}$ viscosities of the oscillations are related to their
respective damping times as $\tau_{\bar{\eta}}$ and $\tau_{\bar{\zeta}}$ as
%
\begin{align}
    \bar{\eta}  &= \frac{1}{(\ell - 1) (2\ell + 1)} \frac{\rho R^2}{\tau_{\bar{\eta}}}\,,
    \\
    \bar{\zeta} &= \left(\frac{5}{4}\right)^{4} \frac{2 (2\ell +3)}{\ell^3} \frac{\rho R^2}{\tau_{\bar{\zeta}}}\,.
\end{align}
%

Given these expressions, Ref.~\cite{Yunes:2016jcc} focuses of $\ell = 2$ and eliminates the density by using
$\rho = m / [(4 \pi / 3) R^3]$, to obtain
%
\begin{align}
    \bar{\eta}_{\rm eff} &\sim 4 \cdot 10^{28} \frac{{\rm g}}{{\rm cm} \cdot {\rm s}}
    \left( \frac{m}{65 M_{\odot}} \right)
    \left( \frac{370 \, {\rm km}}{R} \right)
    \left( \frac{4 \, {\rm ms}}{\tau_{\bar{\eta}}} \right)
    \\
    \bar{\zeta}_{\rm eff} &\sim 3 \cdot 10^{30} \frac{{\rm g}}{{\rm cm} \cdot {\rm s}}
    \left( \frac{m}{65 M_{\odot}} \right)
    \left( \frac{370 \, {\rm km}}{R} \right)
    \left( \frac{4 \, {\rm ms}}{\tau_{\bar{\eta}}} \right)
\end{align}
%
where the expressions where rescalled by a fiducial damping times of $\tau = 4 ms$~\hs{Now we have posteriors on
this can could be used.}, a total mass of $m = 65~M_{\odot}$ and a radius of $R = 370~{\rm km}$ (corresponding to
the orbital separation at the end of the inspiral~\hs{Determined how?}).
%
These orders of magnitude are comparable with those of black holes (in the context of the membrane paradigm), but
larger than other compact objects such as boson stars and neutron stars. See~\cite{Yunes:2016jcc}, Table V.

\section{Others}

\subsection{Overtones}
%
Can we include overtones to the pSEOBNR~\cite{Brito:2018rfr,Ghosh:2021mrv}?

% \subsection{Isospectrality}
% %
% A firm prediction of GR is that the quasinormal modes

\subsection{QNMs and multipole moments}
%
The Refs.~\cite{Krishnendu:2017shb,Krishnendu:2019tjp} binary black hole
inspirals were used to constraint the spin-induced quadrupole moment by
parametrizing it is as $M_2 = - (1 + \delta q) \chi^2 M^3 \equiv \kappa \chi^2 M^3$ (where we follow the
notation of~\cite{Glampedakis:2017cgd}).
%
In~\cite{Glampedakis:2017cgd} the quasinormal mode in the eikonal limit were computed
for a generic ultracompact objects (such that it has a light ring).
%
The final formulas for $\sigma_{220}$ depend on two deformation parameters:
$\delta q$ (which null for Kerr) and a spin-octupole $\delta s_{3}$.
%
It seems possible to extend~\cite{Krishnendu:2017shb,Krishnendu:2019tjp}
into a `inspiral-ringdown' consistency check now that $\delta q$ can be inferred
both from the ringdown and from the inspiral. The spin-octupole would be something new.


\subsection{Final spin and mass fits}
%
In obtaining the modes in~\cite{Ghosh:2021mrv}, we final mass $M_f$ and spin
$\chi_f$ are obtained by a fit to the numerical relativity (NR) simulations, in
particular, from~\cite{Taracchini:2013rva,Hofmann:2016yih}.
%
These papers are about five years old. Can we update these fits and/or use
larger NR catalogs?
%
\hs{These references seem to have a done a very comprehensive work so the
gain-by-effort ratio is probably small. Opinions?}

Some works have followed Ref.~\cite{Buonanno:2007sv} to estimate the remnant black hole spin
in both theory specific~\cite{Jai-akson:2017ldo} and generic set-ups.
%
Since $\chi_f$ in~\cite{Ghosh:2021mrv} is predicted using fits obtained in GR
could we modify them a bit and perhaps related $\delta f$, $\delta \tau$ with
some non-GR parameter $x$ building upon~\cite{Buonanno:2007sv}?
%
Imagining that $\chi_f$ could change dramatically due to $x$;
could we use stack the posteriors on $\chi_f$ of all binary-black-hole events so far
to constraint $x$?

\section{A full parametrized inspiral-merger-ringdown template}

The waveform model used in~\cite{Brito:2018rfr,Ghosh:2021mrv} assumes that
only the merger-ringdown phase for the coalescence has significant GR deviations.
%
On the other hand, we know that the most of the waveform's SNR is contained in the
inspiral.
%
Could include changes to the inspiral as well?

It could be useful to remember that late-inspiral-ringdown waveforms which emulate
GW150914 (examined in~\cite{Ghosh:2021mrv}) are publicly available for
Einstein-dilaton-Gauss-Bonnet and dynamical-Chern-Simons
theories~\cite{Okounkova:2019zjf,Okounkova:2020rqw}.



\bibliography{biblio}

\end{document}
